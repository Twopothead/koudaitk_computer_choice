\question 动态路由选择和静态路由选择的主要区别是
\par\fourch{动态路由选择需要维护整个网络的拓扑结构信息,而静态路由选择只需要维护有限的拓扑结构信息}{\textcolor{red}{动态路由选择需要使用路由选择协议去发现和维护路由信息,而静态路由选择只需要手动配置路由信息}}{动态路由选择的可扩展性要大大优于静态路由选择,因为在网络拓扑结构发生了变化时,路由选择不需要手动配置去通知路由器}{动态路由选择使用路由表,而静态路由选择不使用路由表}
\begin{solution}从路由算法能否随网络的通信量或拓扑自适应地进行调整变化来划分,可分为两类:静态路由选择策略和动态路由选择策略。前者使用手动配置的路由信息,不能及时适应网络状态的变化。后者通过路由选择协议自动发现并维护路由信息,能及时适应网络状态的变化。
\end{solution}
