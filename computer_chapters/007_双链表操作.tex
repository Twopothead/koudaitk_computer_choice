\question (中科院,2007年)在双向链表中删除指针P所指的节点时需要修改指针( )
\par\fourch{\textcolor{red}{p→llink→rlink=p→rlink;p→rlink→llink=p→llink}}{p→llink=p→llink→llink;p→link→rlink=p}{p→rlink→llink=p;p→rlink=p→rlink→rlink}{p→rlink=p→llink→llink;p→llink=p→rlink→rlink}
\begin{solution}删除P所指的节点就是修改p所指节点的前驱的右指针域和后继节点的左指针域,这种题型在模拟操作过程中不能丢失指向某个节点的指针,只有答案A正确
\end{solution}
\question 下列说法中,正确的是( )
\ding{192}.假设某有序表的长度为n,则可以在1~(n+1)的位置上插入元素
\ding{193}.在单链表中,无论是插入还是删除操作,都必须找到其前驱结点
\ding{194}.删除双链表的中间某个结点时,只需修改两个指针域
\ding{195}.将两个各有n和m个元素的有序表(递增)归并成一个有序表,仍保持其递增有序,则最少的比较次数是m+n-1
\par\twoch{仅\ding{192}、\ding{193}、\ding{194}}{\ding{192}、\ding{193}、\ding{194}、\ding{195}}{\textcolor{red}{仅\ding{193}、\ding{194}}}{仅\ding{192}、\ding{194}、\ding{195}}
\begin{solution}\ding{192}:有序表插入的时候是不能指定位置的,因为这样可能使得插入后的表不再是有序表。正确的插入思想是:先通过元素比较找到插入的位置,再在该位置上插入,故\ding{192}错误。
\ding{193}:从单链表插入和删除的语句描述中可以看出,无论是插入还是删除操作,都必须找到其前驱结点,故\ding{193}正确。
\ding{194}:删除双链表中间某个结点时,需要修改前后两个结点的各一个指针域,共计两个指针域,故\ding{194}正确。
\ding{195}:当一个较短的有序表中所有元素均小于另一个较长的有序表中所有的元素,所需比较次数最少。假如一个有序表为1、3、4,另一个有序表为5、6、7、8、12,这样只需比较3次即可,故答案应该是n和m中较小者,即min(n,m),故\ding{195}错误。
\end{solution}
\question (四川大学,2004年)如果单链表中最常用的操作是在最后一个结点后插入一个结点和删除最后一个结点,则(
)存储方式最节省运行时间
\par\twoch{单链表}{带头结点的单链表}{单循环链表}{\textcolor{red}{带头结点的双循环链表}}
\begin{solution}单链表、带头结点的单链表、单循环链表在链表尾删除和插入操作的时间复杂度都为O(n)。
只有带头结点的双循环链表的时间复杂度为O(1)。
\end{solution}
