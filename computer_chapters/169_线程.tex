\question 下列关于线程和进程的叙述中,正确的是( )。
Ⅰ.线程包含CPU现场,可以独立执行程序 Ⅱ.每个线程都有自己独立的地址空间
Ⅲ.线程之间的通信必须使用系统调用函数 Ⅳ.线程切换都需要内核的支持
Ⅴ.线程是资源分配的单位,进程是调度和分配的单位
Ⅵ.不管系统中是否有线程,进程都是拥有资源的独立单位
\par\twoch{Ⅰ、Ⅱ、Ⅳ}{\textcolor{red}{Ⅰ、Ⅵ}}{Ⅱ、Ⅳ}{Ⅲ、Ⅵ}
\begin{solution}线程有自己的栈空间,可以独立执行程序,Ⅰ正确。
线程没有独立的地址空间,依赖于从属进程的地址空间,Ⅱ错误。
当进程有多个用户级线程时,线程之间通信不需要使用系统调用函数,在用户空间就可以进行通信,Ⅲ错误。
同一个进程下的多个用户级线程切换只需要用户空间解决,不需要内核支持,与Ⅲ类似,Ⅳ错误。
进程是系统中资源分配和调度的基本单位,线程不能作为独立的资源分配单位,因此Ⅴ错误,Ⅵ正确。
因此答案选择B选项。
★进程是资源分配的基本单位,这是与线程的主要区别。用户级线程的工作通常不会和内核有关。
\end{solution}
\question 在多对一的线程模型中,当一个多线程的进程中的某个线程被阻塞后
\par\fourch{该进程的其他线程可以继续执行}{\textcolor{red}{整个进程都将阻塞}}{该阻塞线程将被撤销}{会调度进程中某个其他线程继续执行}
\begin{solution}在多对一的线程模型中,多个用户级线程对应一个内核级线程。采用该模型的系统中,线程在用户空间进行管理,效率相对较高。但是,由于多个用户级线程映射到一个内核级线程,只要一个用户级线程阻塞,就会导致整个进程被阻塞。而且由于系统只能识别一个线程(内核级线程),因此即使有多处理机,该进程的若干个用户级线程也只能同时运行一个,不能并行执行。因此答案选择B选项。
\end{solution}
\question 系统动态DLL库中的系统线程被不同的进程所调用,它们是( )的线程
\par\twoch{不同}{\textcolor{red}{相同}}{同步}{互斥}
\begin{solution}系统线程也就是内核级线程,在多线程模型中,可以对应一个或多个用户级线程。在本题中,多个进程调用这个内核级线程,说明对应多个用户级线程。但是用户级线程有多个,对应到内核级线程时,都调用了同一个线程。
\end{solution}
