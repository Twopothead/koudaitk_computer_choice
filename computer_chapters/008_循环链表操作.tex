\question (南京理工大学,2004年)设单循环链表中节点的结构为(data,next),且rear是指向非空的带头结点的单循环链表的尾节点指针。若要删除链表的第一个节点,正确的操作是(
)
\par\fourch{s=rear;rear=rear→next;free(s);}{rear=rear→next;free(s);}{rear=rear→next→next;free(s);}{\textcolor{red}{s=rear→next→next;rear→next→next=s→next;free(s);}}
\begin{solution}考察链表的操作,rear→next指向头结点,头结点→next指向第一个节点
\end{solution}
\question (华中科技大学,2007年)某线性表用带头结点的循环单链表存储,头指针为head,当head→next→next=head成立时,线性表长度可能是(
)
\par\twoch{\textcolor{red}{1}}{2}{3}{4}
\begin{solution}为了简化,我们将循环变成单链表,即head→next→next=NULL,即头结点的下一个结点的指针为空,即除了头结点只有一个结点,头结点是不计入线性表长度的,因此该线性表的长度为1。
\end{solution}
\question (武汉大学,2000年)非空的循环单链表head的尾结点p满足( )
\par\twoch{\textcolor{red}{p→link=head}}{p→link=NULL}{p=NULL}{p=head}
\begin{solution}先考虑p=NULL,那说明p指针指向一个空对象,肯定是不符合题意的。
再看p=head,p指针指向了单链表表头,这说明链表头就是尾结点,这只有在只有一个结点的单链表中才会出现,因此也是不符合题意的。
再看p→link=NULL,p结点的指针指向一个空对象,那这个就不``循环''了,还是不符合题意。
因此只有选项A是正确的,尾结点的指针指向头结点。
【注】本题要求非空是有原因的,若为空就不成立了,因为那时就没有所谓的尾结点了,因此这是一个必须条件。
\end{solution}
