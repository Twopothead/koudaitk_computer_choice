\question (华中科技大学,2003年)在Internet域名体系中,域的下面可以划分子域,各级域名用圆点分开,按照
\par\fourch{从左到右越来越小的方式分4层排列}{从左到右越来越小的方式分多层排列}{从右到左越来越小的方式分4层排列}{\textcolor{red}{从右到左越来越小的方式分多层排列}}
\begin{solution}每个域名都由标号序列组成(各标号分别代表不同级别的域名),各标号之间用点隔开,格式如下所示:
.三级域名.二级域名.顶级域名
注意:级别最低的域名写在最左边,而级别最高的顶级域名写在最右边。
\end{solution}
\question 域名与( )是一一对应的
\par\twoch{IP地址}{MAC地址}{主机名称}{\textcolor{red}{以上都不是}}
\begin{solution}尽管DNS能够完成域名到IP地址的映射,但实际上IP地址和域名并非一一对应。如果一个主机通过两块网卡连接在两个网络上,则具有两个IP地址,但这两个IP地址可能就映射到同一个域名上。如果一个主机具有两个域名管理机构分配的域名,则这两个域名就可能具有相同的IP地址。显然,域名和MAC地址也不是一一对应的(因为一个主机可以通过两块网卡连接在两个网络上,就有两个MAC地址)。
\end{solution}
\question ( )协议不提供差错控制
\par\twoch{TCP}{UDP}{IP}{\textcolor{red}{DNS}}
\begin{solution}在TCP、UDP以及IP的首部都有校验和字段,用于提供差错控制功能,但在DNS的格式中没有校验和字段,因此DNS无法提供差错控制功能。还是要提醒一点:IP数据报只检查首部,而TCP和UDP既检查首部也检查数据部分。
\end{solution}
\question 下面有关DNS的说法中正确的是( )。
\ding{192}.主域名服务器运行域名服务器软件,有域名数据库
\ding{193}.辅助域名服务器运行域名服务器软件,但是没有域名数据库
\ding{194}.一个域名有且只有一个主域名服务器
\par\twoch{\textcolor{red}{\ding{192}、\ding{194}}}{\ding{192}、\ding{193}、\ding{194}}{\ding{192}}{\ding{193}、\ding{194}}
\begin{solution}辅助域名服务器也是有数据库的,故\ding{193}错误;\ding{192}、\ding{194}均正确。
\end{solution}
\question ( )一定可以将其管辖的主机名转换为该主机的IP地址
\par\twoch{本地域名服务器}{根域名服务器}{\textcolor{red}{授权域名服务器}}{代理域名服务器}
\begin{solution}每一个主机都必须在授权域名服务器处注册登记,授权域名服务器一定能够将其管辖的主机名转换为该主机的IP地址。
\end{solution}
\question 下列关于主机的IP地址与域名关系的描述中,正确的是( )。
\ding{192}.IP地址是域名中部分信息的表示 \ding{193}.域名是IP地址中部分信息的表示
\ding{194}.IP地址和域名是等价的 \ding{195}.IP地址和域名分别表达不同的含义
\par\twoch{\ding{193},\ding{194}}{\ding{192},\ding{195}}{\ding{193},\ding{195}}{\textcolor{red}{\ding{194}}}
\begin{solution}IP是主机在Internet中唯一识别的标志,但由于IP是无规则的数字,不适合人们的记忆,而域名容易理解、记忆,他们本质上是等价的,故人们常常通过站点的域名来访问站点。
\end{solution}
