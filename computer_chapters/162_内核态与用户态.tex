\question 若一个用户进程通过read系统调用读取一个磁盘文件中的数据,则下列关于此过程的叙述中,正确的是(
)。 Ⅰ.若该文件的数据不在内存中,则该进程进入睡眠等待状态
Ⅱ.请求read系统调用会导致CPU从用户态切换到核心态
Ⅲ.read系统调用的参数应包含文件的名称
\par\twoch{\textcolor{red}{仅Ⅰ、Ⅱ}}{仅Ⅰ、Ⅲ}{仅Ⅱ、Ⅲ}{Ⅰ、Ⅱ和Ⅲ}
\begin{solution}当用户进程读取的磁盘文件数据不在内存时,转向中断处理,导致CPU从用户态切换到核心态,此时该进程进入睡眠等待状态(其实就是阻塞态,只不过换了个说法),因此Ⅰ、Ⅱ正确。
在调用read之前,需要用open打开该文件,open的作用就是产生一个文件编号或索引指向打开的文件,之后的所有操作都利用这个编号或索引号直接进行,不再考虑物理文件名,所以read系统调用的参数不应包含物理文件名。文件使用结束后要用close关闭文件,消除文件编号或索引。
\end{solution}
\question ``访管''指令( )使用
\par\twoch{\textcolor{red}{仅在用户态下}}{仅在核心态下}{在用户态和核心态下}{在调度时间内}
\begin{solution}访管指令仅在用户态下使用,当用户需要使用只有核心态下才可以用的功能时,使用访管指令,将用户态转变为核心态。
这里容易错误地认为在核心态可以执行任何指令,当然也可以使用访管指令。访管指令的用处在于将用户态转变为核心态以执行内核函数,相当于一个敲门的作用,在核心态下,访管指令就变得无意义了,相当于进门之后就不需要再敲门了一样。
\end{solution}
