\question OSPF协议使用( )分组来保持与其邻居的连接
\par\twoch{\textcolor{red}{Hello}}{Keepalive}{SPF(最短路径优先)}{LSU(链路状态更新)}
\begin{solution}此题属于记忆性的题目,OSPF协议使用Hello分组来保持与其邻居的连接。
\end{solution}
\question 运行OSPF协议的路由器每10s向它的各个接口发送Hello分组,接收到Hello分组的路由器就知道了邻居的存在。如果在(
)秒内没有从特定的邻居接收到这种分组,路由器就认为那个邻居不存在了
\par\twoch{30}{\textcolor{red}{40}}{50}{60}
\begin{solution}OSPF规定路由器失效时间是Hello分组间隔时间的4倍,所以选B。
\end{solution}
\question 关于OSPF,下列说法正确的是( )。
\ding{192}.OSPF的每个区域(area)运行路由选择算法的一个实例
\ding{193}.OSPF路由器向各个活动端口组播Hello分组来发现邻居路由器
\ding{194}.Hello协议还用来选择指定路由器,每个区域选出一个指定路由器
\ding{195}.OSPF默认的路由更新周期为30s
\par\twoch{\textcolor{red}{\ding{192}、\ding{193}、\ding{194}}}{\ding{192}、\ding{193}、\ding{195}}{\ding{192}、\ding{194}、\ding{195}}{\ding{193}、\ding{194}、\ding{195}}
\begin{solution}OSPF使用Hello数据报发现邻居并建立邻接,所以\ding{193}正确。Hello数据报以组播形式每隔一定时间(HelloInterval)发送一次。在不同网络中,HelloInterval的时间也不同,所以\ding{195}错误,事实上RIP默认的路由更新周期为30s。
\end{solution}
\question 关于OSPF,下面的描述中正确的是( )。 \ding{192}.OSPF是一种链路状态协议
\ding{193}.OSPF使用链路状态通告扩散路由信息 \ding{194}.OSPF网络中用区域1来表示主干网段
\ding{195}.OSPF路由器中可以配置多个路由进程
\par\twoch{\ding{192}、\ding{193}、\ding{194}}{\textcolor{red}{\ding{192}、\ding{193}、\ding{195}}}{\ding{192}、\ding{194}、\ding{195}}{\ding{193}、\ding{194}、\ding{195}}
\begin{solution}OSPF是一个性能优越的、开放的、适合大型网络规模的路由协议。链路状态通告(Link
State
Advertisements,LSA)是一种OSPF数据报,包含可在OSPF路由器间共享的链路状态和路由信息。OSPF使用LSA来扩散路由信息。
区域是OSPF进行路由的划分单位。每个AS都有若干个区域,但是有且仅有一个骨干区域,称为Area
0(0号区域)。所有的区域都要连接到Area
0,并且区域之间的路由信息交换也要通过Area 0。
\end{solution}
