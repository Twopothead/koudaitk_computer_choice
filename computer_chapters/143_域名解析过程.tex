\question ( )协议不提供差错控制
\par\twoch{TCP}{UDP}{IP}{\textcolor{red}{DNS}}
\begin{solution}在TCP、UDP以及IP的首部都有校验和字段,用于提供差错控制功能,但在DNS的格式中没有校验和字段,因此DNS无法提供差错控制功能。还是要提醒一点:IP数据报只检查首部,而TCP和UDP既检查首部也检查数据部分。
\end{solution}
\question (华中科技大学)有一个域名分析方式,它要求域名服务器系统一次性完成全部域名---地址变换,这种分析方式叫做
\par\twoch{\textcolor{red}{递归查询}}{本地查询}{远程查询}{迭代查询}
\begin{solution}递归查询是指本地域名服务器只需向根域名服务器查询一次,后面的几次查询都是在其他几个域名服务器之间进行的,所以这种分析方式叫做递归查询。迭代查询是每次请求一个服务器,如果不行再请求别的服务器。
\end{solution}
\question 当客户端请求域名解析时,如果本地DNS服务器不能完成解析,就把请求发送给其他服务器,当某个服务器知道了需要解析的IP地址,把域名解析结果按原路返回给本地DNS服务器,本地DNS服务器再告诉客户端,这种方式称为
\par\twoch{迭代解析}{\textcolor{red}{递归解析}}{迭代与递归解析相结合}{高速缓存解析}
\begin{solution}概念题
\end{solution}
\question 一台主机希望解析域名www.abc.com,如果这台主机配置的DNS地址为A(或称为本地域名服务器),Internet根域名服务器为B,而存储域名www.abc.com与其IP地址对应关系的域名服务器为C,那么这台主机通常先查询
\par\twoch{\textcolor{red}{域名服务器A}}{域名服务器B}{域名服务器C}{不确定}
\begin{solution}当一台主机发出DNS查询报文时,这个查询报文首先被送往该主机的本地域名服务器。当本地域名服务器不能立即回答某个主机的查询时,该本地域名服务器就以DNS客户的身份向某一台根域名服务器查询。若根域名服务器也没有该主机的信息(但此时根域名服务器一定知道该主机的授权域名服务器的IP地址),有以下两种做法:
1)递归查询:根域名服务器向该主机的授权域名服务器发送DNS查询报文,查询结果再逐级返回给原主机。
2)递归与迭代相结合的方法(迭代查询):根域名服务器把授权域名服务器的IP地址返回给本地域名服务器,由本地域名服务器再去查询。
\end{solution}
