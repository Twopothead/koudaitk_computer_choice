\question 下列关于DRAM和SRAM的说法中,错误的是
\ding{192}.SRAM不是易失性存储器,而DRAM是易失性存储器
\ding{193}.DRAM比SRAM集成度更高,因此读写速度也更快
\ding{194}.主存只能由DRAM构成,而高速缓存只能由SRAM构成
\ding{195}.与SRAM相比,DRAM由于需要刷新,所以功耗较高
\par\twoch{\ding{193}、\ding{194}和\ding{195}}{\ding{192}、\ding{194}和\ding{195}}{\ding{192}、\ding{193}和\ding{194}}{\textcolor{red}{\ding{192}、\ding{193}和\ding{194}和\ding{195}}}
\begin{solution}D。
SRAM和DRAM都属于易失性存储器,掉电就会丢失,\ding{192}错误;SRAM的集成度虽然更低,但速度更快,因此通常用于高速缓存Cache,\ding{193}错误;主存可以用SRAM实现,只是成本高,\ding{194}错误;和SRAM相比,DRAM成本低、功耗低,但需要刷新,\ding{195}错误。
\end{solution}
\question (中国科学院,2001年)若动态RAM每毫秒必须刷新100次,每次刷新需100ns,一个存储周期需要200ns,则刷新占存储器总操作时间的百分比是
\par\twoch{0.5\%}{1.5\%}{\textcolor{red}{1\%}}{2\%}
\begin{solution}C。
其实这道题告诉存储周期是多余的,只需取一个参考时间,这里取1ms即可。1ms中用来刷新的时间为100×100ns=10000ns,因此刷新占存储器总操作时间的百分比是10\^{}4ns/10\^{}6ns=1\%。
\end{solution}
\question 某容量为256MB的存储器由若干4M*8位DRAM芯片构成,该DRAM芯片的地址引脚和数据引脚总数是(
)
\par\twoch{\textcolor{red}{19}}{22}{30}{36}
\begin{solution}本题考察DRAM半导体存储芯片的结构,很多不熟悉DRAM工作原理的考生很有可能得出22+8=30的结论。因为4M的地址空间需要22根地址线来标识,加上8根数据线。其实这是不对的,考生在平时可能很少接触到关于DRAM的引脚复用的知识点。我们知道半导体存储芯片的译码驱动方式有线选法和重合法,动态RAM是由许多基本存储元按照行和列地址引脚复用来组成的。我们要选中一个单元就需要行地址(11位)加上列地址(11位),现在采用引脚复用,只需11位,但需要两次发送地址(行地址和列地址)。故11+8=19。
【总结】关于DRAM的引脚复用还是第一次考到,考生记住此知识点即可。另外,此题存在迷惑性,很多考生稍不注意就以为考察的是存储器的扩充,导致误选,题目中问的是DRAM芯片的引脚,因此与存储器无关。
\end{solution}
