\question 在采用SPOOLing技术的系统中,用户暂时未能打印的数据首先会被送到(
)存储起来
\par\twoch{\textcolor{red}{磁盘固定区域}}{内存固定区域}{终端}{打印机}
\begin{solution}采用SPOOLing技术的系统中,用户的打印数据首先由内存经过缓冲区传递至输出井暂存,等输出设备(打印机)空闲时再将输出井中的数据经缓冲区传递到输出设备上。而输出井通常是在磁盘上开辟的一块固定存储区。
\end{solution}
\question 利用虚拟设备达到I/O要求的技术是指
\par\fourch{\textcolor{red}{利用外存作缓冲,将作业与外存交换信息与物理设备交换信息两者独立起来,并使它们并行工作的过程}}{把I/O要求交给多个物理设备分散完成的过程}{把I/O信息先存放在外存,然后由一台物理设备分批完成I/O要求的过程}{把共享设备改为某个作业的独享设备,集中完成I/O要求的过程}
\begin{solution}通过虚拟技术将一台独占设备虚拟成多台逻辑设备,供多个用户进程同时使用,通常把这种经过虚拟的设备称为虚拟设备。如虚拟光驱、虚拟网卡就是虚拟设备。
\end{solution}
\question 下面关于虚拟设备的叙述中正确的是
\par\fourch{虚拟设备允许用户使用比系统中拥有的物理设备更多的设备}{虚拟设备允许用户以标准化方式来使用物理设备}{\textcolor{red}{虚拟设备把一个物理设备变换成多个对应的逻辑设备}}{虚拟设备允许用户程序不必全部装入内存就可以使用系统中的设备}
\begin{solution}虚拟设备是指使用虚拟技术把独占设备改造成多用户共享的设备,也就要把一台具体的物理设备变换成若干个逻辑设备。
\end{solution}
\question SPOOLing技术的主要目的是
\par\fourch{提高CPU和设备交换信息的速度}{提高主、辅存接口}{减轻用户的编程负担}{\textcolor{red}{提高独占设备的利用率}}
\begin{solution}SPOOLing技术是低速输入/输出设备与主机交换的一种技术,通常也称为``假脱机真联机''。它的核心思想是以联机的方式得到脱机的效果。低速设备经通道和外设在主机内存的缓冲存储器与高速设备相联,该高速设备通常是辅存。为了存放从低速设备上输入的信息,或者存放将要输出到低速设备上的信息(来自内存),在辅存分别开辟一固定区域,叫``输出井''(对输出)或者``输入井''(对输入)。简单来说就是在内存中形成缓冲区,在高级设备形成输出井和输入井,传递的时候,从低速设备传入缓冲区,再传到高速设备的输入井,再从高速设备的输出井传到缓冲区,再传到低速设备。
显然它是为了提高设备利用率而使用的技术。
\end{solution}
\question 下列关于I/O设备的论述正确的是
\par\fourch{在现代计算机系统中,只有I/O设备才是有效的中断源}{在中断处理过程中,必须屏蔽中断(即禁止发生新的中断)}{\textcolor{red}{同一用户所使用的I/O设备也可以并行工作}}{SPOOLing是将多台物理I/O设备虚拟为一台逻辑I/O设备}
\begin{solution}A选项错误,按引起中断的原因划分:输入、输出中断,计算机故障中断,实时时钟中断,软件中断。
B选项错误,不一定。 C选项正确,I/O设备间当然可以并行工作。
D选项错误,SPOOLing技术是将一台物理I/O设备虚拟为多台逻辑I/O设备,同样允许多个用户共享一台物理I/O设备。
\end{solution}
\question SPOOLing系统的输入井和输出井表示
\par\twoch{磁盘上的两个存储器}{内存中的两个缓冲区}{输入设备和输出设备}{\textcolor{red}{存放用户的输入数据和输出数据的外存空间}}
\begin{solution}SPOOLing系统主要有以下三部分:
(1)输入井和输出井。这是在磁盘上开辟的两个大存储空间。
(2)输入缓冲区和输出缓冲区。这是在内存中开辟的两个缓冲区。
(3)输入进程和输出进程。
\end{solution}
\question 下列说法正确的有( )。
\ding{192}.如果I/O所花费的时间比CPU处理时间短的多,则缓冲区最有效
\ding{193}.提高单机资源利用率的最关键技术是SPOOLing技术
\ding{194}.在采用SPOOLing技术的系统中,用户的打印数据首先被送到内存固定区域
\ding{195}.在操作系统中,用户在使用I/O设备时,通常采用物理设备名
\par\twoch{\ding{192}、\ding{194}}{\ding{193}、\ding{194}}{\ding{194}}{\textcolor{red}{都错}}
\begin{solution}\ding{192}错误:缓冲区主要解决I/O速度比CPU处理的速度慢而造成的数据积压的矛盾,所以如果I/O花费的时间比CPU处理时间短的多,则缓冲区就没有必要设置。
\ding{193}错误:在单级系统中,最关键的资源就是处理机资源,最大化地提高处理机利用率,就是最大化地提高系统效率。多道程序设计技术是提高处理机利用率的最关键技术。
\ding{194}错误:打印数据先存入输出井,再送入打印机,输出井位于磁盘而不是内存。
\ding{195}错误:采用的是逻辑设备名,不是物理设备名。 综上所述,本题选择D选项。
\end{solution}
\question 下面关于SPOOLing系统的说法中,正确的是
\par\fourch{构成SPOOLing系统的基本条件是有大量内存作为输入井与输出井}{构成SPOOLing系统的基本条件是要有大容量、高速度的硬盘作为输入井和输出井}{当输入设备忙时,SPOOLing系统中的用户程序暂停执行,待I/O空闲时再被唤醒执行输出操作}{\textcolor{red}{SPOOLing系统中的用户程序可以随时将输出数据送到输出井中,待输出设备空闲时再由SPOOLing系统完成数据的输出操作}}
\begin{solution}构成SPOOLing系统的基本条件时要有大容量、高速度的外存作为输入井和输出井,因此A、B选项不对,同时利用SPOOLing技术提高了系统和I/O设备的利用率,进程不必等待I/O操作的完成,因此C选项也不对。
注意:
外储存器是指除计算机内存及CPU缓存以外的储存器,此类储存器一般断电后仍然能保存数据。常见的外储存器有硬盘、软盘、光盘、U盘等。所以B选项是不对的,除了硬盘外,还可以用其他高速的外存储器设备作为输入井和输出井。
\end{solution}
\question (西安电子科技大学,2000年)在关于SPOOLing的叙述中,( )描述不正确
\par\fourch{SPOOLing系统中必须使用独占设备}{SPOOLing系统加快了作业执行的速度}{SPOOLing系统使独占设备变成了共享设备}{\textcolor{red}{SPOOLing系统利用了处理器与通道并行工作的能力}}
\begin{solution}SPOOLing是操作系统中采用的一种将独占设备改造为共享设备的技术,它有效减少了进程等待读入/读出信息的时间,加快了作业的执行速度。不过,无论有没有通道,SPOOLing系统都可以运行,因此D选项是不对的。
\end{solution}
