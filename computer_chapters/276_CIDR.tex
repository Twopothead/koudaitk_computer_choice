\question 设有两个子网202.118.133.0/24和202.118.130.0/24,如果进行路由汇聚,得到的网络地址是
\par\fourch{\textcolor{red}{202.118.128.0/21}}{202.118.128.0/22}{202.118.130.0/22}{202.118.132.0/20}
\begin{solution}前两个字节和最后一个字节不做比较了,只比较第三个字节。 129→10000001
130→10000010 132→10000100 133→10000101
显然,这4个数字只有前5位是完全相同的,因此汇聚后的网络的第三个字节应该是10000000→128。汇聚后的网络的掩码中``1''的数量应该有8+8+5=21,因此答案是202.118.128.0/21。
\end{solution}
\question 以下给出的地址中,属于子网192.168.15.19/28的主机地址是( ~)
Ⅰ.192.168.15.17 ~ ~ Ⅱ.192.168.15.14 Ⅲ.192.168.15.16 ~ ~
Ⅳ.192.168.15.31
\par\twoch{\textcolor{red}{仅Ⅰ}}{仅Ⅰ、Ⅱ}{仅Ⅰ、Ⅲ、Ⅳ}{Ⅰ、Ⅱ、Ⅲ和Ⅳ}
\begin{solution}将IP地址192.168.15.19和28位子网掩码255.255.255.240分别转换为二进制后进行与运算,可以得到该子网的网络地址192.168.15.16。由于网络位是28位,所以主机位占4位,24=16,即该子网段包括16个IP地址。也就是说地址范围是192.168.15.16~192.168.15.31,但16是网络地址,31是广播地址,因此只有192.168.15.17在此范围之内。
\end{solution}
\question (重庆大学)IP地址191.201.0.158的标准子网掩码是
\par\fourch{255.0.0.0}{\textcolor{red}{255.255.0.0}}{255.255.255.0}{255.255.255.255}
\begin{solution}191.201.0.158是一个B类地址,其默认子网掩码是255.255.0.0。
\end{solution}
\question (华东理工大学,2006年)如IP地址为120.14.22.16,掩码为255.255.128.0,则子网地址是
\par\fourch{120.0.0.0}{\textcolor{red}{120.14.0.0}}{120.14.022.0}{22.16}
\begin{solution}将IP地址和子网掩码都转换成二进制数,并按位进行``与''操作即可得到子网地址。IP地址转换成二进制为01111000
00001110 00010110 00010000,子网掩码为11111111 11111111 10000000
0000000,二者相与后得到子网地址为01111000 00001110 00000000
00000000,即为120.14.0.0。
\end{solution}
