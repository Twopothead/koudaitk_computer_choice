\question (上海交通大学,1997年)在3种集中式总线控制方式中,(
)方式对电路故障最敏感
\par\twoch{\textcolor{red}{链式查询}}{计数器定时查询}{独立请求}{分组链式查询}
\begin{solution}链式查询对询问链的电路故障很敏感,如果第i个设备的接口中有关链的电路有故障,那么第i个以后的设备都不能进行工作。
\end{solution}
\question 总线的判优控制主要解决( )的问题
\par\twoch{\textcolor{red}{由哪个主设备占用总线}}{通信双方如何获知传输开始和结束}{通信过程中双方如何协调配合}{B和C}
\begin{solution}总线控制包括判优控制和通信控制。其中判优控制主要解决当多个主设备请求占用总线时,决定由它们当中的哪一个来占用。
\end{solution}
\question 下列关于总线仲裁方式的说法中,正确的有( )。
\ding{192}.独立请求方式响应时间最快,是以增加处理器开销和增加控制线数为代价的
\ding{193}.计数器定时查询方式下,有一根总线请求(BR)和一根设备地址线,若每次计数都从0开始,则设备号小的优先级高
\ding{194}.链式查询方式对电路故障最敏感
\ding{195}.分布式仲裁控制逻辑分散在总线各部件中,不需要中央仲裁器
\par\twoch{\ding{194}和\ding{195}}{\textcolor{red}{\ding{192}、\ding{194}和\ding{195}}}{\ding{192}、\ding{193}和\ding{195}}{\ding{193}、\ding{194}和\ding{195}}
\begin{solution}理解3种仲裁方式的原理、线数和优缺点。
独立请求方式每个设备均由一对总线请求线和总线允许线,总线控制逻辑复杂,但响应速度快,\ding{192}正确。
计数器定时方式采用一组设备地址线(log2(n)),\ding{193}错误。
链式查询方式对硬件电路故障敏感,且优先级不能改变,\ding{194}正确。
分布式仲裁方式不需要中央仲裁器,每个主模块都有自己的仲裁号和仲裁器,多个仲裁器竞争使用总线,\ding{195}正确。
\end{solution}
\question 在计数器定时查询方式下,正确的描述是
\par\twoch{\textcolor{red}{总线设备的优先级可变}}{越靠近控制器的设备,优先级越高}{各设备的优先级相等}{对硬件电路故障敏感}
\begin{solution}在知识点中讲过,计数器的初值可以由程序设置,故优先级次序可以改变。B和D选项是链式查询的特点。C要看情况,计数器每次判优都从``0''开始,此时一旦设备的优先顺序被固定后,设备的优先级就按0,1,2,
,n的顺序降序排列,永远不能改变;计数器也可以从上一次的终点开始计数,即是一种循环方法,此时所有设备使用总线的优先级相等。
\end{solution}
\question 以下叙述中错误的是
\par\fourch{\textcolor{red}{总线结构传送方式可以提高数据的传输速度}}{与独立请求方式相比,链式查询方式对电路的故障更敏感}{总线标准化后,使得在计算机中增删设备非常容易,提高了设备的兼容性和互换性}{总线的带宽是总线本身所能达到的最高传输速率}
\begin{solution}A错误,总线结构传送方式并不能提高数据的传输速度。
B正确,链式查询方式的缺点就是对询问链的电路故障很敏感,如果第i个设备的接口中有关链的电路有故障,那么第i个以后的设备都不能进行工作。
C正确。总线标准化后,使得在计算机中增删设备非常容易,提高了设备的兼容性和互换性,因此I/O总线和通信总线大多是标准化总线。
D正确。总线带宽指总线的最大数据传输率,即在数据传输阶段单位时间内总线上可传输的数据量。
\end{solution}
\question 在链式查询方式下,若有N个设备,则
\par\twoch{\textcolor{red}{只需一条总线请求线}}{需要N条总线请求线}{视情况而定,可能一条,可能N条}{以上说法都不对}
\begin{solution}链式查询下,一共有3条控制线:总线请求线、总线忙线和总线同意线,故不管有多少设备,只有一条总线请求线。
\end{solution}
