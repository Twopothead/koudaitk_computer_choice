\question 下列储存介质中,既可随机访问又可顺序访问的是( )。 Ⅰ.光盘 Ⅱ.SD卡
Ⅲ.U盘 Ⅳ.磁盘
\par\twoch{Ⅱ、Ⅲ和Ⅳ}{Ⅰ、Ⅲ和Ⅳ}{\textcolor{red}{Ⅰ、Ⅱ、Ⅲ和Ⅳ}}{只有Ⅳ}
\begin{solution}顺序访问:是按从前到后的顺序对数据进行读写操作。这种存取方式最为简单。有的存储设备(如磁带)只能支持顺序访问(平时用磁带听歌的时候,如果要听另外一首歌曲,一般都是快进的,也就是顺序访问过去)。
随机访问:也称为直接访问,可以按任意的次序对数据进行读写操作。现在多数存储设备能支持随机访问(如光盘、磁盘、U盘、闪存卡等),当然这些存储设备也一定支持顺序访问。
\end{solution}
\question 下列关于ROM和RAM的说法中,错误的是
Ⅰ.CD-ROM是ROM的一种,因此只能写入一次
Ⅱ.Flash快闪存储器属于随机存取存储器,具有随机存取的功能
Ⅲ.RAM的读出方式是破坏性读出,因此读后需要再生
Ⅳ.SRAM读后不需要刷新,而DRAM读后需要刷新
\par\twoch{Ⅰ和Ⅱ}{Ⅰ、Ⅲ和Ⅳ}{Ⅱ和Ⅲ}{\textcolor{red}{Ⅰ、Ⅱ和Ⅲ}}
\begin{solution}D。
CD-ROM属于光盘存储器,是一种机械式的存储器,和ROM有本质的区别,Ⅰ错误。
Flash存储器是E\^{}2PROM的改进产品,虽然它也可以实现随机存取,但从原理上讲仍属于ROM,而且RAM是易失性存储器,Ⅱ错误。
SRAM的读出方式并不是破坏性的,读出后不需再生,Ⅲ错误。
SRAM采用双稳态触发器来记忆信息,因此不需要再生(刷新);而DRAM采用电容存储电荷的原理来存储信息,只能维持1~2ms,因此即使电源不掉电,信息也会自动消失,因此在电荷消失之前必须要再生(刷新),Ⅳ正确。
\end{solution}
\question (哈尔滨工程大学,2003年)( )存储结构对程序员是透明的
\par\twoch{通用寄存器}{主存}{\textcolor{red}{控制寄存器}}{堆栈}
\begin{solution}C。
通用寄存器、主存、堆栈都是可以被程序员编程操作的,而控制寄存器是控制器的内部结构,只关乎计算机系统的设计人员,对程序员是透明的。
\end{solution}
\question 下列关于PROM与掩模ROM的说法错误的是
\par\twoch{掩模ROM制成后不可改写}{PROM制成后可编写一次,编程之后不可再改写}{\textcolor{red}{PROM制成后可多次改写}}{以上说法都不对}
\begin{solution}C。 PROM(Programmable Read-Only
Memory)------可编程只读存储器,也叫One-Time
Programmable(OTP)ROM(一次可编程只读存储器),是一种可以用程序操作的只读内存。其最主要特征是只允许数据写入一次,如果数据烧入错误只能报废。
注:可擦除的ROM的名字中都带E,如EPROM、EEPROM等,还有一个是Flash
Memory。
\end{solution}
\question 下列说法中错误的是
\par\fourch{EEPROM可电擦,即只需要在特定引脚上加上规定的电压即可擦除}{\textcolor{red}{EEPROM改写时,首先要擦除所有信息,然后重新写入}}{EEPROM写入周期需要几毫秒,远远大于SRAM、DRAM的写入周期}{以上说法都不对}
\begin{solution}B。 EEPROM的读/写操作可按位或按字节进行,写入前不需要擦除全部内容。
\end{solution}
