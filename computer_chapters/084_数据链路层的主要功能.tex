\question (华中理工大学)不属于数据链路层协议考虑的范畴是
\par\twoch{控制对物理传输介质的访问}{相邻节点间的可靠数据传输}{为终端节点隐藏物理拓扑的细节}{\textcolor{red}{定义数据格式}}
\begin{solution}定义数据格式是由表示层来完成的。其他3个选项都是数据链路层协议应该考虑的范畴。
\end{solution}
\question 对于信道比较可靠并且对通信实时性要求高的网络,采用(
)数据链路层服务比较合适
\par\twoch{\textcolor{red}{无确认的无连接服务}}{有确认的无连接服务}{有确认的面向连接的服务}{无确认的面向连接的服务}
\begin{solution}无确认的无连接服务是指源机器向目标机器发送独立的帧,目标机器并不对这些帧进行确认。事先并不建立逻辑连接,事后也不用释放逻辑连接。若由于线路上有噪声而造成了某一帧丢失,则数据链路层并不会检测这样的丢帧现象,也不会恢复。当错误率很低的时候,这一类服务是非常适合的,这时恢复过程可以留给上面的各层来完成。这类服务对于实时通信也是非常适合的,因为实时通信中数据的迟到比数据损坏更加不好。
\end{solution}
\question 假设物理信道的传输成功率是95\%,而平均一个网络层的分组需要10个数据链路层的帧来发送。如果数据链路层采用了无确认的无连接服务,那么发送网络层分组的成功率是(
)
\par\twoch{40\%}{\textcolor{red}{60\%}}{80\%}{95\%}
\begin{solution}要成功发送一个网络层的分组,需要成功发送10个数据链路层帧。成功发送10个数据链路层帧的概率是\includegraphics[width=0.55208in,height=0.19792in]
{texmath/cb7e655Cdpi7B3507D28095295E7B107D}
≈0.598,即大约只有百分之六十的成功率。
这个结论说明了在不可靠的信道上无确认的服务效率很低。为了提高可靠性应该引入有确认的服务。
\end{solution}
