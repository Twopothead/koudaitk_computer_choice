\question 下列关于操作系统的论述中,正确的是( )
\par\fourch{\textcolor{red}{对于批处理作业,必须提供相应的作业控制信息}}{对于分时系统,不一定全部提供人机交互功能}{从响应角度看,分时系统与实时系统的要求相似}{在采用分时操作系统的计算机系统中,用户可以独占计算机操作系统中的文件系统}
\begin{solution}分时系统必须有交互功能,实时系统对于响应的要求比分时系统更高。在分时系统中,用户不会独占文件系统,这是多用户共享的。
\end{solution}
\question 多道程序设计技术能有效提高系统的吞吐量和改善资源利用率,实现这技术需要考虑下列(
)方面的问题。 Ⅰ.如何分配处理机、内存和I/O设备
Ⅱ.如何提高系统安全性,保证不被黑客攻击 Ⅲ. 如何组织不同类型的作业
Ⅳ.如何减少单个作业占用的内存大小,以载入更多的作业
\par\twoch{Ⅱ、Ⅲ、Ⅳ}{\textcolor{red}{Ⅰ、Ⅲ}}{Ⅰ、Ⅳ}{Ⅱ、Ⅲ}
\begin{solution}多道程序设计的目的在于提高系统的吞吐量,改善资源利用率。由于主存中同时存在多个作业,因此要考虑以下几个方面的问题:
应如何分配处理机,以使处理机既能满足各程序运行的需要又有较高的利用率,将处理机分配给某程序后,应何时收回等问题。
如何为每道程序分配必要的内存空间,使它们各得其所又不致因相互重叠而失去信息,应如何防止因某个程序出现异常情况而破坏其他程序等问题。
系统中可能有多种类型的I/O设备供多道程序共享,应如何分配这些I/O设备,如何做到既方便用户对设备的使用,又能提高设备的利用率。
在现在计算机系统中,通常都存放着大量的程序和数据。应如何组织它们才能便于用户使用,并能保证数据的安全性和一致性。
对于系统中的各种应用程序,它们有的属于计算型,有的属于I/O型,有些作业既重要又紧迫,有的作业又要求系统能及时响应,这时应如何组织这些作业。
从上述总结可以看出,选项中Ⅰ、Ⅲ项都是多道程序设计需要考虑的问题;虽然安全性也是多道程序设计要考虑的一个因素,但是这个安全性是指用户在使用数据时不会因为系统组织数据的方法而出现数据不一致或者被错误修改等情况,与题目中所说的黑客攻击不同,网络安全并不是多道程序要考虑的问题,因此Ⅱ不对;同样,如何减少作业的大小也不是多道程序设计要考虑的,而是在设计作业的时候要考虑的,系统只负责执行作业而不会考虑如何设计作业,因此Ⅳ错误。答案选择B选项。
\end{solution}
