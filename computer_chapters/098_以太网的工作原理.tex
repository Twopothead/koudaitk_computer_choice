\question 在以太网中,将以太网地址映射为IP地址的协议是
\par\twoch{SMTP}{ARP}{HTTP}{\textcolor{red}{RARP}}
\begin{solution}A选项为邮件发送协议,B选项为IP地址转为以太网地址的协议,C选项为访问网页的协议。
\end{solution}
\question (重庆大学)10BASE-5标准中的``5''代表的是
\par\twoch{5类双绞线}{第五种标准}{传输速率为5Mbit/s}{\textcolor{red}{最大作用距离为500m}}
\begin{solution}IEEE802.3支持的物理层介质和配置方式有多种,是由一组协议组成的。每一种实现方案都有一个名称代号,由以下3部分组成:
\{数据传输率(Mbit/s)\}\{信号方式\}\{最大段长度(百米)或介质类型\}
最后一部分若是数字的话,表示最大传输距离。若是字母的话,则第一个字节表示介质类型,第二个字母表示工作方式。
\end{solution}
\question (华中理工大学,2003年)IEEE
802.3标准规定,若采用同轴电缆作为传输介质,在无中继器的情况下,传输介质的最大长度不能超过
\par\twoch{\textcolor{red}{500m}}{200m}{100m}{50m}
\begin{solution}IEEE802.3标准定义了两种采用同轴电缆作为传输介质的以太网标准,一种是10Base-2,通常称为细缆以太网,它采用50Ω细同轴电缆,在无中继器的情况下最大传输距离为185m;另一种是10Base-5,常称为粗缆以太网,在无中继器的情况下最大传输距离为500m。
\end{solution}
\question IEEE802.3ae 10Gbit/s以太网标准支持的工作模式是
\par\twoch{\textcolor{red}{全双工}}{半双工}{单工}{全双工和半双工}
\begin{solution}10Gbit/s以太网使用光纤作为传输介质,最大传输距离为40km,只支持全双工数据传输,而不支持半双工工作模式。
\end{solution}
