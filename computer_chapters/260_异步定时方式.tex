\question 关于异步通信中的半互锁方式,下列说法错误的是
\par\fourch{\textcolor{red}{主模块发出请求信号后,不必等待从模块的回答信号即可撤销}}{从模块发出回答信号后,不必等待主模块的请求信号即可撤销}{主模块的请求信号与从模块的回答信号有简单的制约关系}{半互锁方式的工作速度慢于不互锁方式,快于全互锁方式}
\begin{solution}半互锁方式特点:主模块的请求信号和从模块的回答信号有简单的制约关系。即主模块发出请求信号后,必须接到从模块的回答信号后才撤销请求信号,有互锁的关系。而从模块接到请求信号后,发出回答信号,但不必等待获知主模块的请求信号已经撤销,而是隔一段时间自动撤销回答信号,不存在互锁关系。
\end{solution}
\question 在手术过程中,医生将手伸出,等护士将手术刀递上,待医生握紧后,护士才松手。如果把医生和护士看做两个通信模块,上述动作相当于
\par\twoch{同步通信}{\textcolor{red}{异步通信的全互锁方式}}{异步通信的半互锁方式}{异步通信的无互锁方式}
\begin{solution}医生伸出手(即发出请求信号)必须等到护士递上手术刀(等到回答信号),护士也必须等待医生握紧后才可以松手(从模块等待主模块的回答信号),以上整个流程就是异步通信的全互锁方式。
\end{solution}
