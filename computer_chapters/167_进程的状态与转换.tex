\question 假设系统中总共有n个进程存在,则阻塞队列中进程的个数最多有( )个
\par\twoch{n+1}{\textcolor{red}{n}}{n-1}{1}
\begin{solution}本题是看似简单的易错题,需要仔细考虑,顾全所有的情况。答案应该是n。本题极易错认为n个进程应该有一个进程被分配CPU运行,剩下最多n-1个进程在阻塞队列中,而且如果就绪队列中有进程的话,阻塞队列中的进程还将少于n-1个。但考虑到另一种情况,那就是死锁,如果这n个进程由于争夺资源而产生死锁,那么就有n个进程全在阻塞队列中等待相互间的资源的释放,没有执行中和就绪的进程。
如果题目改为问就绪队列最多有多少个,则答案就变为n-1了。一个计算机系统中至少含有一个处理机,也就是说总会有一个请求执行的进程得到处理机,那假设所有进程都提出执行申请,则其中一个得到满足,剩余的n-1个进程插入就绪队列,此时就绪的进程最多,个数为n-1。
\end{solution}
\question (电子科技大学,2005年)某单处理器计算机系统中若同时存在5个进程,则处于执行状态的进程最多可以有
\par\twoch{0个}{\textcolor{red}{1个}}{4个}{5个}
\begin{solution}在某一时刻,单处理器计算机系统中只能有一个进程处于执行状态。
\end{solution}
\question (西北工业大学,1999年)当( )时,进程从执行状态转变为就绪状态
\par\twoch{进程被调度程序选中}{\textcolor{red}{时间片到}}{等待某一事件}{等待的事件发生}
\begin{solution}A情况发生时,进程由就绪状态转变为执行状态。
C情况发生时,进程由执行状态转变为等待状态。
D情况发生时,进程由等待状态转变为就绪状态。
只有B情况发生时,进程才由执行状态转变为就绪状态。
\end{solution}
\question 一个进程的读磁盘操作完成后,操作系统针对该进程必做的是( )
\par\twoch{\textcolor{red}{修改进程状态为就绪态}}{降低进程优先级}{为进程分配用户内存空间}{增加进程的时间片大小}
\begin{solution}进程读磁盘操作完成,即完成I/O操作,那么该进程就会由阻塞状态转为就绪状态。因此操作系统针对该进程必做的是修改进程状态为就绪态。其他三个操作都没有发生依据,可排除。
\end{solution}
\question 下列选项中会导致进程从执行态变为就绪态的事件是()
\par\twoch{执行P(wait)操作}{申请内存失败}{启动I/O 设备}{\textcolor{red}{被高优先级进程抢占}}
\begin{solution}进程间各状态的转化。
\end{solution}
