\question 控制总线主要用来传送( )。 \ding{192}.存储器和I/O设备的地址码
\ding{193}.所有存储器和I/O设备的时序信号 \ding{194}.所有存储器和I/O设备的控制信号
\ding{195}.来自I/O设备和存储器的响应信号
\par\twoch{\ding{193}、\ding{194}}{\ding{192}、\ding{194}、\ding{195}}{\ding{194}、\ding{195}}{\textcolor{red}{\ding{193}、\ding{194}、\ding{195}}}
\begin{solution}存储器和I/O设备的地址码应该由地址线来传送,故\ding{192}错误;控制线应该用来传送来自存储器和I/O设备的时序信号、控制信号、响应信号。
\end{solution}
\question 系统总线中的控制总线的主要功能是
\par\fourch{\textcolor{red}{提供定时信号、操作命令和各种请求/回答信号等}}{提供数据信息}{提供时序信号}{提供主存和I/O模块的回答信号}
\begin{solution}控制总线中传输的控制信号包括总线命令、定时信号(时钟和握手信号等)、总线请求、总线允许、中断请求和中断允许等。因此A选项的描述最全,本题选A。
\end{solution}
\question 总线上的信息的传输总是由
\par\twoch{CPU启动}{总线控制器启动}{\textcolor{red}{总线主设备启动}}{总线从设备启动}
\begin{solution}发起总线请求并在获得总线使用权后能控制总线的设备。如CPU、DMA控制器等都可以作为主设备。
\end{solution}
\question 下列有关总线定时的叙述中,错误的是()
\par\fourch{异步通信方式中,全互锁协议最慢}{\textcolor{red}{异步通信方式中,非互锁协议的可靠性最差}}{同步通信方式中,同步时钟信号可由多设备提供}{半同步通信方式中,握手信号的采样由同步时钟控制}
\begin{solution}考察了总线操作和定时,主要是同步定时与异步定时的定义及其特点。
\end{solution}
