\question CD-ROM
,光盘只读存储器,一种能够存储大量数据的外部存储媒体,一张压缩光盘的直径大约是4.5英寸,1/8英寸厚,能容纳约660兆字节的数据。CD-ROM的光道是
\par\twoch{位记录密度不同的同心圆}{位记录密度相同的同心圆}{位记录密度相同的螺旋线}{\textcolor{red}{位记录密度不同的螺旋线}}
\begin{solution}CD-ROM记录在母盘上的数据呈螺旋状,由中心向外散开,磁盘表面有许许多多微小的坑,那就是记录的数字信息,且位记录密度不同,故本题选D。
\end{solution}
\question 某磁盘的转速为10000转/分,平均寻道时间是6ms,磁盘传输速率是20MB/s,磁盘控制器延迟为0.2ms,读取一个4KB的扇区所需的平均时间约为(
)
\par\twoch{9ms}{\textcolor{red}{9.4ms}}{12ms}{12.4ms}
\begin{solution}首先,计算等待时间。由于磁盘的转速是10000转/分钟,于是可得转一圈的时间为6ms,于是可以得到平均等待时间为3ms;其次,由于磁盘传输速率为20MB/s,所以读取4KB的信息需要0.2ms。于是,可以得到读取一个4KB的扇区所需的平均时间约为:平均等待时间+寻道时间+读取时间+控制器延迟时间=3ms+6ms+0.2ms+0.2ms=9.4ms。
\end{solution}
