\question 总线设计中采用复用传输方式的目的在于
\par\twoch{提高总线的传输带宽}{\textcolor{red}{减少总线中信号线的数量}}{增加总线的功能}{简化总线协议}
\begin{solution}总线设计中,为了减少布线,即减少总线中信号线的数量,常将数据总线和地址总线采用多路复用方式传输。即不同的信号共用一组信号线,分时传送。
\end{solution}
\question 假定一个同步总线的工作频率为33MHz,总线中有32位数据线,每个总线时钟传输一次数据,则该总线的最大数据传输率为
\par\twoch{66MB/s}{\textcolor{red}{132 MB/s}}{526 MB/s}{1056 MB/s}
\begin{solution}一次传输的数据大小为32bit=4B。
一次所花费的时间为一个时钟周期,即1/33MHz。
那么数据传输率=4B/(1/33MHz)=132 MB/s。故本题选B。
\end{solution}
\question 设一个32位微处理器配有16位的外部数据总线,若时钟频率为100MHz,总线周期为5个时钟周期传输一个字,则总线带宽是
\par\twoch{4MB/s}{\textcolor{red}{40MB/s}}{16MB/s}{64MB/s}
\begin{solution}总线带宽就是总线的数据传输率。(处理器的位数与总线带宽无关,是干扰条件)
解法一:
时钟频率为100MHz,所以时钟周期=1/100s,总线周期=5个时钟周期=5×0.01s,总线工作频率=1/0.05s=20MHz,则总线带宽=20MHz×(16/8)B=40MB/s。
解法二:
时钟频率为100MHz,总线传输周期为5个时钟周期,则总线工作频率=时钟频率/5=
20MHz,总线带宽=总线工作频率×(总线宽度/8)=20MHz×(16/8)B=40MB/s。
\end{solution}
\question 总线的数据传输速率可按公式Q=W×F/N计算,其中Q为总线数据传输率,W为总线数据宽度(总线位宽/8),F为总线时钟频率,N为完成一次数据传送所需的总线时钟周期个数。若总线位宽为16位,总线时钟频率为8MHz,完成一次数据传送需2个总线时钟周期,则总线数据传输速率Q为
\par\twoch{16Mb/s}{8Mb/s}{16MB/s}{\textcolor{red}{8MB/s}}
\begin{solution}W=16/8B=2B,N=2,F=8MHz,Q=2B×8MHz/2=8MB/s。
\end{solution}
\question 假设某存储器总线采用同步通信方式,时钟频率为50MHz,每个总线事务以突发方式传输8个字,以支持块长为8个字的cache行读和cache行写,每字4字节。对于读操作,方式顺序是1个时钟周期接收地址,3个时钟周期等待存储器读数,8个时钟周期用于传输8个字。请问若全部访问都为读操作,该存储器的数据传输率为
\par\twoch{114.3MB/s}{126.0MB/s}{\textcolor{red}{133.3MB/s}}{144.3MB/s}
\begin{solution}一次总线事务传输的数据量为8×4B=32B。
所使用的时钟周期数为1+3+8=12,又每个时钟周期为(1/50MHz),那么所使用的总时间为12×(1/50MHz)。
那么读操作的数据传输率为32B/(12×(1/50MHz))=133.3MB/s。
\end{solution}
\question (北京科技大学,2008年)总线宽度与下列哪个选项有关
\par\twoch{控制线根数}{地址线根数}{\textcolor{red}{数据线根数}}{以上都不对}
\begin{solution}总线宽度是指总线上能够同时传输的数据位数,通常用每秒钟传送信息的字节数来衡量。
\end{solution}
\question (武汉理工大学,2005年)在单机系统中,三总线结构计算机的总线系统组成是
\par\fourch{片内总线、系统总线和通信总线}{数据总线、地址总线和控制总线}{\textcolor{red}{系统总线、内存总线和I/O总线}}{ISA总线、VESA总线和PCI总线}
\begin{solution}A选项为总线按照功能分类;B选项为总线按照传送信息进行分类;D选项为总线标准。
\end{solution}
