\question (北京理工大学)关于数据报通信子网,下述方法( )是正确的
\par\fourch{第一个分组需要路由选择,其他分组沿着选定的路由传输}{第一个分组的传输延迟较大,其他分组的传输延迟很小}{\textcolor{red}{每个分组都包含源端和目的端的完整地址}}{分组可以按顺序、正确地传递给目的站点}
\begin{solution}在数据报服务中网络随时接受主机发送的分组(即数据报),网络为每个分组独立地选择路由。网络尽最大努力将分组交付给目的主机,但网络对源主机没有任何承诺。网络不保证所传送的分组不丢失,也不保证按源主机发送分组的先后顺序以及在时限内必须将分组交付给目的主机。
\end{solution}
\question (重庆大学)以下没有采用存储转发机制的交换方式是( )
\par\twoch{\textcolor{red}{电路交换}}{报文交换}{分组交换}{信元交换}
\begin{solution}在电路交换中,电路建立后中间节点不需要存储直接转发。
\end{solution}
\question 下列哪些是虚电路的特点?( ) \ding{192}.传输前建立逻辑连接 \ding{193}.分组按序到达
\ding{194}.分组开销小 \ding{195}.分组单独选择路由
\par\twoch{\textcolor{red}{仅\ding{192}、\ding{193}、\ding{194}}}{仅\ding{193}、\ding{194}}{仅\ding{192}、\ding{194}、\ding{195}}{仅\ding{192}、\ding{193}、\ding{195}}
\begin{solution}首先,由于虚电路传输分组前需要建立一条逻辑连接,属于同一条虚电路的分组按照同一路由转发,所以分组一定是按序到达的。其次,由于目的地址仅在建立虚电路的时间使用,所以之后每个分组使用长度较短的虚电路号放在分组首部即可。相比数据报服务来讲,虚电路分组开销小。综上所述,\ding{192}、\ding{193}、\ding{194}属于虚电路的特点。
\end{solution}
\question (南京师范大学)根据报文交换的基本原理,可以将其交换系统的功能概括为(
)
\par\twoch{存储系统}{转发系统}{\textcolor{red}{存储-转发系统}}{传输-控制系统}
\begin{solution}在报文交换中,不需要两个站之间建立一条专用通路。如果一个站想要发送一个报文(信息的一个逻辑单元),只需要把一个目的地址附加在报文上,然后报文通过网络从节点到节点进行传送。在每个节点中,接收整个报文,暂存整个报文,然后发送到下一个节点。
\end{solution}
\question (浙江工商大学)下列有关交换技术的论述,正确的是( )
\par\fourch{电路交换要求在通信的双方之间建立一条实际的物理通路,但在通信过程中,这条通路可以与别的通信方式共享}{现有的公用数据网都采用报文交换技术}{报文交换可以满足实时或交互式的通信要求}{\textcolor{red}{分组交换将一个大报文分割成分组,并以分组为单位进行存储转发,在接收端再将各分组重新装成一个完整的报文}}
\begin{solution}采用电路交换时用户独占分配的传输线路或信道带宽,而不是共享,故A选项错误;现有的公用数据网都采用分组交换,而不是报文交换技术,故B选项错误;采用报文交换时,由于数据进入交换节点后要经历存储、转发这一过程,从而引起转发时延(包括接收报文、检验正确性、排队、发送时间等),故不能满足实时或交互式的通信要求,故C选项错误。D选项是对分组交换正确的描述。
\end{solution}
\question 下列说法错误的是( )。 \ding{192}.在传输数据之前,电路交换先要建立一条逻辑连接
\ding{193}.采用报文交换时,报文的内容不一定按顺序到达目的结点
\ding{194}.电路交换比分组交换的实时性好 \ding{195}.报文交换方式适合交互式通信
\par\twoch{仅\ding{192}、\ding{193}、\ding{194}}{\textcolor{red}{仅\ding{192}、\ding{193}、\ding{195}}}{仅\ding{193}、\ding{194}、\ding{195}}{仅\ding{195}}
\begin{solution}\ding{192}:电路交换建立的是一条物理连接,而虚电路建立的才是逻辑连接,故\ding{192}错误。
\ding{193}:报文交换是中间结点对整个报文存储转发,报文内容一定是顺序到达,故\ding{193}错误。
\ding{194}:电路交换无须存储转发,属于直通的,故比分组交换的实时性好,故\ding{194}正确。
\ding{195}:报文交换在中间结点需要对整个报文进行存储转发,导致延时较长,因此不适合交互式通信,故\ding{195}错误。
\end{solution}
\question 市话网在数据传输期间,在源结点与目的结点之间有一条利用中间结点构成的物理通路。这种市话网采用(
)技术
\par\twoch{报文交换}{\textcolor{red}{电路交换}}{分组交换}{数据交换}
\begin{solution}电路交换:由于电路交换在通信之前要在通信双方之间建立一条被双方独占的物理通路(由通信双方之间的交换设备和链路逐段连接而成),故市话网采用的是电路交换。
\end{solution}
\question (重庆邮电大学,2007年)下列哪一项不是虚电路的特点?( )
\par\twoch{分组按照同一路由转发}{顺序到达}{分组开销少}{\textcolor{red}{支持广播}}
\begin{solution}虚电路方式具有以下特点:
(1)用户之间通信必须建立连接,数据传输过程中不再需要寻找路径,相对数据报方式时延较小。
(2)通常分组走同样的路径,所以分组一定是按序到达目的主机的。
(3)分组首部并不包含目的地址,而是包含虚电路标识符,相对数据报方式开销小。
(4)当某个交换机或某条链路出现故障而彻底失效时,所有经过该交换机或该链路的虚电路将遭到破坏。
注意:虚电路不支持组播和广播。
\end{solution}
\question (中南大学)下列哪种交换技术可能导致失序?( )
\par\twoch{电路交换}{报文交换}{虚电路分组交换}{\textcolor{red}{数据报分组交换}}
\begin{solution}在数据报分组交换中,同一报文中的分组可能由不同的传输路径通过通信子网而到达目的地,因此可能导致失序。
\end{solution}
\question 虚电路属于( )
\par\twoch{电路交换}{报文交换}{\textcolor{red}{分组交换}}{混合交换}
\begin{solution}分组交换被分为两种:面向连接的虚电路方式和无连接的数据报方式。
\end{solution}
