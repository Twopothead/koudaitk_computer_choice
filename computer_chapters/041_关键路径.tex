\question (华南理工大学,2007年)在下列网中,( )是边不带权值的图
\par\twoch{邮电图}{\textcolor{red}{AOV网}}{公路网}{AOE网}
\begin{solution}AOV网只是反映各个活动之间的先后关系,因此边是不带权值的。
\end{solution}
\question (北京航空航天大学,2002年)若某带权图为G=(V,E),其中V=\{v1,v2,v3,v4,v5,v6,v7,v8,v9,v10\},E=\{(v1,v2)5,(v1,v3)6,(v2,v5)3,(v3,v5)6,(v3,v4)3,(v4,v5)3,(v4,v7)1,(v4,v8)4,(v5,v6)4,(v5,v7)2,(v6,v10)4,(v7,v9)5,(v8,v9)2,(v9,v10)2\}(注:顶点偶对右边的数据表示边上的权值),则G的关键路径的长度为(
)
\par\twoch{19}{20}{\textcolor{red}{21}}{22}
\begin{solution}只要按已知信息画出图,即可得出关键路径长度为21。有两条路径,分别为:
v1、v3、v5、v7、v9、v10和v1、v3、v4、v5、v7、v9、v10。
\end{solution}
\question (南京理工大学,1998年)下面关于求关键路径的说法中,不正确的是( )
\par\fourch{求关键路径是以拓扑排序为基础的}{一个事件的最早开始时间同以该事件为尾的弧的活动最早开始时间相同}{\textcolor{red}{一个事件的最迟开始时间为以该事件为尾的弧的活动最迟开始时间与该活动的持续时间的差}}{关键活动一定位于关键路径上}
\begin{solution}关键路径求最早发生时间需要先拓扑排序,求最迟发生时间需要逆拓扑排序,因此A正确。
B叙述正确,C中应该是``以该事件为尾的弧的活动最迟完成时间与该活动的持续时间的差''。
D正确,即便存在多条关键路径,其上的活动也均是关键活动。
\end{solution}
\question (华南理工大学,2005年)对AOE网的关键路径,下面的说法( )是正确的
\par\fourch{提高关键路径上的一个关键活动的速度,必然使整个工程缩短工期}{完成工程的最短时间是从始点到终点的最短路径的长度}{一个AOE网的关键路径中有一条,但关键活动可有多个}{\textcolor{red}{任何一项活动持续时间的改变都可能会影响关键路径的改变}}
\begin{solution}A错误,不一定,当存在多条关键路径时,一条关键路径上的关键活动速度提高了,不一定能缩短整个工程的工期。正确的叙述是:``缩短所有关键路径上共有的任意一个关键活动的持续时间,必然使整个工程缩短工期''。
B错误,是最长路径。 C错误,可以有多条关键路径。 D正确,当然可能。
【小技巧】看到``可能''、``某些''就找正例来证明它,看到``必然''、``任意''、``所有''就找个负例来否定它。
\end{solution}
\question (中国矿业大学,2004年)在AOE网络中,顶点表示( )
\par\twoch{时间}{\textcolor{red}{事件}}{活动}{次序}
\begin{solution}考察AOE的概念,AOE是边表示活动,顶点表示所有入边活动到此完成的事件的网络
\end{solution}
\question (南京林业大学,2005年)关键路径是AOE网中( )
\par\twoch{从源点到汇点的最短路径}{\textcolor{red}{从源点到汇点的最长路径}}{最长的回路}{最短的回路}
\begin{solution}考察关键路径的概念,关键路径是整个工期所完成的时间,也就是从源点到汇点的最长路径长度
\end{solution}
