\question (南京理工大学,1999年)对序列15,9,7,8,20,-1,4用希尔排序方法排序,经一趟后序列变为15,-1,4,8,20,9,7则该次采用的增量是(
)
\par\twoch{1}{\textcolor{red}{4}}{3}{2}
\begin{solution}根据希尔排序的特点,分别尝试增量1、2、3,看子序列是否有序,依次排除A、D、C,因此本题选B。
本题也可用选项推测法,若A成立,则B、C、D都成立;若D成立,则B成立;然后就判断增量3和4即可。
\end{solution}
\question (北京师范大学,2004年)用某种排序方法对线性表\{24,88,21,48,15,27,69,35,20\}进行排序时,元素序列的变化情况如下:
(1)24,88,21,48,15,27,69,35,20 (2)20,15,21,24,48,27,69,35,88
(3)15,20,21,24,35,27,48,69,88 (4)15,20,21,24,27,35,48,69,88
则所采用的排序方法是( )
\par\twoch{\textcolor{red}{快速排序}}{选择排序}{希尔排序}{归并排序}
\begin{solution}如果是选择排序,则在4轮排序过程中无法得到最后的排序结构。如果是希尔排序不可能在第一步将20换到第一位。同理也不是归并排序。这4次过程中是子序列同时进行的快速排序
\end{solution}
