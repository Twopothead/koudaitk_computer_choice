\question 一个磁盘的转数为7200r/min,每个磁道有160个扇区,每扇区有512B,那么理想情况下,其数据传输是(
)
\par\twoch{7200×160KB/s}{7200KB/s}{\textcolor{red}{9600KB/s}}{19200KB/s}
\begin{solution}磁盘的转速为7200r/min=120r/s,转一圈经过了160个扇区,每个扇区为512B,所以数据传输率=120r/s×(160×512)B×1024=9600KB/s。
这题唯一需要注意的是单位的转换。
\end{solution}
\question 设磁盘的转速为3000转/min,盘面划分成10个扇区,则读取一个扇区的时间为
\par\twoch{20ms}{5ms}{\textcolor{red}{2ms}}{1ms}
\begin{solution}磁盘转速为3000r/min,每转的时间为1/3000min,即20ms,每个盘面有10个扇区,因此访问一个扇区的时间是2ms。
\end{solution}
\question 下列关于存储器的论述错误的是
\par\fourch{虚拟盘是一种易失性存储器,因此它通常只用于存放临时文件}{优化文件物理块的分布可显著地减少寻道时间,因此能有效地提高磁盘I/O的速度}{\textcolor{red}{对于随机访问的文件,可通过提前读提高对数据的访问速度}}{延迟写可减少启动磁盘的次数,因此能等效地提高磁盘I/O的速度}
\begin{solution}A选项正确,所谓``虚拟盘''是指用计算机的随机存储器(RAM)部分来模拟一个硬盘驱动器,执行一般的文件存储操作。真正的硬盘和虚拟盘两者间最重要区别就是虚拟盘仅存在于内存中,当关机或重新启动计算机时,虚拟盘上的信息将被丢失,所以在关机或重新启动计算机前,一定要及时把在虚拟盘上的重要的数据存放到真正的硬盘中。
B选项正确。
C选项错误,既然是随机访问的文件,那么其局部性就很差了,提前读必然是无效的。
D选项正确。延迟写是指把要写的计算机数据先都放到内存里,等积累多了再一次性写到硬盘,降低对硬盘的读写损耗。
\end{solution}
