\question 流量控制实际上是对( )的控制
\par\twoch{发送方、接收方数据流量}{接收方数据流量}{\textcolor{red}{发送方数据流量}}{链路上任意两结点间的数据流量}
\begin{solution}流量控制就是要控制发送方发送数据的速率,使接收方来得及接收。
\end{solution}
\question 流量控制是为防止( )所需要的
\par\twoch{位错误}{发送方缓冲区溢出}{\textcolor{red}{接收方缓冲区溢出}}{接收方与发送方间冲突}
\begin{solution}流量控制就是要控制发送方发送数据的速率,使接收方来得及接收,目的就是怕接收方缓冲区溢出。
\end{solution}
\question 若数据链路的发送窗口大小为6,现在正在发送4号帧,当发送方接到3号帧的确认帧后,发送方还可连续发送的帧数是
\par\twoch{2}{3}{4}{\textcolor{red}{5}}
\begin{solution}发送方接收到3号帧的确认,说明到3号帧为止的所有帧都接收了,分析知发送方只有4号帧还未收到确认帧,而窗口大小为6,故发送方还可继续发送5个帧。
\end{solution}
\question 下列哪一项正确地描述了流量控制?
\par\twoch{一种管理有限带宽的方法}{一种同步连接两台主机的方法}{\textcolor{red}{一种确保数据完整性的方法}}{以上说法均不正确}
\begin{solution}流量控制用于防止在端口阻塞的情况下丢帧,这种方法是当发送或接收缓冲区开始溢出时通过将阻塞信号发送回源地址实现的。流量控制可以有效地防止由于网络中瞬间的大量数据对网络带来的冲击,保证用户网络高效而稳定的运行。描述最准确的答案应该是C选项。
\end{solution}
\question 下列哪一项最能描述窗口的大小?
\par\fourch{软件允许并能迅速处理数据的窗口的最大值}{\textcolor{red}{等待一个确认时能传送的信息量}}{为使数据能发送,必须提前建立的窗口大小}{监视程序打开的窗口大小,它并不等于监视程序的大小}
\begin{solution}窗口大小是``等待一个确认时能传送的信息量''。
\end{solution}
\question 主机甲与主机乙之间使用后退N帧协议(GBN)传输数据,甲的发送窗口尺寸为1000,数据帧长为1000字节,信道带宽为100Mbps,乙每收到一个数据帧立即利用一个短帧(忽略其传输延迟)进行确认。若甲乙之间的单向传播时延是50
ns,则甲可以达到的最大平均数据传输速率约为( )
\par\twoch{10Mbps}{20Mbps}{\textcolor{red}{80Mbps}}{100Mbps}
\begin{solution}本题考察的是最大传输速率的计算,同时结合了滑动窗口协议中GBN协议的考察,题目问的是最大平均数据传输速率,那么我们就应当从如何达到最大这个临界点出发去考虑,根据GBN的规则,发送方可以连续发送数据帧,接收方每接收到一个数据帧之后都要进行确认,从发送第一个帧的第一个比特开始到发送方接收到第一帧确认帧往返时延是100ms,而发送方要发送完1000帧的数据需要80ms,所以在这100ms内是可以发完这1000帧,故最大平均数据传输速率为:1000×1000×8/100ms=80Mbps。
【总结】关于平均最大速率的计算是每年的重头戏,重点考察考生对于发送时延、传播时延以及对于最大吞吐量的临界点的把握,同时还要结合具体的协议分析。以上为什么是接收到第一帧的确认才是最大的临界点?(如果发送方发完1000帧之后,还没有收到对第一帧的确认,那么这段时间内,发送方还是需要等待的,尽管你发送速率确实很快,但是求最大平均传输速率的时候,这段等待时间还是要加上的,所以当你再怎么提升发送速率,平均数据传输速率是不会变的。那么为什么要等待?这就是GBN滑动窗口的机制了,在没有收到第一帧的确认帧的时候,发送窗口是不能滑动的,自然就不能再发送数据了。)只要接收到第一帧的确认,发送窗口就可以向后滑动一帧,又可以继续发送了,这也就是为什么等接收到第一帧确认才是最大临界点的原因)。
\end{solution}
