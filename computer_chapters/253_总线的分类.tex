\question 内部总线(又称片内总线)是指
\par\fourch{\textcolor{red}{CPU内部连接各寄存器及运算部件之间的总线}}{CPU和计算机系统的其他高速功能部件之间互相连接的总线}{多个计算机系统之间互相连接的总线}{计算机系统和其他系统之间互相连接的总线}
\begin{solution}如在CPU内部,寄存器之间和算术逻辑部件ALU与控制部件之间传输数据所用的总线称为片内总线(即芯片内部的总线)。
\end{solution}
\question (西北工业大学,2006年)在(
)的运算器中需要在ALU的两个输入端加上两个缓冲寄存器
\par\twoch{\textcolor{red}{单总线结构}}{双总线结构}{三总线结构}{都需要加}
\begin{solution}对于单总线结构,因为任一时刻只能有一个操作数在单总线上传输,为了把两个操作数输入到ALU,需要分两次来做,且需要两个缓冲寄存器。
\end{solution}
