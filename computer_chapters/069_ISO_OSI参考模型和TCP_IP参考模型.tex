\question 在OSI参考模型的划分原则中,下列叙述不正确的是
\par\fourch{\textcolor{red}{网络中各结点可以有不同的层次}}{不同的结点相同的层具有相同的功能}{同一结点内相邻的层之间通过接口通信}{每层使用下层提供的服务}
\begin{solution}OSI参考模型将整个通信功能划分为7个层次,其原则:不同的结点都有相同的层次;不同结点的同等层具有相同的功能;同一结点相邻层之间通过接口通信;每一层使用下层提供的服务,并向其上层提供服务;不同结点的同等层按照协议实现对等层之间的通信。
\end{solution}
\question 在OSI参考模型中,直接为会话层提供服务的是( )
\par\twoch{应用层}{表示层}{\textcolor{red}{传输层}}{网络层}
\begin{solution}本题考察OSI的七层参考模型的概念,OSI七层参考模型从下往上依次是物理层、数据链路层、网络层、传输层、会话层、表示层、应用层,相邻的两层下层为上层提供服务。所以直接为会话层提供服务的是传输层。
【总结】关于网络中的七层模型、五层模型、四层模型。
七层模型:物理层、数据链路层、网络层、传输层、会话层、表示层、应用层;
五层模型(TCP/IP分层模型):物理层、数据链路层、网络层、传输层、应用层;(教科书中采用此分层模型,此分层方法只是为了理解主机工作的原理,实际的因特网分四层,见下);
四层模型(因特网分层模型):网络接口层、网间层、传输层、应用层;
其中七层参考模型是考试的一个热点:OSI七层模型是一种框架性的设计方法
,建立七层模型的主要目的是为解决异构网络互连时所遇到的兼容性问题,其最主要的功能就是帮助不同类型的主机实现数据传输。它的最大优点是将服务、接口和协议这三个概念明确地区分开来,通过七个层次化的结构模型使不同的系统、不同的网络之间实现可靠的通讯。
\end{solution}
\question 当一台计算机从FTP服务器下载文件时,在该FTP服务器上对数据进行封装的5个转换步骤是
\par\fourch{\textcolor{red}{数据、报文、IP分组、数据帧、比特流}}{数据、IP分组、报文、数据帧、比特流}{报文、数据、数据帧、IP分组、比特流}{比特流、IP分组、报文、数据帧、数据}
\begin{solution}应用层是用户的数据,数据+传输层首部=报文;报文+网络层首部=IP分组;IP分组+数据链路层首部与尾部=数据帧;数据帧到了物理层变成比特流。
注意:物理层是不参与封装的,物理层以0、1比特流的形式透明地传输数据链路层递交的帧。网络层、应用层都把上层递交的数据加上首部,数据链路层接收上层递交的数据不但要给其加上首部,还要加上尾部。
\end{solution}
\question 以下说法错误的是( )。
Ⅰ.广播式网络一般只包含3层,即物理层、数据链路层、网络层
Ⅱ.Internet的核心协议是TCP/IP
Ⅲ.在Internet中,网络层的服务访问点是端口号
\par\twoch{Ⅰ、Ⅱ和Ⅲ}{只有Ⅲ}{\textcolor{red}{Ⅰ、Ⅲ}}{Ⅰ、Ⅱ}
\begin{solution}由于广播式网络并不存在路由选择问题,故没有网络层,故Ⅰ错误;Internet的核心协议是TCP/IP协议,故Ⅱ正确;在Internet中,网络层的服务访问点是IP地址,传输层的服务访问点是端口号,故Ⅲ错误。
\end{solution}
\question (中南大学)属于表示层的功能是
\par\twoch{交互管理}{透明传输}{死琐管理}{\textcolor{red}{文本压缩}}
\begin{solution}表示层除了有数据的加密和解密等功能外,还有文本压缩以及数据格式转换等功能。
\end{solution}
