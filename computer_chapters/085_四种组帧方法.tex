\question (华东理工大学)帧头和帧尾都使用01111110标志,数据块作为位流来处理,这种传输方案称为
\par\twoch{面向字符的同步传输}{异步传输}{\textcolor{red}{面向位的同步传输}}{起止式传输}
\begin{solution}所谓面向字符就是将帧看成一个字符序列,在链路上所传送的数据必须由规定字符集(如ASCII字符集)中的字符组成(对应ASCII字符集即传输ASCII码),在链路上传送的控制信息也必须由同一个字符集中的若干指定的控制字符构成。最典型的面向字符的协议是IBM的BSC协议(二进制同步通信)。所谓面向比特就是将帧看成一个位序列(即将传输信息看做0、1组成的序列),在帧头和帧尾加上同步标志。
\end{solution}
\question PPP填充方式所面向的数据单元是
\par\twoch{分组}{帧}{比特}{\textcolor{red}{字符}}
\begin{solution}考查PPP和HDLC协议的区别。PPP是一种面向字节型的协议,或者说是面向字符型的协议;而HDLC协议是面向比特型的协议。
\end{solution}
\question 若一个PPP帧的数据部分是7D 5E AB 7D 5D 7D 5E(十六进制),则真正的数据是
\par\twoch{\textcolor{red}{7E AB 7D 7E}}{AB 7D 5E}{AB 7D 5D}{7D 5D 7E}
\begin{solution}PPP异步传输时,把信息字段中出现的每一个0x7E字节转变为两个字节序列0x7D
0x5E,每一个0x7D字节转变为两字节序列0x7D 0x5D,现在转变后的数据为7D 5E
AB 7D 5D 7D 5E,只需要把0x7D 0x5E转为0x7E,把0x7D
0x5D转为0x7D,最后可得真实的数据为7E AB 7D 7E。
\end{solution}
\question (北京邮电大学,2005年)下述协议中,( )不是数据链路层的标准
\par\twoch{\textcolor{red}{ICMP}}{HDLC}{PPP}{SLIP}
\begin{solution}网际控制报文ICMP是网络层协议。PPP是在SLIP基础上发展而来的,都是数据链路层协议。
\end{solution}
\question 根据HDLC帧中控制字段前两位的取值,可将HDLC帧划分为3类,这3类包括( )。
\ding{192}.信息帧 \ding{193}.无编号帧 \ding{194}.监控帧 \ding{195}.确认帧
\par\twoch{\textcolor{red}{\ding{192},\ding{193},\ding{194}}}{\ding{192},\ding{193},\ding{195}}{\ding{192},\ding{194},\ding{195}}{\ding{193},\ding{194},\ding{195}}
\begin{solution}HDLC中有信息帧(I帧)、监控帧(S帧)和无编号帧(U帧)等3种不同类型的帧。
\end{solution}
\question 若HDLC帧的数据段中出现比特串``011111100111110111'',则比特填充后的输出为
\par\fourch{\textcolor{red}{01111101001111100111}}{01111111001111100111}{01111111001111110111}{01111101001111110111}
\begin{solution}HDLC帧使用零比特填充以实现透明传输,即发送时每碰到5个1在后面加一个0,接收时每收到5个1,去掉后面紧跟的0。
\end{solution}
\question HDLC协议对01111100 01111110组帧后对应的比特串为( )
\par\fourch{\textcolor{red}{01111100 00111110 10}}{01111100 01111101 01111110}{01111100 01111101 0}{01111100 01111110 01111101}
\begin{solution}HDLC协议对比特串进行组帧时,HDLC数据帧以位模式0111
1110标识每一个帧的开始和结束,因此在帧数据中凡是出现了5个连续的``1''的时候,就会在输出的位流中填充一个``0''。
\end{solution}
