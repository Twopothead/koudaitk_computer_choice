\question (重庆邮电大学,2007年)单模光纤的传输距离比多模光纤的传输距离( )
\par\twoch{短}{\textcolor{red}{长}}{相同}{无法比较}
\begin{solution}单模光纤的直径只有一个光的波长,光线在其中一直向前传播,不会发生反射;多模光纤中存在多条不同角度入射的光线,通过全反射传输。由于传输过程中光脉冲会逐渐展宽,所以多模光纤只合适于近距离传输,故选B。
\end{solution}
\question (中南大学)当以100Mbit/s的速度连接若干台计算机,具有最小费用的可用传输介质是(
)
\par\twoch{\textcolor{red}{5类UTP}}{3类UTP}{光纤}{同轴电缆}
\begin{solution}同轴电缆的最大传输速率为10Mbit/s;3类UTP的最大传输速率为16Mbit/s;光纤和5类UTP的最大传输速率虽然都可以达到100Mbit/s,但是光纤的成本要比5类UTP高得多,故选A。
\end{solution}
\question (中南大学)光纤系统的实际速率主要受限于( )
\par\twoch{单模光纤的带宽}{多模光纤的带宽}{光产生的速率}{\textcolor{red}{光电转换的速率}}
\begin{solution}光的频率非常高,达108MHz,其传输带宽非常大,实际速率主要受限于光电转换的速率。
\end{solution}
\question 10Base-T指的是( )
\par\fourch{10M波特率,使用数字信号,使用双绞线}{\textcolor{red}{10Mbit/s,使用数字信号,使用双绞线}}{10M波特率,使用模拟信号,使用双绞线}{10Mbit/s,使用模拟信号,使用双绞线}
\begin{solution}10表示每秒传输10Mbit数据,因此是10Mbit/s。Base表示采用基带传输,所以为数字信号。T表示使用了双绞线(Twisted-pair)。
\end{solution}
\question 同轴电缆比双绞线的传输速度更快,得益于( )
\par\fourch{同轴电缆的铜心比双绞线粗,能通过更大的电流}{同轴电缆的阻抗比较标准,减少了信号的衰减}{\textcolor{red}{同轴电缆具有更高的屏蔽性,同时有更好的抗噪声性}}{以上都对}
\begin{solution}同轴电缆以硬铜线为芯,外面包上一层绝缘的材料,绝缘材料的外面包围上一层密织的网状导体,导体的外面又覆盖上一层保护性的塑料外壳。同轴电缆的这种结构使得它具有更高的屏蔽性,从而既有很高的带宽,又有很好的抗噪特性。所以,同轴电缆的带宽更高得益于它的高屏蔽性。
\end{solution}
\question (中南大学)( )传输介质可以支持100Mbit/s速率,并且最长传输距离为1000m
\par\twoch{\textcolor{red}{光纤}}{无线}{同轴电缆}{双绞线}
\begin{solution}在网络中,同轴电缆传输速率只能达到10Mbit/s,最长传输距离为500m;双绞线尽管传输速率能达到100Mbit/s甚至更高,但最长传输距离仅为100m;目前,无线局域网速率最高也就是54Mbit/s,不采用放大设备时,传输距离也只能在100m左右。
\end{solution}
