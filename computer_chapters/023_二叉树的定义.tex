\question (中国科学院,2006年)现有一``遗传''关系:设x是y的父亲,则x可以把它的属性遗传给y,表示该遗传关系最适合的数据结构为(
)
\par\twoch{向量}{\textcolor{red}{树}}{图}{二叉树}
\begin{solution}树因为有祖先和孩子结点,所以是表示遗传关系最好的数据结构。
本题可用排除法,一个父亲可对应多个孩子,这个关系向量没法表示,二叉树也没法表示。
遗传关系不可能出现环,因此图也不合适。
\end{solution}
\question (南京理工大学,1999年)在下述结论中,正确的是( ) ①
只有一个结点的二叉树的度为0 ② 二叉树的度为2 ③ 二叉树的左右子树可任意交换
④ 深度为K的完全二叉树结点个数小于或等于深度相同的满二叉树
\par\twoch{①②③}{②③④}{②④}{\textcolor{red}{①④}}
\begin{solution}①
正确,树的度即为树中各结点度的最大值,这里就只有一个叶结点,因此树的度为0。
②
错误,二叉树也可以是只有根结点的,可以是每个结点的度都为1的,因此二叉树的度可以为0,也可以为1。
③ 错误,二叉树的子树有左右之分,不能颠倒。 ④
正确,因为一棵完全二叉树,一定是由深度相同的满二叉树,从右至左从下至上,挨个删除结点所得到的,因此结点数肯定小于或等于深度相同的满二叉树。
这种类多选的单选题,考研经常出现。技巧就是找到错误的选项排除,然后对比剩下的选项,选出自己认为正确的。比如这个题,③是肯定错误的,则立马排除了A、B。再看C跟D,只有①②不同,我们再考虑②,二叉树的度可以不为2,排除C选项,从而选中D选项。
\end{solution}
