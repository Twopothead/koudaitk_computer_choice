\question (重庆大学,2004年)若要在O(1)的时间复杂度上实现两个循环链表头尾相接,则对应两个循环链表各设置一个指针,分别指向(
)
\par\twoch{各自的头结点}{\textcolor{red}{各自的尾结点}}{各自的第一个元素结点}{一个表的头结点,另一个表的尾结点}
\begin{solution}两个循环链表头尾相接,需要改变头结点和尾结点之间的指针,而这个指针是从尾结点指向头结点的,所以只有将两个指针分别指向自己循环链表的尾结点才能完成操作。
实现的代码如下: void connect(LNode *A,LNode *\&B)
//假设A、B为非空带头结点的循环链表的尾指针 \{ LNode *p=A→next;
//保存A表的头结点 A→next=B→next→next; //B的开始结点链接到A表尾
free(B→next); //释放B表的头结点 B→next=p;
//将B表的尾结点链接到A表的头结点 \}
【小技巧】一般出现循环链表的题目时,尾指针的作用总是大于头指针的,因为头指针可通过尾指针直接得到。因此这样的题目一般都会选择带尾指针的选项。
\end{solution}
\question (南开大学,2005年)在一个链队列中,假设f和r分别为队首和队尾指针,则插入s所指结点的运算是(
)
\par\twoch{f→next=s;f=s;}{\textcolor{red}{r→next=s;r=s;}}{s→next=r;r=s;}{s→next=f;f=s;}
\begin{solution}插入是在队尾进行的,即在r所指结点之后插入结点s,因此正确的操作为B。
\end{solution}
\question (广东工业大学,2002年)在链队列的出队操作中,修改尾指针的情况发生在(
)
\par\twoch{\textcolor{red}{变成空队列的时候}}{变成满队列的时候}{队列只剩下一个元素的时候}{任何时候}
\begin{solution}出队一般只修改头指针,但如果头指针和尾指针指向同一个结点,且需要进行出队操作时,那么就需要修改尾指针,因此本题选A。
\end{solution}
\question 下列关于链式栈的叙述中,错误的是( ~)。

~\ding{192}.链式栈只能顺序存取,而顺序栈不但能顺序存取,还能直接存取
\ding{193}.因为链式栈没有栈满问题,所以进行进栈操作,不需要判断任何条件

~\ding{194}.在链式队列的出队操作中,需要修改尾指针的情况发生在空队列的时候
\par\twoch{仅\ding{192}}{仅\ding{192}、\ding{193}}{仅\ding{193}}{\textcolor{red}{\ding{192}、\ding{193}、\ding{194}}}
\begin{solution}\ding{192}:栈要求只能在表的一端(栈顶)访问、插入和删除,这决定了栈无论采用何种存储方法表示,只能顺序访问,不能直接存取,故\ding{192}错误。

~\ding{193}:每创建新的栈结点时还要判断是否动态分配成功,若不成功,则进栈操作失败。
StackNode *s=new StackNode; if(s==NULL)\{ ~
~printf(``结点存储分配失败!\textbackslash{}n''); \} 故\ding{193}错误。

~\ding{194}:首先要清楚链式队列需要两个指针,即头指针和尾指针。当链队列需要插入元素时,在链式队列尾部插入一个新的结点,并且修改尾指针;当链队列需要删除元素时,在链式队列头部删除一个结点,并且修改头指针。所以当链式队列需要进行入队操作时,应该只需修改尾指针即可。但是有一种特殊情况(考生务必记住,因为不少考生在写链式队列出队的算法时,并没有考虑到去判断这种情况),就是当此时只有一个元素时,不妨设此时链式队列有头结点,那么当唯一一个元素出队时,应该将头指针指向头结点,并且此时尾指针也是指向该唯一的元素,所以此时需要修改尾指针,并且使尾指针指向头结点,故\ding{194}错误。
\end{solution}
