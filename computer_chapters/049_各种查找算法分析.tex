\question (中国科学院,2006年)查找低效的数据结构是( )
\par\twoch{有序顺序表}{二叉排序树}{\textcolor{red}{堆}}{平衡的二叉排序树}
\begin{solution}有序顺序表,可以进行复杂度为O(logn)的折半查找。
对于二叉排序树和平衡的二叉排序树,都可以进行复杂度为O(logn)的查找。
但是对于堆,只能进行复杂度为O(n)的顺序查找。 因此本题选C。
\end{solution}
\question 下列选项中,不能构成折半查找中关键字比较序列的是
\par\fourch{\textcolor{red}{500,200,450,180}}{500,450,200,180}{180,500,200,450}{180,200,500,450}
\begin{solution}二分查找算法。
\end{solution}
\question (华中科技大学,2005年)对于顺序查找,假定查找成功与不成功的可能性相同,对每个记录的查找概率也相同,此时顺序查找的平均查找长度为(
)
\par\twoch{(n+1)/2}{(n+1)/4}{(n-1)/2}{\textcolor{red}{0.75n+0.25}}
\begin{solution}查找成功的平均查找长度为(n+1)/2,查找不成功的查找长度为n,故平均查找长度为:0.75n+0.25
\end{solution}
