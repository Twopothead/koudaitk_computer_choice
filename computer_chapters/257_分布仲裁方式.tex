\question 下列关于总线仲裁方式的说法中,正确的有( )。
Ⅰ.独立请求方式响应时间最快,是以增加处理器开销和增加控制线数为代价的
Ⅱ.计数器定时查询方式下,有一根总线请求(BR)和一根设备地址线,若每次计数都从0开始,则设备号小的优先级高
Ⅲ.链式查询方式对电路故障最敏感
Ⅳ.分布式仲裁控制逻辑分散在总线各部件中,不需要中央仲裁器
\par\twoch{Ⅲ和Ⅳ}{\textcolor{red}{Ⅰ、Ⅲ和Ⅳ}}{Ⅰ、Ⅱ和Ⅳ}{Ⅱ、Ⅲ和Ⅳ}
\begin{solution}理解3种仲裁方式的原理、线数和优缺点。
独立请求方式每个设备均由一对总线请求线和总线允许线,总线控制逻辑复杂,但响应速度快,Ⅰ正确。
计数器定时方式采用一组设备地址线(log2(n)),Ⅱ错误。
链式查询方式对硬件电路故障敏感,且优先级不能改变,Ⅲ正确。
分布式仲裁方式不需要中央仲裁器,每个主模块都有自己的仲裁号和仲裁器,多个仲裁器竞争使用总线,Ⅳ正确。
\end{solution}
\question 以下叙述中错误的是
\par\fourch{\textcolor{red}{总线结构传送方式可以提高数据的传输速度}}{与独立请求方式相比,链式查询方式对电路的故障更敏感}{总线标准化后,使得在计算机中增删设备非常容易,提高了设备的兼容性和互换性}{总线的带宽是总线本身所能达到的最高传输速率}
\begin{solution}A错误,总线结构传送方式并不能提高数据的传输速度。
B正确,链式查询方式的缺点就是对询问链的电路故障很敏感,如果第i个设备的接口中有关链的电路有故障,那么第i个以后的设备都不能进行工作。
C正确。总线标准化后,使得在计算机中增删设备非常容易,提高了设备的兼容性和互换性,因此I/O总线和通信总线大多是标准化总线。
D正确。总线带宽指总线的最大数据传输率,即在数据传输阶段单位时间内总线上可传输的数据量。
\end{solution}
