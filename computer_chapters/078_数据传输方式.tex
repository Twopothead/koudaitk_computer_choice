\question 在发送器上产生的时延被称为
\par\twoch{排队时延}{\textcolor{red}{传输时延}}{处理时延}{传播时延}
\begin{solution}1)在发送器上产生的时延被称为传输时延,或者称为发送时延。
2)在传输链路上产生的时延被称为传播时延。
3)主机或路由器在接收到分组时所花费的时间称为处理时延。
4)分组在进过网络传输时,要经过许多路由器,但分组在进入路由器后要先在输入队列中排队等待处理。在路由器确定转发接口后,还需要在输出队列中排队等待转发,这些都被称为排队时延。
\end{solution}
\question 因特网上的数据交换方式是( )
\par\twoch{电路交换}{报文交换}{\textcolor{red}{分组交换}}{光交换}
\begin{solution}电路交换主要用于电话网,报文交换主要用于早期的电报网。因特网使用的是分组交换,具体包括数据报和虚电路两种方式。
\end{solution}
\question 采用1200bit/s同步传输时,若每帧含56bit同步信息,48bit控制信位和4096bit数据位,那么传输1024B需要(
)s
\par\twoch{1}{4}{\textcolor{red}{7}}{14}
\begin{solution}计算每帧帧长=(56+48+4096)bit=4200bit,1024B=8192bit,由于每帧都有4096bit数据位,故可将8192bit分成2帧传输,一共需要传输8400bit,而同步传输的速率是1200bit/s,故传输8400bit需要7s。
\end{solution}
\question (华中理工大学,2001年)利用模拟通信信道传输数字信号的方法称为( )
\par\twoch{同步传输}{异步传输}{基带传输}{\textcolor{red}{频带传输}}
\begin{solution}信道上传送的信号分为基带信号和宽带信号。基带信号是将数字信号0和1直接用两种不同的电压表示,然后传送到数字信道上去传输,称为基带传输;宽带信号是将基带信号进行调制后形成模拟信号,然后再传送到模拟信道上去传输,称为频带传输。总之,记住一句话:基带对应数字信号,宽带对应模拟信号。
\end{solution}
\question 下列关于异步传输的描述,正确的是( )。 \ding{192}.每次传输一个数据块
\ding{193}.收/发端不需要进行同步 \ding{194}.对时序的要求较低
\par\twoch{仅\ding{192}、\ding{193}}{\textcolor{red}{仅\ding{193}、\ding{194}}}{\ding{192}、\ding{193}和\ding{194}}{仅\ding{194}}
\begin{solution}\ding{192}:异步传输是面向字节的传输,也就是每次传输一个字节,故\ding{192}错误。
\ding{193}:收/发端无须进行同步,因为每个字节都加上了同步信号。也就是说,当一个字节发送完后,可以经过任意长的时间间隔再发送下一个字节,故\ding{193}正确。
\ding{194}:异步传输只需要使用具有一般精准度的时钟就行,比同步传输的时序要求低,故\ding{194}正确。
\end{solution}
\question 在大多数情况下,同步传输和异步传输的过程中,分别使用( )作为传输单位
\ding{192}.位 \ding{193}.字节 \ding{194}.帧 \ding{195}.分组
\par\twoch{\ding{192}、\ding{193}}{\ding{193}、\ding{194}}{\textcolor{red}{\ding{194}、\ding{193}}}{\ding{193}、\ding{195}}
\begin{solution}异步传输以字节为传输单位,每一字节增加一个起始位和一个终止位。同步传输以数据块(帧)为传输单位(可以参见本章习题8,一次性传4200bit),为了使接收方能判定数据块的开始和结束,需要在每个数据块的开始处加一个帧头,在结尾处加一个帧尾。接收方判别到帧头就开始接收数据块,直到接收到帧尾为止。
补充知识点:从以上分析可以大致来讨论同步传输和异步传输的效率。同步传输可以从习题8看出,帧头和帧尾只占数据位很小的一部分,几乎可以忽略不计,可以认为同步传输的传输效率近似为100\%,但是异步传输每传8bit就要加一个起始位和一个终止位,可以得到异步传输的效率为80\%,所以同步传输比异步传输的效率高。
注意:此题应看清题目的条件限制,大多数情况下异步传输是以8bit长的字符为单位,也就是1B。当然,特殊情况会有,也有可能字符长度超过8bit,小概率事件不予考虑。
\end{solution}
\question 下列关于卫星通信的说法,错误的是( )
\par\fourch{卫星通信的通信距离大,覆盖的范围广}{使用卫星通信易于实现广播通信和多址通信}{\textcolor{red}{卫星通信不受气候的影响,误码率很低}}{通信费用高,时延较大是卫星通信的不足之处}
\begin{solution}卫星通信是微波通信的一种特殊形式,通过地球同步卫星作为中继来转发微波信号,可以克服地面微波通信距离的限制。卫星通信的优点是通信距离远、费用与通信距离无关、覆盖面积大、通信容量大、不受地理条件的制约、易于实现多址和移动通信,缺点是费用较高、传输延迟大、对环境气候较为敏感。
\end{solution}
\question 利用模拟通信信道传输数字信号被称为( )
\par\twoch{基带传输}{\textcolor{red}{频带传输}}{调频}{调幅}
\begin{solution}首先,要将数字信号调制为模拟信号,才能使其在模拟通信信道传输。而基带传输的信号是未被调制的数字信号,故A选项错误。其次,C选项与D选项仅仅是调制的方式,故C选项与D选项错误。频带传输就是先将基带信号调制成便于在模拟信道中传输的、具有较高频率范围的模拟信号(称为频带信号),再将这种频带信号在模拟通信信道中传输。
\end{solution}
\question 全双工以太网传输技术的特点是( )。 \ding{192}.能同时发送和接收帧
\ding{193}.不受CSMA/CD限制 \ding{194}.不能同时发送和接收帧 \ding{195}.受CSMA/CD限制
\par\twoch{\textcolor{red}{\ding{192},\ding{193}}}{\ding{192},\ding{195}}{\ding{193},\ding{194}}{\ding{194},\ding{195}}
\begin{solution}因为全双工既有接收数据的通道又有发送数据的通道,所以能同时发送和接收帧。CSMA/CD协议是防止发送和接收冲突的协议,而全双工根本不存在冲突,所以不受CSMA/CD的限制。
\end{solution}
