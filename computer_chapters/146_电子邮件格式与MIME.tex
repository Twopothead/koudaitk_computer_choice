\question 电子邮件经过MIME扩展后,可以将非ASCII码内容表示成ASCII码内容,其中base64的编码方式是
\par\fourch{ASCII码字符保持不变,非ASCII码字符用=XX表示,其中XX是该字符的十六进制值}{\textcolor{red}{不管是否是ASCII码字符,每3个字符用另4个ASCII字符表示}}{以64为基数,将所有非ASCII码字符用该字符的十六进制值加64后的字符表示}{将每4个非ASCII码字符用6个ASCII码字符表示}
\begin{solution}选项A是quoted-printable编码方式,所以排除;base64编码方式不管是否是ASCII码字符,每3个字符用另外4个ASCII码字符表示。
\end{solution}
\question 下列关于SMTP协议的叙述中,正确的是( )。 \ding{192}.只支持传输7比特ASCII码内容
\ding{193}.支持在邮件服务器之间发送邮件 \ding{194}.支持从用户代理向邮件服务器发送邮件
\ding{195}.支持从邮件服务器向用户代理发送邮件
\par\twoch{\textcolor{red}{仅\ding{192}、\ding{193}和\ding{194}}}{仅\ding{192}、\ding{193}和\ding{195}}{仅\ding{192}、\ding{194}和\ding{195}}{仅\ding{193}、\ding{194}和\ding{195}}
\begin{solution}SMTP只支持传输7比特ASCII码内容,故I选项正确;用户代理到邮件服务器,邮件服务器到邮件服务器都是使用SMTP,故II和III正确;而从邮件服务器到用户代理发送邮件使用的是POP3,故IV错误。
\end{solution}
