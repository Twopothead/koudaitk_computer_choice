\question (江苏大学,2004年)在数据结构中,从存储结构上可以把数据结构分成( ~)
\par\fourch{\textcolor{red}{顺序结构和链式结构}}{紧凑结构和非紧凑结构}{线性结构和非线性结构}{动态结构和静态结构}
\begin{solution}数据元素之间的关系在计算机中有两种不同的表示方法:顺序映像和非顺序映像。对应的两种不同的存储结构分别是顺序存储结构和链式存储结构。
\end{solution}
\question (北京交通大学,2000年)以下与数据的存储结构无关的术语是( ~)
\par\twoch{循环队列}{链表}{散列表}{\textcolor{red}{栈}}
\begin{solution}数据的物理结构又称为存储结构,是数据的逻辑结构在计算机中的表示。
循环队列是建立在顺序存储结构上的,因此A相关。
链表是以链式结构存储的,因此B相关。
散列存储方法本质上是顺序存储方法的扩展,散列表本质上是顺序表的扩展,因此C相关。
栈是逻辑结构,因为栈可以是顺序存储也可以是链式存储。
【注】做这类题主要判断该结构是不是就对应了某种存储方法,如果可以用多种存储方法实现,就是数据的逻辑结构,如果只能用一种存储方法实现,它就跟存储结构相关。
\end{solution}
