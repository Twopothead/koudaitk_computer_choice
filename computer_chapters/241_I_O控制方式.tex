\question 在I/O设备控制方式的发展过程中,最主要的推动力是
\par\twoch{提高资源利用率}{提高系统吞吐量}{\textcolor{red}{减少CPU对I/O控制的干扰}}{缓解CPU速度和I/O速度不匹配的矛盾}
\begin{solution}在I/O控制的发展过程中,始终贯穿着这样一个宗旨:尽量减少CPU对I/O控制的干预,把主机从繁杂的I/O控制事务中解脱出来,以更多地去完成其数据处理任务。
\end{solution}
\question 考虑56Kbit/s调制解调器的性能,驱动程序输出一个字符后就阻塞,当一个字符打印完毕后,产生一个中断通知阻塞的驱动程序,输入下一个字符,然后再阻塞。如果发消息、输出一个字符和阻塞的时间总和为0.1ms,那么由于处理调制解调器而占用的CPU时间比率是(
)(假设每个字符有一个开始位和一个结束位,共占10位)
\par\twoch{\textcolor{red}{56\%}}{57\%}{58\%}{59\%}
\begin{solution}因为一个字符占10位,因此在56Kbit/s的速率下,每秒传送:56000/10=5600个字符,即产生5600次中断。每次中断需0.1ms,故处理调制解调器占用CPU时间总共为5600×0.1ms=
560ms。计算时间比率:560ms/1s=56\%,所以选择A选项。
\end{solution}
\question 在一个32位100MHz的单总线计算机系统中(每10ns一个周期),磁盘控制器使用DMA以40MB/s的速率从存储器中读出数据或者向存储器写入数据。假设计算机在没有被周期挪用的情况下,在每个循环周期中读取并执行一个32位的指令。这样做,磁盘控制器使指令的执行速度降低的比例是
\par\twoch{\textcolor{red}{10\%}}{20\%}{30\%}{40\%}
\begin{solution}在32位单总线的系统中,磁盘控制器使用DMA传输数据的速率为40MB/s,即每100ns传输4B(32b)的数据。控制器每读取10个指令就挪用1个周期。因此,磁盘控制器使指令的执行速度降低了10\%。
\end{solution}
