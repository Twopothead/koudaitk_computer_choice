\question (青岛大学,2004年)向一个栈顶指针为top的链栈中插入一个节点s,则执行( )
\par\twoch{top→next=s;}{s→next=top→next;top→next=s;}{\textcolor{red}{s→next=top;top=s;}}{s→next=top;top=top→next;}
\begin{solution}栈顶指针指向栈顶的元素,因此插入s后,一定要修改栈顶指针指向刚插入的元素
\end{solution}
\question 下列关于栈的说法中,正确的是( )。
\ding{192}.若进栈顺序为a、b、c,则通过出栈操作可能得到5个a、b、c的不同排列
\ding{193}.链式栈的栈顶指针一定指向栈的链尾
\ding{194}.两个栈共享一个向量空间的好处是减少了存取时间
\par\twoch{\textcolor{red}{仅\ding{192}}}{仅\ding{192}、\ding{193}}{仅\ding{193}}{仅\ding{193}、\ding{194}}
\begin{solution}\ding{192}:该选项旨在让考生知道一个公式。对于n个不同元素进栈,出栈序列的个数满足Catalan函数可以马上得出,当n=3时,出栈序列个数为5故\ding{192}正确。
\ding{193}:链式栈一般采用单链表,栈顶指针即为链头指针。进栈和出栈均在链头进行,每次都要修改栈顶指针,链空即栈空(top==NULL),故\ding{193}错误。
\ding{194}:由于栈中数据的操作只有入栈和出栈,且时间复杂度均为O(1),因此并没有减少存取时间,故\ding{194}错误。
\end{solution}
\question 下列关于链式栈的叙述中,错误的是( ~)。

~\ding{192}.链式栈只能顺序存取,而顺序栈不但能顺序存取,还能直接存取
\ding{193}.因为链式栈没有栈满问题,所以进行进栈操作,不需要判断任何条件

~\ding{194}.在链式队列的出队操作中,需要修改尾指针的情况发生在空队列的时候
\par\twoch{仅\ding{192}}{仅\ding{192}、\ding{193}}{仅\ding{193}}{\textcolor{red}{\ding{192}、\ding{193}、\ding{194}}}
\begin{solution}\ding{192}:栈要求只能在表的一端(栈顶)访问、插入和删除,这决定了栈无论采用何种存储方法表示,只能顺序访问,不能直接存取,故\ding{192}错误。

~\ding{193}:每创建新的栈结点时还要判断是否动态分配成功,若不成功,则进栈操作失败。
StackNode *s=new StackNode; if(s==NULL)\{ ~
~printf(``结点存储分配失败!\textbackslash{}n''); \} 故\ding{193}错误。

~\ding{194}:首先要清楚链式队列需要两个指针,即头指针和尾指针。当链队列需要插入元素时,在链式队列尾部插入一个新的结点,并且修改尾指针;当链队列需要删除元素时,在链式队列头部删除一个结点,并且修改头指针。所以当链式队列需要进行入队操作时,应该只需修改尾指针即可。但是有一种特殊情况(考生务必记住,因为不少考生在写链式队列出队的算法时,并没有考虑到去判断这种情况),就是当此时只有一个元素时,不妨设此时链式队列有头结点,那么当唯一一个元素出队时,应该将头指针指向头结点,并且此时尾指针也是指向该唯一的元素,所以此时需要修改尾指针,并且使尾指针指向头结点,故\ding{194}错误。
\end{solution}
