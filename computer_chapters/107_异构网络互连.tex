\question (北京理工大学)在路由器互联的多个局域网中,通常要求每个局域网的
\par\fourch{数据链路层协议和物理层协议必须相同}{数据链路层协议必须相同,而物理层协议可以不同}{数据链路层协议可以不同,而物理层协议必须相同}{\textcolor{red}{数据链路层协议和物理层协议都可以不同}}
\begin{solution}虚拟互联网络也就是逻辑互联网络,它的意思就是互连起来的各种物理网络的异构性本来是客观存在的,但是利用IP可以使这些性能各异的网络让用户看起来好像是一个统一的网络,这就是网络层的异构网络互连功能。路由器工作在网络层,互联多个局域网时数据链路层协议和物理层协议都可以不相同。
总结:本层及本层以下的协议可以不同,但是高层协议必须相同。例如,在数据链路层互连,物理层协议与数据链路层协议可以不同,但是网络层及其以上协议必须相同。
\end{solution}
\question 路由器进行间接交付的对象是
\par\twoch{脉冲信号}{帧}{\textcolor{red}{IP数据报}}{UDP数据报}
\begin{solution}路由器间接交付是在IP层上实行跨网段的交付,所以需要使用IP地址,间接交付的对象是IP数据报。
\end{solution}
\question 以下说法错误的是( )。 Ⅰ.路由选择分直接交付和间接交付
Ⅱ.直接交付时,两台机器可以不在同一物理网段内
Ⅲ.间接交付时,不涉及直接交付 Ⅳ.直接交付时,不涉及路由器
\par\twoch{Ⅰ和Ⅱ}{\textcolor{red}{Ⅱ和Ⅲ}}{Ⅲ和Ⅳ}{Ⅰ和Ⅳ}
\begin{solution}首先路由选择分为直接交付和间接交付,当两台主机在同一物理网段内时,就使用直接交付,反之,使用间接交付,所以Ⅰ是正确的,Ⅱ是错误的;间接传送的最后一个路由器肯定是直接交付,所以Ⅲ错误;直接交付时,是在同一物理网段内,所以不涉及路由器。综上所述,Ⅱ和Ⅲ是错误的。
可能疑问点一:通过网桥连接的网段,从这个网段的一台主机发向另一个网段的主机,中间并没有经过路由器,因为它们处在一个网络中。这也叫做直接交付吗?
解析:直接交付是指在一个物理网络上把数据报从一台主机传输到另一台主机。间接交付是指当源主机和目的主机分别处于不同的物理网络上时,数据报由源主机通过中间的路由器把数据报间接地传输到目的主机的过程。因此即使是网桥连接的,但是都属于同一物理网络,所以仍然属于直接交付。
可能疑问点二:书上说路由器将数据报直接交付给主机A,这里的直接交付不是涉及路由器吗?
解析:直接交付不涉及路由器,意思是比如A和B两点通信,不管A和B是主机还是路由器,只需看中间过程是否经过路由器。经过,就是间接;不经过,就是直接交付。
\end{solution}
