\question 一般存储系统由三级组成,下列关于各级存储器的作用及速度、容量的叙述中正确的是
\par\fourch{主存存放正在CPU中运行的程序,速度较快,容量较大}{Cache存放当前所有访问频繁的数据,特点是速度最快,容量较小}{\textcolor{red}{外存存放需联机保存但暂不执行的程序和数据,容量很大且速度很慢}}{外存存放需联机保存但暂不执行的程序和数据,容量很大且速度很快}
\begin{solution}C。
主存速度快,但容量较小;Cache存放当前部分访问频繁的数据,这些数据是主存数据的复制品,不一定能存放当前所有访问频繁的数据;外存(辅存)容量大,成本低,速度慢。
\end{solution}
\question (西安理工大学)计算机系统中的存储器系统是指
\par\twoch{RAM}{ROM}{\textcolor{red}{主存储器和快速缓冲存储器}}{内存储器和外存储器}
\begin{solution}C。 通常所说的计算机存储系统是指主存储器和快速缓冲存储器。
\end{solution}
\question (哈尔滨工程大学,2003年)( )存储结构对程序员是透明的
\par\twoch{通用寄存器}{主存}{\textcolor{red}{控制寄存器}}{堆栈}
\begin{solution}C。
通用寄存器、主存、堆栈都是可以被程序员编程操作的,而控制寄存器是控制器的内部结构,只关乎计算机系统的设计人员,对程序员是透明的。
\end{solution}
\question 在下列计算机的存储部件中,CPU不能直接访问的是( )
\par\twoch{主存储器}{\textcolor{red}{辅助存储器}}{寄存器}{Cache}
\begin{solution}计算机不能直接访问辅存,辅存中的内容只有先调入主存储器(内存)才能被CPU直接访问。
\end{solution}
