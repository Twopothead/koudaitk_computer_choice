\question 下列关于设备属性的论述中,正确的是
\par\fourch{字符设备的基本特征是可寻址的,即能指定输入的源地址和输出的目标地址}{\textcolor{red}{共享设备必须是可寻址的和可随机访问的设备}}{共享设备是指同一时刻允许多个进程同时访问的设备}{在分配共享设备和独占设备时都可能引起进程死锁}
\begin{solution}可寻址是块设备的基本特征,故A不正确。共享设备是指一段时间内允许多个进程同时访问的设备;在同一时间内,即对于某一时刻共享设备仍然只允许一个进程访问,故C不正确。分配共享设备是不会引起进程死锁的,故D不正确。
\end{solution}
\question (武汉理工大学,2005年)磁盘是可共享的设备,因此每一时刻( )作业启动它
\par\twoch{可以有任意多个}{能限定多个}{至少能有一个}{\textcolor{red}{至多能有一个}}
\begin{solution}磁盘是可共享的设备,是指在某一时间段内可以允许多个用户或进程使用它,但是在某一时刻,最多只有一个作业在使用它,因为磁盘空闲是可能的。
\end{solution}
\question 大多数设备控制器是由三部分组成的,其中用于实现对设备的控制的是
\par\twoch{设备控制器与处理机的接口}{设备控制器与设备的接口}{\textcolor{red}{I/O逻辑}}{都不对}
\begin{solution}设备控制器由以下三部分组成:  设备控制器与处理机的接口。
用于实现CPU与设备控制器之间的通信。  设备控制器与设备的接口。
一个设备控制器上有一个或多个设备接口,可以连接一个或多个设备。设备控制器中的I/O逻辑根据处理机发来的地址信号,去选择一个设备接口。
 I/O逻辑。 用于实现对设备的控制。
\end{solution}
\question 系统管理设备是通过一些数据结构来进行的,下面的(
)不属于设备管理数据结构
\par\twoch{\textcolor{red}{FCB}}{DCT}{SDT}{COCT}
\begin{solution}FCB是文件控制块,与设备管理无关。DCT是设备控制表,SDT是系统设备表,COCT是控制器控制表,三者都是设备管理中的重要的数据结构。
\end{solution}
\question 下列关于I/O设备的论述正确的是
\par\fourch{在现代计算机系统中,只有I/O设备才是有效的中断源}{在中断处理过程中,必须屏蔽中断(即禁止发生新的中断)}{\textcolor{red}{同一用户所使用的I/O设备也可以并行工作}}{SPOOLing是将多台物理I/O设备虚拟为一台逻辑I/O设备}
\begin{solution}A选项错误,按引起中断的原因划分:输入、输出中断,计算机故障中断,实时时钟中断,软件中断。
B选项错误,不一定。 C选项正确,I/O设备间当然可以并行工作。
D选项错误,SPOOLing技术是将一台物理I/O设备虚拟为多台逻辑I/O设备,同样允许多个用户共享一台物理I/O设备。
\end{solution}
