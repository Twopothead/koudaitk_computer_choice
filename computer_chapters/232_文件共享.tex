\question 文件系统采用二级目录结构,这样可以( )。 Ⅰ.缩短访问文件存储器的时间
Ⅱ.实现文件共享 Ⅲ.节省主存空间 Ⅳ.解决不同用户之间的文件名冲突
\par\twoch{Ⅳ}{\textcolor{red}{Ⅰ和Ⅳ}}{Ⅲ和Ⅳ}{Ⅰ、Ⅱ和Ⅳ}
\begin{solution}二级目录结构的优点有:  提高了检索目录的速度,故Ⅰ正确。 
可以解决文件重名问题,故Ⅳ正确。 
不同用户还可使用不同的文件名来访问系统中的同一个共享文件,但这并不是实现文件共享的方式,故Ⅱ错误。
Ⅲ更是明显错误,目录越多目录文件就越多,占用的主存空间自然就多了,不可能是节省主存空间。所以本题选择B选项。
知识点回顾:
文件的共享方式有:基于索引结点的共享方式和利用符号链实现文件共享。
基于索引结点的共享方式:
将文件的物理地址及其他的文件属性等信息不再放置在目录项中,而是放在索引结点中。目录项中有文件名和指向索引结点的指针,两个不同的目录项只需要指向相同的索引结点即可实现文件共享,即一个共享文件只有一个索引结点,不同的文件名的目录项需要共享的话只需要在目录项中指向该索引结点即可。
在索引结点中再增加一个计数值来统计指向该索引结点的目录项的个数,这样一来需要删除该文件的时候可以判断计数值,只有计数值为1时才删除该索引结点。若计数器大于1,则把计数值减1即可。
利用符号链实现文件共享:
该方法是创建一个称为链接的新目录项。例如,为使用户A能共享用户B的一个文件F,在目录表中为用户A创建一个到文件F链接的新目录项。链接实际上是用另一个文件或目录的指针,可以是绝对路径或相对路径。这样的链接方式被称为符号链接。
\end{solution}
\question 设文件F1的当前引用计数值为1,
先建立F1的符号链接(软链接)文件F2,再建立F1的硬链接文件F3,然后删除F1。此时,F2和F3的引用计数值分别是(
)
\par\twoch{0、1}{\textcolor{red}{1、1}}{1、2}{2、1}
\begin{solution}创建符号链接文件时,该文件会创建自己的inode结构,引用计数值与目标文件独立,而硬链接文件和目标文件共享inode结构;在删除文件时,引用计数器减1,当引用计数器为0时,才真正删除,并释放inode结构。创建文件F2后,其文件引用计数器为1且不会随着F1的变化而变化;创建文件F3后,由于共享inode,目标文件引用计数器加1,变为2,当删除文件F1时,引用计数器减1,此时文件F3所共享的引用计数器为1。所以,F2和F3的文件引用计数器都是1。
【总结】
硬链接:不改变文件的inode,与原文件共用inode结构,每多一个硬链接,引用计数器加1;删除一个硬链接,引用计数器减1。当计数器为0时,删除文件。
软链接:产生新的inode,与原文件彼此独立,原文件的引用计数器变化对新软链接文件的引用计数器没有影响。
\end{solution}
