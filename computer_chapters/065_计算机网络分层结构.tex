\question 当数据由主机A送传至主机B时,不参与数据封装工作的是
\par\twoch{\textcolor{red}{物理层}}{数据链路层}{网络层}{传输层}
\begin{solution}从上层往下层传输的时候,需要加上一个首部,数据链路层不仅要加首部还要加尾部。而数据链路层传输到物理层,仅仅是将数据链路层中的帧变成比特流的形式在传输介质中传输,不需要加首部,即不需要数据封装。
\end{solution}
\question 计算机网络系统的基本组成是
\par\fourch{交换机、服务器、传输介质、用户计算机}{\textcolor{red}{计算机硬件资源、计算机软件资源、数据资源}}{操作系统、数据库与应用软件}{以上均不正确}
\begin{solution}计算机网络系统的基本组成应该包括计算机硬件资源、计算机软件资源、数据资源,A与C选项分别只涉及了硬件与软件。
\end{solution}
\question 人们将网络层次结构模型和各层协议集合定义为计算机网络的
\par\twoch{拓扑结构}{开放系统互连模型}{\textcolor{red}{体系结构}}{协议集}
\begin{solution}网络层次结构模型和各层协议集合定义为计算机网络的体系结构。
\end{solution}
\question (华中科技大学,1999)( )不是对网络模型进行分层的目标
\par\twoch{提供标准语言}{\textcolor{red}{定义功能执行的方法}}{定义标准界面}{增加功能之间的独立性}
\begin{solution}这道题变相地考查哪项不属于网络体系结构所描述的内容,此题2010年统考考查过。计算机网络的各层及其协议的集合称为网络体系结构,网络体系结构是抽象的,它不应该包括各层协议及功能的具体实现细节(定义功能执行的方法)。这些内部实现细节应该由工作人员完成,我们并不需要知道。
\end{solution}
\question 下列哪些项描述了网络体系结构中的分层概念? \ding{192}.保持网络灵活且易于修改
\ding{193}.定义了各个功能执行的方法 \ding{194}.把相关的网络功能组合在一层中
\par\twoch{\ding{192}、\ding{193}和\ding{194}}{只有\ding{194}}{\textcolor{red}{\ding{192}、\ding{194}}}{\ding{192}、\ding{193}}
\begin{solution}分层就是将所有的功能进行分类,每一层分别来实现不一样的功能。这样,当某层的功能需要改变,
    只需修改某层即可,而不需要修改其他层次,保持了网络的灵活且易于修改,故\ding{192}、\ding{194}正确。计算机网络各层及其协议的集合称为体系结构,
    体系结构是抽象的概念,并不包括各层协议及功能的具体实现细节,故\ding{193}错误。
\end{solution}
