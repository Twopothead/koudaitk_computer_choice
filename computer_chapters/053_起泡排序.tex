\question (北京航空航天大学,1999年)排序趟数与序列的原始状态有关的排序方法是(
)排序法
\par\twoch{插入}{选择}{\textcolor{red}{冒泡}}{基数}
\begin{solution}排序方法的趟数和原始序列有关的是交换类的排序,包括冒泡排序。
\end{solution}
\question (中国科学技术大学,2005年)对数据序列(8,9,10,4,5,6,20,1,2)采用(由后向前次序的)冒泡排序,需要进行的趟数(遍数)至少是(
)
\par\twoch{3}{4}{\textcolor{red}{5}}{8}
\begin{solution}本题问的是至少多少遍,所以每次在冒泡之前,检查一下余下的序列是否已经排好序,可以优化冒泡排序。当进行了5趟之后,序列的排列是(1,2,4,5,6,8,9,10,20),已经排好序,所以算法可以结束,即至少需要进行的趟数是5。
\end{solution}
