\question (哈尔滨工业大学,2007年)对于循环队列,正确的是( )
\par\twoch{无法判断队列是否为空}{无法判断队列是否为满}{队列不可能满}{\textcolor{red}{以上说法都不对}}
\begin{solution}循环队列,教材均给出了判空和判满的条件,故A和B错,C更是错误。因此本题选D。
\end{solution}
\question (中山大学,1999年)循环队列存储在数组A{[}0 m{]}中,则入队时的操作为(
)
\par\twoch{rear=rear+1}{rear=(rear+1) mod (m-1)}{rear=(rear+1) mod m}{\textcolor{red}{rear=(rear+1) mod (m+1)}}
\begin{solution}入队时队尾变化为rear=(rear+1)\%maxSize,但此处还是容易误选C选项,因为这里的maxSize不等于m,数组A{[}0
m{]}是有m+1个元素的,因此本题选D。
\end{solution}
\question (浙江大学,1994年)若用一个大小为6的数组来实现循环队列,且当前rear和front的值分别为0和3,当从队列中删除一个元素,再加入两个元素后,rear和front的值分别为(
)
\par\twoch{1和5}{\textcolor{red}{2和4}}{4和2}{5和1}
\begin{solution}删除一个元素,执行(front+1)\%maxSize,之后front为4。
增加一个元素,执行(rear+1)\%maxSize,之后rear为1,再增加一个元素后,rear为2。
因此最后,rear=2,front=4。
\end{solution}
\question 已知循环队列存储在一维数组A{[}0,\ldots{},n-1{]}中,且队列非空时front和rear分别指向队头和队尾元素。若初始时队列为空,且要求第1个进入队列的元素存储在A{[}0{]}处,则初始时front和rear的值分别是(
)
\par\twoch{0,0}{\textcolor{red}{0,n-1}}{n-1,0}{n-1,n-1}
\begin{solution}在循环队列中,进队操作是队尾指针rear循环加1,并在该处放置进队的元素,而队首指针不变。本题要求第一个进入队列的元素存储在A{[}0{]}处,队列非空时front和rear分别指向队头和队尾元素,此时front和rear均为0。则rear初值应为n-1,因为这样(rear+1)\%n=0。又front的值没有变,则front的初值应为0。
【总结】
在大多数教材中,循环队列的队头指针front设计为指向队列中队头元素的前一个位置,而队尾指针rear指向队尾元素的位置,该题对此进行了变化。
\end{solution}
\question 循环队列存放在一组数组A{[}0..M-1{]}中,end1指向队头元素,end2指向队尾元素的后一个位置。假设队列两端均可进行入队和出队操作,队列中最多能容纳M-1个元素,初始时为空。下列判断队空和队满的条件中,正确的是(
)
\par\fourch{\textcolor{red}{队空:end1 = = end2;
队满:end1 = = (end2+1) mod M}}{队空:end1 = = end2;
队满:end2 = = (end1+1) mod (M-1)}{队空:end2 = = (end1+1) mod M;
队满:end1 = = (end2+1) mod M}{队空:end1 = = (end2+1) mod M;
队满:end2 = = (end1+1) mod (M-1)}
\begin{solution}根据已知,队内中最多能容纳M-1个元素,说明该循环队列采用牺牲一个存储空间来区别队空和队满的情况,即队尾元素的后一个位置,即end2指向的位置。
首先排除B和D。循环队列的坐标空间为M,因此肯定都是对M求余。
然后再看C选项,end2指向end1的后一个位置时,说明end1为队尾元素,即不为空,矛盾。
\end{solution}
\question (西南交通大学,2005年)在具有n个单元的顺序存储的循环队列中,假定front和rear分别为队头指针和队尾指针,则判断队满的条件为(
)
\par\twoch
{rear\%n=front}
{(front+1)\%n=rear}
{rear\%n-1=front}
{\textcolor{red}{(rear+1)\%n=front}}

\begin{solution}
队满的情况一定是:如果再插入一个元素,此时尾指针再后移一位那么一定就会与头指针相遇,也就是说rear和front之间预留一个存储空间用来判断队满
\end{solution}
