\question ( )用于连接大量的低速或中速I/O设备
\par\twoch{数据选择通道}{\textcolor{red}{字节多路通道}}{数据多路通道}{I/O处理器}
\begin{solution}字节多路通道,它通常含有许多非分配型子通道,其数量可达几十到几百个,每一个通道连接一台I/O设备,并控制该设备的I/O操作。这些子通道按时间片轮转方式共享主通道。各个通道循环使用主通道,多个通道每次完成其I/O设备的一个字节的交换,然后让出主通道的使用权。这样只要字节多路通道扫描每个子通道的速率足够快,而连接到子通道上的设备的速率不是太高时,便不至于丢失信息。
\end{solution}
\question 以下哪种情况出现时,会引起CPU自动查询有无中断请求,进而可能进入中断响应周期
\par\twoch{\textcolor{red}{一条指令执行结束}}{一次I/O操作结束}{一次中断处理结束}{一次DMA操作结束}
\begin{solution}中断方式下,CPU总是在一条指令执行完、取下一条指令之前去查询有无中断请求。如果,此时是开中断状态,并有未被屏蔽的中断请求发生,则CPU自动执行一条隐指令,进行中断响应,完成关中断、保护断点、取中断向量3个操作。因此本题应该选A。
\end{solution}
\question DMA方式的接口电路中有程序中断部件,其作用是
\par\twoch{实现数据传送}{向CPU提出总线使用权}{\textcolor{red}{向CPU提出传输结束}}{发中断请求}
\begin{solution}DMA控制器中的中断机构用于数据块传送完毕时向CPU提出中断请求,CPU将进行DMA传送的结尾处理。
归纳总结:DMA控制器中的中断部件,仅当一个数据块传送完毕才被触发,向CPU提出中断请求,以通知CPU进行DMA传送的结尾处理。它与中断控制器中的功能不相同。
\end{solution}
\question 以下是有关对DMA方式的叙述: I.DMA控制器向CPU请求的是总线使用权
II.DMA方式可用于键盘和鼠标器的数据输入
III.DMA方式下整个I/O过程完全不需要CPU介入
IV.DMA方式需要用中断处理进行辅助操作 以上叙述中,错误的是
\par\twoch{I、II}{\textcolor{red}{II、III}}{II、IV}{III、IV}
\begin{solution}I正确,在实现DMA传输时,是由DMA控制器直接掌管总线,因此,存在着一个总线控制权转移问题。即DMA传输前,CPU要把总线控制权交给DMA控制器,而在结束DMA传输后,DMA控制器应立即把总线控制权再交回给CPU。
II错误,键盘和鼠标器的数据输入应该使用中断请求方式。
III错误,DMA的数据交换过程包括3个步骤,DMA控制器的初始化、DMA传送、DMA传送结束处理。其中,CPU参与初始化和后处理两部分工作,因此,III选项错误,其中传送后处理部分,是通过向CPU发出``DMA结束''中断请求,由CPU执行相应的中断服务程序进行数据校验等后处理工作的,因此IV正确。
\end{solution}
\question DMA方式的并行性是指
\par\fourch{多个I/O设备可同时并行地通过DMA控制器进行数据传送}{I/O设备和主存并行工作}{CPU和主存并行工作}{\textcolor{red}{CPU和DMA控制器并行工作}}
\begin{solution}DMA直接存储器存取,是一种快速传送数据的机制,DMA技术的重要性在于,利用它进行数据存取时不需要CPU进行干预,即通过CPU和DMA控制器并行工作,提高系统执行应用程序的效率。
\end{solution}
\question 以下关于DMA控制器和CPU关系的叙述中,错误的是
\par\fourch{DMA控制器和CPU都可以作为总线的主控设备}{\textcolor{red}{周期挪用法中,若CPU和DMA发生访存冲突,则CPU优先级高}}{CPU可以通过执行I/O指令来访问DMA控制器}{CPU可通过执行指令来启动DMA控制器}
\begin{solution}周期挪用的基本思想是,当外设准备好一个数据时,DMA控制器就向CPU申请一次总线控制权,CPU在一次总线操作结束时一旦发现有DMA请求,就立即释放总线,让出一个周期给DMA控制器,由DMA控制器控制总线在主存和外设之间传送一个数据,传送结束后立即释放总线,下次外设准备好数据时,又重复上述过程,直到所有数据传送完毕。再这种情况下,CPU的工作几乎不受影响,只是在万一出现访存冲突(即CPU和DMA控制器同时要求访问同一个主存时),CPU挪出一个周期给DMA,由DMA访问主存,而CPU延迟访问主存。这里CPU挪用的就是主存存取周期。
\end{solution}
\question CPU响应DMA请求的条件是当前( )执行完
\par\twoch{时钟周期}{\textcolor{red}{总线周期}}{硬件和软件}{固件}
\begin{solution}DMA请求申请的是总线控制权,故CPU只有在一次总线操作结束时(总线周期执行完)时,才会响应DMA请求。因此本题选B。
\end{solution}
\question 在DMA方式传送数据的过程中,由于没有破坏(
)的内容,所以CPU可以正常工作(访存除外)
\par\twoch{程序计数器}{\textcolor{red}{程序计数器和寄存器}}{指令寄存器}{非以上答案}
\begin{solution}DMA仅挪用了一个存储周期,不改变CPU现场,因此无需占用CPU的寄存器及程序计数器。于此不同的是中断控制方式,它必须进行CPU现场保护和恢复操作。
\end{solution}
\question 以下关于DMA的叙述中,正确的是
\par\fourch{DMA方式下,在主存和外设之间有一条物理通路直接相连}{DMA方式下,CPU没有开销}{CPU对DMA请求和中断请求的最长响应时间是相等的}{\textcolor{red}{周期挪用方式下,DMA控制器窃取的是主存的存储周期}}
\begin{solution}A错误,DMA方式并不是说在主存和外设之间建立一条物理上的直接通路,而是在主存和外设之间通过外设接口、系统总线以及总线桥接部件等连接,建立一个信息可以互相通达的通路。因此``直接通路''是逻辑上的含义,并非物理上的。
B错误,DMA的数据交换过程包括3个步骤,DMA控制器的初始化、DMA传送、DMA传送结束处理。其中,CPU参与初始化和后处理两部分工作,因此,不是一点开销都没有。
C错误,DMA方式下,向CPU请求的是总线控制权,要求CPU让出总线控制权给DMA控制器,由DMA控制器来控制总线完成主存和外设之间的数据交换,因此,CPU只要用完总线后就可以响应请求,释放总线,让出总线控制权。CPU总是在一次总线事务完成后响应,因此,DMA方式响应时间应该少于一个总线周期。而中断方式下请求的是CPU时间,要求CPU中止正在执行的程序,转到中断服务程序去执行,通过执行中断服务程序,对中断事件进行相应的处理。CPU总是要等到一条指令执行结束后,才去查询有无中断请求,所以响应时间少于一个指令周期的时间。因此,这两个响应时间并不相等。
D正确,在周期挪用方式下,每当I/O设备发出DMA请求时,I/O设备便挪用或窃取总线占用权一个或几个主存周期,而DMA不请求时,CPU仍继续访问主存。
\end{solution}
