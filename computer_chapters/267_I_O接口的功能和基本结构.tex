\question (
)是CPU与I/O设备之间的接口,它接收从CPU发来的命令,并去控制I/O设备工作,以使处理器从繁杂的设备控制事务中解脱出来
\par\twoch{DAM控制器}{\textcolor{red}{设备控制器}}{中断控制器}{I/O端口}
\begin{solution}设备控制器是计算机中的一个实体,其主要职责是控制一个或多个I/O设备,以实现I/O设备和计算机之间的数据交换。它是CPU与I/O设备之间的接口,它接收从CPU发来的命令,并去控制I/O设备工作,以使处理器从繁杂的设备控制事务中解脱出来。
\end{solution}
\question (西南交通大学,2005年)串行输入接口通常需要包含
\par\twoch{串/并转换电路}{并/串转换电路}{\textcolor{red}{A和B}}{以上都不需要}
\begin{solution}串行通信基本原理:并行数据传送和串行数据传送。并行数据传送的特点:各数据位同时传送,控制简单、速度快、效率高;成本高,且距离通常小于30m。计算机内部的数据传送都使用并行数据传送。串行数据传送的特点:数据传送按位数进行,最少只需一根传输线,成本低,可利用电话网等现成的设备;速度慢,控制复杂。距离可从几米到几千公里。计算机通信(串行通信)是指计算机与外部设备或计算机与计算机之间的信息交换。串行接口的基本结构主要是异步接收/发送器,它包括并行数据和串行数据之间的相互转换。
\end{solution}
\question I/O接口中数据缓冲器的作用是
\par\fourch{\textcolor{red}{用来暂存I/O设备和CPU之间传送的数据}}{用来暂存I/O设备的状态}{用来暂存CPU发出的命令}{以上全部}
\begin{solution}I/O接口中: 有用于存放输入/输出数据的数据缓冲器,也称为数据端口。
有用于记录设备或接口状态的状态寄存器,也称为状态端口。
有用于存放控制信息的命令(控制)寄存器,也称为命令(控制)端口。
\end{solution}
\question (北京理工大学,2004年)I/O接口中数据缓冲器的作用是
\par\fourch{\textcolor{red}{用来暂存外设和CPU之间数据的传送}}{用来暂存外设的状态}{用来暂存CPU发出的指令}{以上全部}
\begin{solution}所有的缓存器都是用来临时存放数据的,等CPU或其他设备需要的时候就从缓存里调取,提高速度。
\end{solution}
