\question 若一个用户进程通过read系统调用读取一个磁盘文件中的数据,则下列关于此过程的叙述中,正确的是(
)。 Ⅰ.若该文件的数据不在内存中,则该进程进入睡眠等待状态
Ⅱ.请求read系统调用会导致CPU从用户态切换到核心态
Ⅲ.read系统调用的参数应包含文件的名称
\par\twoch{\textcolor{red}{仅Ⅰ、Ⅱ}}{仅Ⅰ、Ⅲ}{仅Ⅱ、Ⅲ}{Ⅰ、Ⅱ和Ⅲ}
\begin{solution}当用户进程读取的磁盘文件数据不在内存时,转向中断处理,导致CPU从用户态切换到核心态,此时该进程进入睡眠等待状态(其实就是阻塞态,只不过换了个说法),因此Ⅰ、Ⅱ正确。
在调用read之前,需要用open打开该文件,open的作用就是产生一个文件编号或索引指向打开的文件,之后的所有操作都利用这个编号或索引号直接进行,不再考虑物理文件名,所以read系统调用的参数不应包含物理文件名。文件使用结束后要用close关闭文件,消除文件编号或索引。
\end{solution}
\question 在中断发生后,进入中断处理的程序属于
\par\twoch{用户程序}{可能是应用程序,也可能是操作系统程序}{\textcolor{red}{操作系统程序}}{既不是应用程序,也不是操作系统程序}
\begin{solution}中断处理程序只能是操作系统程序,不可能是应用程序。中断处理属于系统中会对系统产生重大影响的动作,因此只允许核心态程序执行;而应用程序通常指用户程序,运行在用户态下,不能进行这些操作。
\end{solution}
