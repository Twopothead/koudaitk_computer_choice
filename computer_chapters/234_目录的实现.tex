\question 文件系统采用二级目录结构,这样可以( )。 \ding{192}.缩短访问文件存储器的时间
\ding{193}.实现文件共享 \ding{194}.节省主存空间 \ding{195}.解决不同用户之间的文件名冲突
\par\twoch{\ding{195}}{\textcolor{red}{\ding{192}和\ding{195}}}{\ding{194}和\ding{195}}{\ding{192}、\ding{193}和\ding{195}}
\begin{solution}二级目录结构的优点有:  提高了检索目录的速度,故\ding{192}正确。 
可以解决文件重名问题,故\ding{195}正确。 
不同用户还可使用不同的文件名来访问系统中的同一个共享文件,但这并不是实现文件共享的方式,故\ding{193}错误。
\ding{194}更是明显错误,目录越多目录文件就越多,占用的主存空间自然就多了,不可能是节省主存空间。所以本题选择B选项。
知识点回顾:
文件的共享方式有:基于索引结点的共享方式和利用符号链实现文件共享。
基于索引结点的共享方式:
将文件的物理地址及其他的文件属性等信息不再放置在目录项中,而是放在索引结点中。目录项中有文件名和指向索引结点的指针,两个不同的目录项只需要指向相同的索引结点即可实现文件共享,即一个共享文件只有一个索引结点,不同的文件名的目录项需要共享的话只需要在目录项中指向该索引结点即可。
在索引结点中再增加一个计数值来统计指向该索引结点的目录项的个数,这样一来需要删除该文件的时候可以判断计数值,只有计数值为1时才删除该索引结点。若计数器大于1,则把计数值减1即可。
利用符号链实现文件共享:
该方法是创建一个称为链接的新目录项。例如,为使用户A能共享用户B的一个文件F,在目录表中为用户A创建一个到文件F链接的新目录项。链接实际上是用另一个文件或目录的指针,可以是绝对路径或相对路径。这样的链接方式被称为符号链接。
\end{solution}
\question 下面的说法中,错误的是( )。
\ding{192}.一个文件在同一系统中、不同介质上的复制文件,应采用同一种物理结构
\ding{193}.对一个文件的访问,常由用户访问权限和用户优先级共同限制
\ding{194}.文件系统采用树形目录结构后,对于不同用户的文件,其文件名应当不同
\ding{195}.为防止系统故障造成系统内文件受损,常采用存取控制矩阵方法保护文件
\par\twoch{\ding{193}}{\ding{192}、\ding{194}}{\ding{192}、\ding{194}、\ding{195}}{\textcolor{red}{全部}}
\begin{solution}\ding{192}错误:一个文件存放在磁带中通常采用连续存放,文件在硬盘上一般采用离散存放,不同的文件系统存放的方式是不一样的。\ding{193}错误:对一个文件的访问,常由用户访问权限和文件属性共同限制。\ding{194}错误:文件系统采用树形目录结构后,对于不同用户的文件,文件名可以相同或不同。\ding{195}错误:常采用备份的方法保护文件,而存取控制矩阵方法用于多用户之间的存取权限保护。
\end{solution}
\question 下面关于目录检索的论述中,正确的是
\par\fourch{由于Hash法具有较快的检索速度,故现代操作系统中都用它来替代传统的顺序检索方法}{在利用顺序检索法时,对树形目录应采用文件的路径名,应从根目录开始逐级检索}{\textcolor{red}{在利用顺序检索法时,只要路径名的一个分量名未找到,便应停止查找}}{在顺序检索法的查找完成后,即可得到文件的物理地址}
\begin{solution}本题考查目录检索的内容。实现用户对文件的按名存取,系统先利用用户提供的文件名形成检索路径,再对目录进行查询。在顺序检索时,路径名的一个分量名未找到,说明路径名中的某个目录或文件不存在,就不需要再查找了。A选项,目录进行查询的方式有两种:线性检索法和Hash方法,线性检索法即root/../filename,现代操作系统中一般采用这种方式查找文件。B选项,为了加快文件查找速度,可以设立当前目录,于是文件路径可以从当前目录进行查找。C选项正确。D选项,在顺序检索法的查找完成后,得到文件的逻辑地址。
\end{solution}
\question 某系统中,一个FCB占用64B,盘块大小为1KB,文件目录中共有3200个FCB,故查找一个文件平均启动磁盘次数为
\par\twoch{50}{64}{\textcolor{red}{100}}{200}
\begin{solution}为了找到一个目录项,平均需要调入盘块N/2次,其中N为目录文件所占用的总盘块数。每调入一个盘块即为启动磁盘一次。
从题目要求看,系统中的所有文件目录皆被存放于一个目录文件中。目录文件所占用的盘块数N可按下式计算。
N=3200/(1024/64)=200 因此,平均需要调入的盘块数为N/2=100。
\end{solution}
\question 在请求分页系统中,页面分配策略与页面置换策略不能组合使用的是()
\par\fourch{可变分配,全局置换}{可变分配,局部置换}{\textcolor{red}{固定分配,全局置换}}{固定分配,局部置换}
\begin{solution}页面分配策略和页面置换策略的概念和相应的方法。
\end{solution}
