\question 对n(n≥2)个权值均不相同的字符构成赫夫曼树。下列关于该赫夫曼树的叙述中,错误的是(
)
\par\twoch{\textcolor{red}{该树一定是一棵完全二叉树}}{树中一定没有度为1的结点}{树中两个权值最小的结点一定是兄弟结点}{树中任一非叶结点的权值一定不小于下一层任一结点的权值}
\begin{solution}赫夫曼树为带权路径长度最小的二叉树,不一定是完全二叉树,故A错误。
赫夫曼树中没有度为1的结点,B正确。
构造赫夫曼树时,最先选取两个权值最小的结点作为左右子树构造一棵新的二叉树,C正确。
从赫夫曼树的构造过程,从森林中选取两棵根结点的权值最小的子树分别作为左、右子树构造一棵新的二叉树T。首先新二叉树T的根结点的权值必然大于其自己左、右子树根结点的权值,因为它等于左、右子树根结点值之和。
我们还需要考虑的是,新二叉树T根结点的兄弟结点N的权值是否大于T的左、右子树根结点的权值。若N的权值小于T左、右子树根结点的权值,则在构造T时,T的左、右子树根结点的权值不是最小的,这则与T1的赫夫曼树构造过程相矛盾。综上,赫夫曼树中任一非叶结点的权值一定不小于下一层任一结点的权值,故D正确。
【总结】
假设有n个权值,则构造出的赫夫曼树有n个叶子结点。n个权值分别设为w1、w2、\ldots{}、wn,则赫夫曼树的构造规则为:
(1)将w1、w2、\ldots{},wn看成是有n棵树的森林(每棵树仅有一个结点);
(2)在森林中选出两个根结点的权值最小的树合并,作为一棵新树的左、右子树,且新树的根结点权值为其左、右子树根结点权值之和;
(3)从森林中删除选取的两棵树,并将新树加入森林;
(4)重复(2)、(3)步,直到森林中只剩一棵树为止,该树即为所求得的赫夫曼树。
\end{solution}
\question (哈尔滨工程大学,2005年)设哈弗曼编码的长度不超过4,若已对两个字符编码为1和01,则还可以对(
)个字符编码
\par\twoch{2}{3}{\textcolor{red}{4}}{5}
\begin{solution}根据前缀编码的规则,剩下的字符编码不能有1和01的前缀,则剩余的只能以00开头,又因为长度不超过4,故剩余的两位一共有四种情况0000、0001、0010、0011
\end{solution}
