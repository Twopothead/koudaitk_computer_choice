\question 在统一编址的方式下,存储单元和I/O设备是靠( )来区分的
\par\twoch{\textcolor{red}{不同的地址码}}{不同的地址线}{不同的指令}{不同的数据线}
\begin{solution}在统一编址情况下,尽管存储单元和I/O设备共用一个存储空间,但是一般都会规定某块地址空间作为I/O设备的地址,凡是在这块地址范围内的访问,就是对I/O设备的访问。
\end{solution}
\question 下列有关I/O接口的叙述中,错误的是( )
\par\fourch{状态端口和控制端口可以合用同一个寄存器}{I/O接口中CPU可访问的寄存器称为I/O端口}{采用独立编址方式时,I/O端口地址和主存地址可能相同}{\textcolor{red}{采用统一编址方式时,CPU不能用访存指令访问I/O端口}}
\begin{solution}本题考查I/O端口及其编址,即使不知道A、B是否正确,也知道答案肯定在C和D中选,因为C、D其实是两个矛盾的选项,C中,当采用独立编址的时候,这样CPU要访问I/O的时候,有可能I/O的端口地址可能和主存的地址是一样的,必须要用不同的指令加以区分,所以不能用访存指令来访问I/O端口,而D中,既然已经统一编址了,那么任意指出一个地址,要么就是主存地址,要么就是I/O端口地址,所以根据地址就能区分,可以用像访存指令那样访问端口。
【总结】状态端口可以和控制端口合用一个寄存器,状态端口用来标记I/O设备的工作状态,通过状态端口,这时候控制器才能通过控制端口做出相应的操作,也就是说状态端口和控制端口的状态实际上是对应的。I/O端口和I/O接口是两个完全不同的概念,I/O端口是指CPU能够读或写的寄存器。
\end{solution}
