\question (中科院,2005年)以下关于链式存储结构的叙述中,( )是不正确的
\par\fourch{节点除自身信息外还包括指针域,因此存储密度小于顺序存储结构}{逻辑上相邻的节点物理上不必邻接}{\textcolor{red}{可以通过计算直接确定第i个节点的存储地址}}{插入、删除运算操作方便,不必移动节点}
\begin{solution}链式存储结构的地址可以是不连续的,因此不能通过相对位移量来确定第i个节点的存储地址
\end{solution}
\question (中南大学,2004年)静态链表中的指针表示的是( )
\par\twoch{下一元素的地址}{内存储器的地址}{\textcolor{red}{下一元素在数组中的位置}}{左链或者右链指向的元素地址}
\begin{solution}虽然静态链表用数组来存储,但其指针域和一般链表一样,指出了表中下一个结点的位置(相对地址),但并未直接给出下一元素的地址,因此本题选C。
【注】本题有时会加入一个易错选的项:数组下标。因为有些教材的关于静态链表的示意图中的指针域就是用数组下标来表示的。但这其实是错的,就像之前解析所说的,静态链表的指针域和一般链表一样,因此给出的是相对地址,而不是下标。
\end{solution}
\question (北京理工大学,2006年)线性表的顺序存储结构是一种( )
\par\twoch{\textcolor{red}{随机存取的存储结构}}{顺序存取的存储结构}{索引存取的存储结构}{散列存取的存储结构}
\begin{solution}由于线性表每一个数据元素的存储位置都和线性表的起始位置相差一个和数据元素在线性表中的位序成正比的常数,由此,只要确定了存储线性表的起始位置,线性表中任一数据元素都可随机存取,所以线性表的顺序存储结构是一种随机存取的存储结构。
\end{solution}
\question (北京航空航天大学,2002年)线性表中各链接点之间的地址( )
\par\twoch{必须连续}{部分地址必须连续}{\textcolor{red}{不一定连续}}{连续与否无所谓}
\begin{solution}比如链式存储结构的链接点地址就不连续。
\end{solution}
