\question (西安交通大学,2004年)某机浮点数格式为:数符1位、阶符1位、阶码5位、尾数9位(共16位)。若机内采用阶移尾补规格化浮点数表示,那么它能表示的最小负数为(
)
\par\fourch{\textcolor{red}{}}{}{}{}
\begin{solution}当阶码为5位时,移码所能表示的最大真值与补码是一样的,即31,故可以排除B和D。上面讲过,当尾数采用补码时,最大的规格化数是-1(1.000000000),故它能表示的最小负数为\includegraphics[width=0.34375in,height=0.15625in]{texmath/ea62b55Cdpi7B3507D-25E7B317D}。
\end{solution}
