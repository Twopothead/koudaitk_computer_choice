\question 当源IP地址和目的IP地址属于同一子网的分组到达时,路由器
\par\twoch{将该分组向其他端口转发}{将该分组向其他子网转发}{将该分组向目的子网转发}{\textcolor{red}{不向任何子网转发}}
\begin{solution}当一个分组到达路由器时,路由器根据目的IP地址和子网掩码,获得目的子网号,再查找路由表,将该分组从相应的端口向相应的子网转发。由于目的IP地址与源IP地址属于同一子网,所以路由器不会将该分组向任何子网或端口转发。
\end{solution}
\question 路由器进行间接交付的对象是
\par\twoch{脉冲信号}{帧}{\textcolor{red}{IP数据报}}{UDP数据报}
\begin{solution}路由器间接交付是在IP层上实行跨网段的交付,所以需要使用IP地址,间接交付的对象是IP数据报。
\end{solution}
\question 在因特网中,IP分组从源结点到目的结点可能要经过多个网络和路由器。在传输过程中,IP分组首部中的
\par\fourch{\textcolor{red}{源地址和目的地址都不会发生变化}}{源地址有可能发生变化而目的地址不会发生变化}{源地址不会发生变化而目的地址有可能不会发生变化}{源地址和目的地址都有可能发生变}
\begin{solution}在因特网中(无需考虑专用网NAT),当一个路由器接收到一个IP分组时,路由器根据IP分组首部中的目的IP地址进行路由选择,但不改变源IP地址和目的IP地址的值。即使IP分组被分片,原IP分组的源IP地址和目的IP地址也将复制到每个分片的首部。因此,在整个传输过程中,IP分组中的源IP地址和目的IP地址都不发生变化。
\end{solution}
\question 以下说法错误的是( )。 Ⅰ.路由选择分直接交付和间接交付
Ⅱ.直接交付时,两台机器可以不在同一物理网段内
Ⅲ.间接交付时,不涉及直接交付 Ⅳ.直接交付时,不涉及路由器
\par\twoch{Ⅰ和Ⅱ}{\textcolor{red}{Ⅱ和Ⅲ}}{Ⅲ和Ⅳ}{Ⅰ和Ⅳ}
\begin{solution}首先路由选择分为直接交付和间接交付,当两台主机在同一物理网段内时,就使用直接交付,反之,使用间接交付,所以Ⅰ是正确的,Ⅱ是错误的;间接传送的最后一个路由器肯定是直接交付,所以Ⅲ错误;直接交付时,是在同一物理网段内,所以不涉及路由器。综上所述,Ⅱ和Ⅲ是错误的。
可能疑问点一:通过网桥连接的网段,从这个网段的一台主机发向另一个网段的主机,中间并没有经过路由器,因为它们处在一个网络中。这也叫做直接交付吗?
解析:直接交付是指在一个物理网络上把数据报从一台主机传输到另一台主机。间接交付是指当源主机和目的主机分别处于不同的物理网络上时,数据报由源主机通过中间的路由器把数据报间接地传输到目的主机的过程。因此即使是网桥连接的,但是都属于同一物理网络,所以仍然属于直接交付。
可能疑问点二:书上说路由器将数据报直接交付给主机A,这里的直接交付不是涉及路由器吗?
解析:直接交付不涉及路由器,意思是比如A和B两点通信,不管A和B是主机还是路由器,只需看中间过程是否经过路由器。经过,就是间接;不经过,就是直接交付。
\end{solution}
