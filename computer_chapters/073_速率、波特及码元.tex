\question 比特的传播时延与链路带宽的关系是
\par\twoch{\textcolor{red}{没有关系}}{反比关系}{正比关系}{无法确定}
\begin{solution}传播时延=信道长度/电磁波在信道上的传播速率,而链路的带宽仅能衡量发送时延,所以说传播时延与链路带宽没有任何关系。
\end{solution}
\question (北京科技大学,2004年)数据传输速率是指( )
\par\twoch{每秒传输的字节数}{电磁波在传输介质上的传播速率}{\textcolor{red}{每秒传输的比特数}}{每秒传输的码元个数}
\begin{solution}数据传输率和带宽是同义词,指每秒传输的比特数。码元传输速率是指每秒传输的码元数。另外,数据传输速率并不是电磁波在传输介质上的传输速率,后者的单位是m/s,两者是完全不同的概念。
\end{solution}
\question (华东理工大学,2006年)采用8种相位,每种相位各有两种幅度的QAM调制方法,在1200Baud的信号传输速率下能达到的数据传输速率为(
)
\par\twoch{2400bit/s}{3600bit/s}{9600bit/s}{\textcolor{red}{4800bit/s}}
\begin{solution}采用8种相位,每种相位有两种幅度的QAM调制方法,所以每个信号可以有16种变化,可以求得每个码元携带4bit的数据,所以数据传输速率为1200×4bit/s=4800bit/s。
\end{solution}
\question 假设某以太网的数据传输速率为10Mbit/s,则其码元传输速率是( )
\par\twoch{10MBaud/s}{\textcolor{red}{20MBaud/s}}{40MBaud/s}{不能确定}
\begin{solution}解答此题前需要清楚以太网的编码方式为曼彻斯特编码,即将每一个码元分成两个相等的间隔,码元1是在前一个间隔为高电平而后一个间隔为低电平;码元0正好相反,从低电平变到高电平。掌握了这些,这道题就很简单了。首先,码元传输速率即为波特率,以太网使用曼彻斯特码,就意味着发送的每一位都有两个信号周期。标准以太网的数据传输速率是10Mbit/s,因此,波特率是数据传输速率的两倍,即20MBaud/s,也就是说,码元的传输速率为20MBaud/s。
\end{solution}
\question (重庆邮电大学,2007年)测得一个以太网数据的波特率为40Mbit/s,那么其数据率是
\par\twoch{10Mbit/s}{\textcolor{red}{20Mbit/s}}{40Mbit/s}{80Mbit/s}
\begin{solution}以太网采用了曼彻斯特编码,意味着每发一位就需要两个信号周期,那么波特率就是数据率的两倍。
\end{solution}
\question 下列因素中,不会影响信道数据传输速率的是( )
\par\twoch{信噪比}{频率带宽}{调制速率}{\textcolor{red}{信号传播速度}}
\begin{solution}本题考察的是信道传输速率,很容易选出答案是D,信号的传播速度是指信号在介质中的传播的速度,而数据的传输速率是指站点发送数据或者接受数据的快慢,是用来衡量站点单位时间内发送数据量的能力,两者是两个相互独立的概念,不要以为传播速度快就会使传输速率变快。
【总结】根据香农定理,信噪比越高,极限传输速率越大;根据奈奎斯特定律,增加频率带宽,也能增加数据传输速率;这两种计算方法是针对在不同的场合下对传输速率的刻画,二者不存在矛盾;《计算机网络高分笔记》中关于数据传输速率有两种解析:比特率和波特率,前者很好理解,后者是指码元的传输速率,将数字信号调制成模拟信号时,调制速率越快,单位时间内产生的码元越多,自然数据的传输速率越快。
\end{solution}
