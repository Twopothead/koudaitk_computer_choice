\question 考虑56Kbit/s调制解调器的性能,驱动程序输出一个字符后就阻塞,当一个字符打印完毕后,产生一个中断通知阻塞的驱动程序,输入下一个字符,然后再阻塞。如果发消息、输出一个字符和阻塞的时间总和为0.1ms,那么由于处理调制解调器而占用的CPU时间比率是(
)(假设每个字符有一个开始位和一个结束位,共占10位)
\par\twoch{\textcolor{red}{56\%}}{57\%}{58\%}{59\%}
\begin{solution}因为一个字符占10位,因此在56Kbit/s的速率下,每秒传送:56000/10=5600个字符,即产生5600次中断。每次中断需0.1ms,故处理调制解调器占用CPU时间总共为5600×0.1ms=
560ms。计算时间比率:560ms/1s=56\%,所以选择A选项。
\end{solution}
