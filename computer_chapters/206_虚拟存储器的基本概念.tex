\question 某计算机的Cache和主存的映射方式采用组相联映射方式,且采取分页管理存储方式,页面大小为128B,Cache容量为64页,按4页分组,主容量为4096页。那么主存地址的位数为(
)。【按字节寻址】
\par\twoch{12}{18}{\textcolor{red}{19}}{20}
\begin{solution}C。 本题有较多干扰条件,基础不扎实的同学容易受到误导。
主存地址,只跟主存总容量,和寻址单位相关。
主存总容量为4096×128B=2\^{}19B。又寻址单位一般默认为字节。则主存中单元数为2\^{}19。则主存地址的位数为19。
\end{solution}
\question 以下关于虚存的说法中错误的是
\par\fourch{虚存的目的是为了给每个用户提供独立的、比较大的编程空间}{虚存中每次访问一个虚地址,至少都要两次访存}{\textcolor{red}{虚存系统中,有时每个用户的编程空间小于实存分配给用户的内存空间}}{都不对}
\begin{solution}C。 虚存其实就是把辅助存储器作为对主存储器的扩充,
向用户提供一个比实际主存大得多的的地址空间,A选项明显是正确的。
若采用页式虚拟存储器,虚存中每次访问一个虚地址,先要访问页表,再访问主存数据,需两次访存。B选项中的``访存''如果改成``访问内存''就是错的,因为多数系统是有设置快表的,在这些系统中也就是一次访问相联存储器,一次访问内存。而用``访存''一定是正确的。
C选项用户的编程空间一定是大于等于系统分配给用户的实际内存空间。故C错误。
\end{solution}
\question 下列关于Cache和虚拟存储器的说法中,错误的有
Ⅰ.当Cache失效(即不命中)时,处理器将会切换进程,以更新Cache中的内容
Ⅱ.当虚拟存储器失效(如缺页)时,处理器将会切换进程,以更新主存中的内容
Ⅲ.Cache和虚拟存储器由硬件和OS共同实现,对应用程序员均是透明的
Ⅳ.虚拟存储器的容量等于主存和辅存的容量之和
\par\twoch{Ⅰ和Ⅳ}{Ⅲ和Ⅳ}{Ⅰ、Ⅱ和Ⅲ}{\textcolor{red}{Ⅰ、Ⅲ和Ⅳ}}
\begin{solution}D。
Cache和虚拟存储器的原理是基于程序访问的局部性原理,但它们实现的方法和作用均不相同。
Cache失效与虚拟存储器失效的处理方法不同,Cache完全由硬件实现,不涉及软件端;虚拟存储器由硬件和OS共同完成,缺页时才会发出缺页中断,故Ⅰ错误,Ⅱ正确,Ⅲ错误。在虚拟存储器中,主存的内容只是辅存的一部分内容,Ⅳ错误。
\end{solution}
