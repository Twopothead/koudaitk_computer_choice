\question 局部性原理是一个持久的概念,对硬件和软件系统的设计和性能都有着极大的影响。局部性通常有两种不同的形式:时间局部性和空间局部性。程序员是否编写出高速缓存友好的代码,就取决于这两方面的问题。对于下面这个函数,说法正确的是
int sumvec(int v{[}N{]}) \{ int i,sum=0; for(i=0;i
\par\fourch{对于变量i和sum,循环体具有良好的空间局部性}{对于变量i、sum和v[N],循环体具有良好的空间局部性}{\textcolor{red}{对于变量i和sum,循环体具有良好的时间局部性}}{对于变量i、sum和v[N],循环体具有良好的时间局部性}
\begin{solution}C。
对于局部变量i和sum,循环体有良好的时间局部性。实际上,因为它们都是局部变量,任何合理的优化编译器都会把它们缓存在寄存器文件中,也就是存储器层次的最高层。所以A,B错。
现在考虑对向量v的步长为1的应用。一般而言,如果一个高速缓存的块大小为B字节,那么一个步长为k的引用模式(这里k是以字为单位的)平均每次循环迭代会有min(1,
(wordsize×k)/B)次缓存不命中。当k=1时,它取最小值,所以对v的步长为1的引用确实是高速缓存友好的,即拥有良好的空间局部性。所以D错,只有C的说法是正确的。
\end{solution}
\question 某计算机的存储系统由Cache-主存系统构成,Cache的存取周期为10ns,主存的存取周期为50ns。在CPU执行一段程序时,Cache完成存取的次数为4800次,主存完成的存取次数为200次,该Cache-主存系统的效率是(
)。【注:计算机存取时,同时访问Cache和主存,Cache访问命中,则主存访问失效;Cache访问未命中,则等待主存访问】
\par\twoch{0.833}{0.856}{0.958}{\textcolor{red}{0.862}}
\begin{solution}D。
这类题,教材有两种公式,其实对应不同的前提。涉及的缩写说明:h为Cache命中率,tc为一次Cache访问时间,tm为一次主存访问时间。
①
系统先进行Cache访问,Cache命中则结束,Cache不命中,再进行主存访问。则平均访问时间计算公式为h*tc+(1-h)*(tc+tm)。
②
系统同时进行Cache访问和主存访问,Cache命中,则主存访问失效,Cache未命中,则等待主存访问。则平均访问时间计算公式为h*tc+(1-h)*tm。
本题属于第②种情况,命中率=4800/(4800+200)=0.96,平均访问时间=0.96*10ns+
(1-0.96)*50ns=11.6ns,故效率=10/11.6=0.862。
\end{solution}
\question 在Cache和主存构成的两级存储体系中,Cache的存取时间是100ns,主存的存取时间是1000ns,如果希望有效(平均)存取时间不超过Cache存取时间的115\%,则Cache的命中率至少应为(
)。【注:计算机存取时,先访问Cache,访问未命中,再访问主存】
\par\twoch{90.5\%}{95.5\%}{\textcolor{red}{98.5\%}}{99.5\%}
\begin{solution}C。
设Cache命中率为a,则(1000+100)(1-a)+100a≤115,解得a≥0.985,故至少为98.5\%。
注:虽然也可以采用同时访问Cache和主存的方式,此时不命中的访问时间为1000ns。题设中已经说明,Cache不命中的时间为访问Cache和主存的时间之和。
\end{solution}
\question (中国科学院,2001年)在Cache和主存构成的两组存储体系中,Cache的存取时间为100ns,主存的存取时间为1000ns,如果希望有效(平均)存取时间不超过115ns,则Cache的命中率至少应为
\par\twoch{90\%}{98\%}{95\%}{\textcolor{red}{99\%}}
\begin{solution}D。
假设命中率为A,100×A+1000(1-A)≤115,可以得出A≈0.983,故Cache的命中率至少应为99\%。
\end{solution}
