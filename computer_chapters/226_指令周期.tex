\question 指令周期是指
\par\fourch{CPU从主存取出一条指令的时间}{CPU执行一条指令的时间}{\textcolor{red}{CPU从主存取出一条指令加上CPU执行这条指令的时间}}{时钟周期时间}
\begin{solution}C。
系统主时钟一个周期信号所持续的时间称为时钟周期,是处理操作最基本的单位。
机器周期指的是完成一个基本操作的时间单元,如取指周期、取数周期。
微处理器执行一条指令的时间(包括取指和执行指令所需的全部时间)称为指令周期。
时钟周期、总线周期和指令周期之间的关系是:一个机器周期由若干个时钟周期组成,一个指令周期由若干个总线周期组成。
关于机器周期和总线周期的关系:机器周期指的是完成一个基本操作的时间,这个基本操作有时可能包含总线读写,因而包含总线周期,但是有时可能与总线读写无关,所以,并无明确的相互包含的关系。
\end{solution}
