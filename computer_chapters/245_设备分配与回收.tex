\question 在下列问题中,( )不是设备分配中应考虑的问题
\par\twoch{\textcolor{red}{及时性}}{设备的固有属性}{设备独立性}{安全性}
\begin{solution}设备的固有属性决定了设备的使用方式;设备独立性可以提高设备分配的灵活性和设备的利用率;设备安全性可以保证分配设备时不会导致死锁等问题。设备分配时通常不考虑及时性。
\end{solution}
\question 设备分配程序分配外部设备时,按哪种顺序分配下面三个资源?( ) \ding{192}.控制器
\ding{193}.设备 \ding{194}.通道
\par\twoch{\ding{192}→\ding{193}→\ding{194}}{\textcolor{red}{\ding{193}→\ding{192}→\ding{194}}}{\ding{193}→\ding{194}→\ding{192}}{\ding{194}→\ding{193}→\ding{192}}
\begin{solution}基本的设备分配程序。
通过一个具有I/O通道的系统的例子来介绍设备分配过程。当某进程提出I/O请求后,系统的设备分配程序可按下述步骤进行设备分配。
(1)分配设备。
首先根据I/O请求中的物理设备名,查找系统设备表SDT,从中找出该设备的设备控制表DCT(Device
Control
Table),再根据DCT中的设备状态字段,可知该设备是否正忙。若忙,便将请求I/O的进程的PCB挂在设备等待队列上;否则,便按照一定的算法来计算本次设备分配的安全性。如果不会导致系统进入不安全状态,便将设备分配给请求进程;否则,仍将其PCB插入设备等待队列。
(2)分配控制器。
在系统把设备分配给请求I/O的进程后,再到其DCT中找出与该设备连接的控制器的控制器控制表(Controller
Control
Table,COCT),从COCT的状态字段中可知该控制是否忙碌。若忙,便将请求I/O进程的PCB挂在该控制器的等待队列上;否则,便将该控制器分配给进程。
(3)分配通道。
在该COCT中又可以找到与该控制器连接通道的通道控制表(Channel Control
Table,CHCT),再根据CHCT内的状态信息,可知该通道是否忙碌。若忙,便将请求I/O的进程挂在该通道的等待队列上;否则,将该通道分配进程。只有在设备、控制器和通道三者都分配成功时,这次的设备才算成功。然后,便可启动该I/O设备进行数据传送。
\end{solution}
