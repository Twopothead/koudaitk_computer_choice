\question 下列关于操作系统的论述中,正确的是( )
\par\fourch{\textcolor{red}{对于批处理作业,必须提供相应的作业控制信息}}{对于分时系统,不一定全部提供人机交互功能}{从响应角度看,分时系统与实时系统的要求相似}{在采用分时操作系统的计算机系统中,用户可以独占计算机操作系统中的文件系统}
\begin{solution}分时系统必须有交互功能,实时系统对于响应的要求比分时系统更高。在分时系统中,用户不会独占文件系统,这是多用户共享的。
\end{solution}
\question 下列特征中不属于分时系统的是
\par\twoch{及时性}{多路性}{\textcolor{red}{调度性}}{独占性}
\begin{solution}分时操作系统具有四个特征:多路性、交互性、独占性、及时性。详见本书分时系统讲解部分。
\end{solution}
\question 多道程序设计技术能有效提高系统的吞吐量和改善资源利用率,实现这技术需要考虑下列(
)方面的问题。 \ding{192}.如何分配处理机、内存和I/O设备
\ding{193}.如何提高系统安全性,保证不被黑客攻击 \ding{194}. 如何组织不同类型的作业
\ding{195}.如何减少单个作业占用的内存大小,以载入更多的作业
\par\twoch{\ding{193}、\ding{194}、\ding{195}}{\textcolor{red}{\ding{192}、\ding{194}}}{\ding{192}、\ding{195}}{\ding{193}、\ding{194}}
\begin{solution}多道程序设计的目的在于提高系统的吞吐量,改善资源利用率。由于主存中同时存在多个作业,因此要考虑以下几个方面的问题:
应如何分配处理机,以使处理机既能满足各程序运行的需要又有较高的利用率,将处理机分配给某程序后,应何时收回等问题。
如何为每道程序分配必要的内存空间,使它们各得其所又不致因相互重叠而失去信息,应如何防止因某个程序出现异常情况而破坏其他程序等问题。
系统中可能有多种类型的I/O设备供多道程序共享,应如何分配这些I/O设备,如何做到既方便用户对设备的使用,又能提高设备的利用率。
在现在计算机系统中,通常都存放着大量的程序和数据。应如何组织它们才能便于用户使用,并能保证数据的安全性和一致性。
对于系统中的各种应用程序,它们有的属于计算型,有的属于I/O型,有些作业既重要又紧迫,有的作业又要求系统能及时响应,这时应如何组织这些作业。
从上述总结可以看出,选项中\ding{192}、\ding{194}项都是多道程序设计需要考虑的问题;虽然安全性也是多道程序设计要考虑的一个因素,但是这个安全性是指用户在使用数据时不会因为系统组织数据的方法而出现数据不一致或者被错误修改等情况,与题目中所说的黑客攻击不同,网络安全并不是多道程序要考虑的问题,因此\ding{193}不对;同样,如何减少作业的大小也不是多道程序设计要考虑的,而是在设计作业的时候要考虑的,系统只负责执行作业而不会考虑如何设计作业,因此\ding{195}错误。答案选择B选项。
\end{solution}
\question 批处理系统的主要缺点是
\par\twoch{CPU利用率}{不能并发执行}{\textcolor{red}{缺少交互性}}{以上都不是}
\begin{solution}根据在内存中允许存放的作业数,批处理系统又分为单道批处理系统和多道批处理系统。
现在的批处理系统主要指多道批处理系统,它通常用在以科学计算为主的大中型计算机上。由于多道程序能交替使用CPU,提高了CPU及其他系统资源的利用率,同时也提高了系统的效率。多道批处理系统的缺点是延长了作业的周转时间,用户不能进行直接干预,缺少交互性,不利于程序的开发与调试。
\end{solution}
\question 下面关于操作系统的叙述中,正确的是
\par\fourch{\textcolor{red}{批处理作业必须具有作业控制信息}}{分时系统不一定都具有人机交互功能}{从响应时间的角度来看,实时系统与分时系统差不多}{由于采用了分时技术,用户可以独占计算机的资源}
\begin{solution}操作系统的基本类型有以下三类:批处理系统、实时操作系统和分时操作系统。每种操作系统具有不同的特性。
批处理系统的特性如下:  用户脱机使用计算机。  成批处理。 
多道程序运行。 实时操作系统的特性如下:  及时响应。  高可靠性。
分时操作系统的特性如下:  多路性。  交互性。  独占性。 
及时性(注意与实时操作系统及时性的区别)。
批处理系统具有脱机使用的特点,因此当用户提交作业时,必须同时提交对于作业的控制信息,提交之后用户通常不会再干预作业的执行,因此A选项正确;分时系统具有交互性,目的就是让多用户同时使用计算机,如果没有人机交互就无意义了,因此B选项错误;实时系统和分时系统的响应时间差别比较大,通常不是一个数量级,实时系统对响应时间要求更高,因此C选项错误;分时技术的目的是多用户共享计算机资源,虽然使每个用户都感觉自己在独占计算机,实际上并没有独占,因此D选项错误。综上分析,答案为A。
操作系统的分类以及不同种类操作系统的特点,是大纲要求的一个较为简单的知识点,本题对这个知识点进行了小结。
\end{solution}
\question 下面6个系统中,必须是实时操作系统的有( )个。 计算机辅助设计系统
航空订票系统 过程控制系统 机器翻译系统 办公自动化系统 计算机激光拍照系统
\par\twoch{1}{2}{\textcolor{red}{3}}{4}
\begin{solution}订票时,必须及时刷新系统中剩余票数,对时间有要求;过程控制对时间有严格要求;
办公自动化对响应时间有要求。
\end{solution}
