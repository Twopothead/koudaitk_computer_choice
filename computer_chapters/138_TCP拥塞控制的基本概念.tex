\question 有一个TCP连接,当其拥塞窗口为32个分组大小时超时。假设网络的RTT是固定的5s,不考虑比特开销,即分组不丢失,则系统在超时后处于慢启动阶段的时间有(
)s
\par\twoch{10}{\textcolor{red}{20}}{30}{40}
\begin{solution}当超时时,其阈值为32/2=16。按照慢开始算法,发送窗口的初始值设置为1,然后依次增大为2、4、8、16,需要经过4次RTT的时间达到阈值16,也即慢启动阶段结束,所以系统在超时后处于慢启动阶段的时间有4RTT的时间,即20s。
\end{solution}
\question 假设在没有发生拥塞的情况下,在一条往返时间RTT为10ms的线路上采用慢开始控制策略。如果接收窗口的大小为24KB,最大报文段MSS为2KB。那么需要(
)ms发送方才能发送出一个完全窗口
\par\twoch{30ms}{\textcolor{red}{40ms}}{50ms}{60ms}
\begin{solution}所谓``慢开始''就是由小到大逐渐增大发送端的拥塞窗口数值。慢开始算法的基本原理是:在连接建立时,将拥塞窗口的大小初始化为一个MSS的大小,此后拥塞窗口每经过一个RTT,就按指数规律增长一次,直到出现报文段传输超时或达到所设定的慢开始门限值ssthresh。
本题中,按照慢开始算法,发送窗口的初始值为拥塞窗口的初始值即MSS的大小2KB,然后依次增大为4KB、8KB、16KB,然后是接收窗口的大小24KB,即达到第一个完全窗口。因此,达到第一个完全窗口所需的时间为4×RTT=40ms。
\end{solution}
\question 在一个TCP连接中,MSS为1KB,当拥塞窗口为34KB时收到了3个冗余ACK报文。如果在接下来的4个RTT内报文段传输都是成功的,那么当这些报文段均得到确认后,拥塞窗口的大小是
\par\twoch{8KB}{16KB}{20KB}{\textcolor{red}{21KB}}
\begin{solution}题干中很明确地说明收到了3个冗余ACK报文,说明接下来需要使用快恢复算法。先将慢开始门限设置为17KB,此时与慢开始算法不同的是,初始拥塞窗口不再是1个MSS的大小,而是直接将拥塞窗口设置为新的慢开始门限,即17KB,所以后面4次成功传输将分别以17KB、18KB、19KB、20KB作为拥塞窗口的大小。此时,当第4个报文段传输成功时,拥塞窗口将变成21KB。
\end{solution}
\question 在 TCP 中,发送方的窗口大小是由( )的大小决定的
\par\twoch{仅接收方允许的窗口}{接收方允许的窗口和发送方允许的窗口}{\textcolor{red}{接收方允许的窗口和拥塞窗口}}{发送方允许的窗口和拥塞窗口}
\begin{solution}记忆题,发送方的窗口大小=min{[}接收方允许的窗口,拥塞窗口{]}。
\end{solution}
