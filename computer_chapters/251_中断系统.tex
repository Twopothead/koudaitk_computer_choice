\question 中断扫描机构是( )扫描一次中断寄存器
\par\twoch{每隔一个时间片}{\textcolor{red}{每条指令执行周期内最后时刻}}{每当进程释放CPU}{每产生一次中断}
\begin{solution}处理器执行完一条指令后,硬件的中断装置(中断扫描机构)立即检查有无中断事件发生。若无中断事件发生,则处理器继续执行下面的指令;若有中断事件发生,则暂停现行进程的运行,而让操作系统中的中断处理程序占用处理器,这一过程称为``中断响应''。
\end{solution}
\question 若有多个中断同时发生,系统将根据中断优先级响应优先级最高的中断请求。若要调整中断事件的处理次序,可以利用(
)
\par\twoch{中断嵌套}{中断向量}{中断响应}{\textcolor{red}{中断屏蔽}}
\begin{solution}中断屏蔽技术向使用者提供了一种手段,即可以用程序控制中断系统,动态地调度多重中断优先处理的次序,从而提高了中断系统的灵活性。
归纳总结:中断处理次序和中断响应次序是两个不同的概念,中断响应次序是由硬件排队电路决定的,无法改变。但是,中断处理次序可以由屏蔽码来改变,这里所说的改变优先次序指改变中断的处理次序。
\end{solution}
\question 单级中断系统中,中断服务程序执行顺序是( ~)。 Ⅰ.保护现场 Ⅱ.开中断
Ⅲ.关中断 Ⅳ.保存断点 Ⅴ.中断事件处理 Ⅵ.恢复现场 Ⅶ.中断返回
\par\fourch{\textcolor{red}{Ⅰ→Ⅴ→Ⅵ→Ⅱ→Ⅶ}}{Ⅲ→Ⅰ→Ⅴ→Ⅶ}{Ⅲ→Ⅳ→Ⅴ→Ⅵ→Ⅶ}{Ⅳ→Ⅰ→Ⅴ→Ⅵ→Ⅶ}
\begin{solution}在单级中断系统中,整个中断处理过程为:①关中断;②保存断点;③识别中断源;④保护现场;⑤中断事件处理;⑥恢复现场;⑦开中断;⑧中断返回。其中①、②、③由中断隐指令(硬件)完成,而后面的④~⑧都是由中断服务程序完成。
【总结】 ~ ~ ~
~程序中断有单级中断和多级中断之分,单级中断在CPU执行中断服务程序的过程中不能被再打断,即不允许中断嵌套;而多级中断在执行某个中断服务程序的过程中,CPU可以去响应级别更高的中断请求,即允许中断嵌套。
中断的条件:中断请求要获得CPU响应,必须满足3个条件。 ~ ~ ~
~1)中断屏蔽触发器处于非屏蔽状态,使外设的中断请求信号能发给CPU。 ~ ~ ~
~2)中断允许触发器处于开中断状态,使CPU允许响应中断。 ~ ~ ~
~3)一条指令执行完毕。
\end{solution}
