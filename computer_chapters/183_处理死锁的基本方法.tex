\question 某个系统采用如下资源分配策略:如果一个进程提出资源请求得不到满足,而此时没有由于等待资源而被阻塞的进程,则自己就被阻塞。若当此时已有等待资源而被阻塞的进程,则检查所有由于等待资源而被阻塞的进程,如果它们有申请进程所需要的资源,则将这些资源剥夺并分配给申请进程。这种策略会导致(
)
\par\twoch{死锁}{死锁}{回退}{\textcolor{red}{饥饿}}
\begin{solution}本题策略不会导致死锁,因为破坏了不剥夺这一条件。但是这种分配策略会导致某些进程长时间等待所需资源,因为被阻塞进程所持有的资源可以剥夺,所以被阻塞进程的资源数量在阻塞期间可能会变少,若系统不断出现其他进程申请资源,某些被阻塞进程会被一直剥夺资源,同时系统无法保证在有限时间内将这些阻塞进程唤醒。
\end{solution}
