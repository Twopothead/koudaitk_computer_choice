\question (武汉理工大学,2004年)在存储管理中,采用覆盖和交换技术的目的是
\par\twoch{提高CPU的效率}{\textcolor{red}{节约主存空间}}{物理上扩充主存容量}{实现主存共享}
\begin{solution}覆盖和交换都是内存扩充技术,都是实现内存和外存的信息交换。覆盖和交换都是为了在较小的内存空间中用重复使用的方法来节省存储空间,逻辑上扩充了主存。对换是以进程为基本单位的交换。覆盖是以进程的互不相关的局部为单位进行的交换。
\end{solution}
\question 在使用交换技术时,如果一个进程正在( )时,则不能交换出主存
\par\twoch{创建}{\textcolor{red}{I/O操作}}{处于临界段}{死锁}
\begin{solution}进程正在进行I/O操作时不能换出主存,否则它的I/O数据区将被新换入的进程占用,导致错误。不过可以在操作系统中开辟I/O缓冲区,将数据从外设输入或将数据输出到外设的I/O活动在系统缓冲区中进行,这时在系统缓冲区与外设进行I/O操作时,进程交换不受限制。
\end{solution}
\question 下列关于交换与覆盖的叙述正确的有( )。
Ⅰ.覆盖技术仍适用于段页式存储管理
Ⅱ.在以进程为单位进行对换时,每次都需将整个进程换出
Ⅲ.挂在就绪队列上等待的进程有可能被交换到内存
Ⅳ.在请求分页系统的页表中访问字段表示该页在调入内存后是否被修改过,它决定了在对该页进行对换操作时,是否要写回到外存上
\par\twoch{Ⅰ、Ⅲ}{Ⅱ、Ⅲ}{\textcolor{red}{Ⅲ}}{Ⅲ、Ⅳ}
\begin{solution}Ⅰ错误,覆盖技术主要用于早期的操作系统中,并不适合段页式存储管理。而对换技术在现代操作系统中仍有较强的生命力。覆盖技术的基本思想:一个程序不需要把所有的指令和数据都装入内存,而是将程序划分为若干个功能相对独立的程序段,按照程序的逻辑结构让那些不会同时执行的程序段共享同一块内存区。这样使用户感觉到内存扩大了,从而达到内存扩充的目的。覆盖技术要求程序员提供一个清晰地覆盖结构,即由程序员来完成把一个程序划分为不同的程序段并规定好他们的执行和覆盖顺序。操作系统根据程序员提供的覆盖结构来完成执行过程程序段的覆盖(十分麻烦,果断被弃用)。覆盖和交换都是内存扩充技术,都是实现内存和外存的信息交换。对换是以进程为基本单位的交换。覆盖是以进程的互不相关的局部为单位进行交换。
Ⅱ错误,在以进程为单位进行对换时,并非每次都要将整个进程换出。这是因为:
(1)从结构上来看,进程是由程序、数据和进程控制块组成的。其中,进程控制块通常都长驻内存而不被换出。
(2)如果进程对应的程序或数据正被其他进程所共享,则也不能换出。
Ⅲ正确,为了使那些暂时不能运行的进程(当然包括就绪状态的进程)不再占用宝贵的内存资源,而将他们调至外存上去等待,把此时的进程状态称为就绪驻外存状态或挂起状态。
Ⅳ错误,这里混淆了访问字段和修改位的作用。访问字段用于记录本页在一段时间内被访问的次数。修改位用于表示该页在调入内存后是否被修改过。由于内存中的每一页都在外存上保留一份副本,该位决定了是否要将该页写回到外存上,以保证外存中所保留的始终是最新副本。
\end{solution}
\question 以下叙述错误的是
\par\twoch{\textcolor{red}{覆盖对程序员是透明的}}{交换对程序员是透明的}{在分页系统环境下,分页对程序员是透明的}{联想寄存器的地址变换对操作系统是透明的}
\begin{solution}覆盖对程序员是不透明的。为了节省内存,提高覆盖的效果,用户在编制程序时就要安排好程序的覆盖结构,覆盖的目的是希望运行更大的程序。
同为节省内存,覆盖技术用于一个作业的内部,交换技术则用于不同的作业。交换对程序员是透明的,交换的目的是希望能够运行更多的程序。
因此A错误,B正确。
C正确,分页的实现是由操作系统自动实现的,因此分页对程序员是透明的。
D正确,联想寄存器的地址变换是由硬件实施的,不需要通过操作系统软件(指令)实施。但在实施快表的淘汰算法时,要通过操作系统实施。
\end{solution}
\question 在现代计算机系统中,存储器是十分重要的资源,能否合理有效地使用存储器在很大程度上反映了操作系统的性能,并直接影响到整个计算机系统作用的发挥。可以通过哪些途径来提高主存利用率(
)。 Ⅰ.将连续分配方式改为离散分配方式 Ⅱ.增加对换机制
Ⅲ.引入虚拟存储机制 Ⅳ.引入存储器共享机制
\par\twoch{Ⅰ、Ⅱ和Ⅲ}{Ⅰ、Ⅱ和Ⅳ}{Ⅱ、Ⅲ和Ⅳ}{\textcolor{red}{全是}}
\begin{solution}Ⅰ对,将连续分配方式改为离散分配方式以减少内存的零头。
Ⅱ对,增加对换机制,将那些暂时不能运行的进程或暂不需要的程序或数据换出至外存,以腾出内存来装入运行的程序。
Ⅲ对,引入虚拟存储机制,使更多的作业能够装入内存,提高CPU和内存利用率。
引入动态链接机制,当程序在运行中需要调用某段程序时才将该程序装入内存,从而避免装入不会用到的程序段和数据。
Ⅳ对,引入存储器共享机制,允许一个正文段或数据段被若干程序共享以消除内存中的重复拷贝现象。
\end{solution}
\question 对外存对换区的管理应以( )为主要目标
\par\twoch{提高系统吞吐量}{提高存储空间的利用率}{降低存储费用}{\textcolor{red}{提高换入、换出速度}}
\begin{solution}内存管理是为了提高内存利用率。引入覆盖和交换技术就是为了在较小的内存空间中用重复使用的方法来节省存储空间。覆盖和交换技术付出的代价是,需要消耗更多的处理机时间,它实际上是一种以时间换空间的技术。为此,从节省处理机时间来讲,换入、换出速度越快,付出的时间代价就越小,反之就越大,当时间代价达到一定程度时,覆盖和交换技术就没有意义了。
\end{solution}
