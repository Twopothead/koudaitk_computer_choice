\question 总线的通信控制主要解决( )问题
\par\twoch{由哪个主设备占用总线}{通信双方如何获知传输开始和结束}{通信过程中双方如何协调配合}{\textcolor{red}{B和C}}
\begin{solution}总线控制包括两个方面:判优控制和通信控制。A选项是属于判优控制;而总线通信控制主要解决通信双方如何获知传输开始和传输结束,以及通信双方如何协调和如何配合,故B和C选项属于总线通信控制。
\end{solution}
\question 某同步总线采用数据总线和地址总线复用方式,其中地址/数据线有32根,总线时钟频率为66MHz,每个时钟周期传送两次数据(上升沿和下降沿各传送一次数据),该总线的最大数据传输率(总线带宽)是(
)
\par\twoch{132 MB/s}{264 MB/s}{\textcolor{red}{528 MB/s}}{1056 MB/s}
\begin{solution}本题考察总线的同步定时方式,地址线/数据线32根,则每次传输的数据占4B,总线的时钟频率为66MHz,即一秒内有66M个时钟周期,既然题目已经告诉我们每个时钟周期传送两次数据,则1秒内传送数据=66M×2×4B=528MB,即最大传输速率是528MB/s。
【总结】此题包含两个比较易混的地方:第一是同步和异步;第二是数据传输率中的MB/s和计算机存储容量当中MB这两个量纲的区别;
第一:只有采用同步方式,才可以用时钟频率来衡量1秒传输了数据多少次,因为同步定时方式是指系统采用一个统一的时钟信号来协调收发双发的传送。而异步方式采用的是应答模式。
第二:只要涉及到传输速率、频率方面的时候,这里的1M=1000K,在涉及到存储容量方面时,这里的1M=1024K,这个概念曾经在真题中有考查过,记住就好。
\end{solution}
\question 一次总线事务中,主设备只需给出一个首地址,从设备就能从首地址开始的若干连续单元读出或写入多个数据。这种总线事务方式称为(
)
\par\twoch{并行传输}{串行传输}{\textcolor{red}{突发传输}}{同步传输}
\begin{solution}这里考察的是突发式传输的定义。 【总结】
并行传输:并行传输是在传输中有多个数据位同时在设备之间进行的传输,使表示一个数据的所有数据位能同时沿着各自的信道并排的传输。并行传输时,一次可以传一个字符,收发双方不存在同步的问题,而且速度快、控制方式简单。但是,并行传输需要多个物理通道。
串行传输:串行传输是指数据的二进制代码在一条物理信道上以位为单位按时间顺序逐位传输的方式。串行传输时,发送端逐位发送,接收端逐位接受,同时,还要对所接受的字符进行确认,所以收发双方要采取同步措施。串行传输相对并行传输而言,传输速度慢,但只需一条物理信道,线路投资小,易于实现,特别适合远距离传输。串行传输是目前数据传输的主要方式。数据的传输在一条信号线路上按位进行的传输方式。
同步传输:同步传输方式是指系统采用一个统一的时钟信号来协调收发双发的传送。
\end{solution}
