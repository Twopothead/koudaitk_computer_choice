\question 下列哪些存储分配方案可能使系统抖动( )。 Ⅰ.动态分区分配
Ⅱ.简单页式分配 Ⅲ.虚拟页式 Ⅳ.简单段页式 Ⅴ.简单段式 Ⅵ.虚拟段式
\par\twoch{Ⅰ、Ⅱ和Ⅴ}{Ⅲ和Ⅳ}{只有Ⅲ}{\textcolor{red}{Ⅲ和Ⅵ}}
\begin{solution}要通过对存储分配的理解来推断系统是否会发生抖动,所以本题也需要了解不同的存储分配方案的内容。抖动现象是指刚刚被换出的页很快又要被访问,为此又要换出其他页,而该页又很快被访问,如此频繁地置换页面,以致大部分时间都花在页面置换上。对换的信息量过大,内存容量不足不是引起系统抖动现象的原因,而选择的置换算法不当才是引起抖动的根本原因。例如,先进先出算法就可能会产生抖动现象。本题中只有虚拟页式和虚拟段式才存在换入换出的操作,简单页式和简单段式已经全部将程序调入内存,因此不需要置换,也就没有了抖动现象。
\end{solution}
\question 下列说法正确的有( ) Ⅰ.先进先出(FIFO)页面置换算法会产生Belady现象
Ⅱ.最近最少使用(LRU)页面置换算法会产生Belady现象
Ⅲ.在进程运行时,如果它的工作集页面都在虚拟存储器内,则能够使该进程有效地运行,否则会出现频繁的页面调入/调出现象
Ⅳ.在进程运行时,如果它的工作集页面都在主存储器内,则能够使该进程有效地运行,否则会出现频繁的页面调入/调出现象
\par\twoch{Ⅰ、Ⅲ}{\textcolor{red}{Ⅰ、Ⅳ}}{Ⅱ、Ⅲ}{Ⅱ、Ⅳ}
\begin{solution}Ⅰ正确,举个例子:使用先进先出(FIFO)页面置换算法,页面引用串为1、2、3、4、1、2、5、1、2、3、4、5时,当分配3帧时产生9次缺页中断,分配4帧时产生10次缺页中断。Ⅱ错误,最近最少使用(LRU)页面置换算法没有这样的问题。Ⅲ错误,Ⅳ正确:若页面在内存中,不会产生缺页中断,也不会出现页面的调入/调出。虚拟存储器的说法不正确。
\end{solution}
\question (湖南大学,2005年)在页面置换算法中,存在Belady现象的算法是
\par\twoch{最佳页面置换算法(OPT)}{\textcolor{red}{先进先出置换算法(FIFO)}}{最近最久未使用(LRU)}{最近未使用算法(NRU)}
\begin{solution}所谓Belady现象是指采用FIFO算法时,如果对一个进程未分配它所要求的全部页面,有时就会出现分配的页面数增多但缺页率反而提高的异常现象。Belady现象的原因是FIFO算法的置换特征与进程访问内存的动态特征是矛盾的,即被置换的页面并不是进程不会访问的。OPT、LRU、NRU等页面置换算法都遵循了局部性原理,不会出现Belady异常。
\end{solution}
\question 在页面置换算法中,存在Belady现象的算法是
\par\twoch{最佳页面置换算法(OPT)}{\textcolor{red}{先进先出置换算法(FIFO)}}{最近最久未使用(LRU)}{最近未使用算法(NRU)}
\begin{solution}所谓Belady现象是指:采用FIFO算法时,如果对---个进程未分配它所要求的全部页面,有时就会出现分配的页面数增多但缺页率反而提高的异常现象。Belady现象的原因是FIFO算法的置换特征与进程访问内存的动态特征是矛盾的,即被置换的页面并不是进程不会访问的。
OPT、LRU、NRU等页面置换算法都遵从了局部性原理,不会出现Belady异常。
\end{solution}
\question 当系统发生抖动(thrashing)时,可以采取的有效措施是( )。
Ⅰ.撤销部分进程 Ⅱ.增加磁盘交换区的容量 Ⅲ.提高用户进程的优先级
\par\twoch{\textcolor{red}{仅Ⅰ}}{仅Ⅱ}{仅Ⅲ}{仅Ⅰ,Ⅱ}
\begin{solution}在采用请求调页存储管理系统中,内存中只存放进程的部分页,缺页时进行页面置换,将所缺页调入替换掉暂时不用的页。抖动现象是指刚刚被换出的页面马上就被访问,因此马上又要换入,使系统频繁置换页面,将大部分时间用于处理页面置换上,降低系统效率。
对于抖动现象,产生的根本原因在于进程缺页次数过多,降低缺页次数的手段都能够有效防止抖动发生,因此是否减少缺页次数就成为判断的依据。
下面据此分析各选项:
A选项:撤销部分进程可以减少系统中进程的数量,相当于减少了``缺页中断源'',可以有效降低缺页次数,防止系统抖动。
B选项:改变优先级只是改变了进程执行的顺序,但依然会产生相同数量的缺页次数,因此无效。
C选项:增大交换区容量比较容易让读者误认为是增大了进程的驻留集,其实增大交换区的含义是增大进行页面置换时的临时存储区,在各种书中这部分空间默认总是足够大的,增加交换区最多也只是能提高页面置换速度,但对于减少缺页次数没有帮助,所以本选项也无效。
因此本题答案选A。 【总结】
减少抖动的方法不仅有减少进程数一种,还有增大进程驻留集、改变缺页置换策略和优化程序结构等,这些都可以有效防止抖动。
\end{solution}
\question 在页式虚拟存储管理系统中,采用某些页面置换算法,会出现Belady异常现象,即进程的缺页次数会随着分配给该进程的页框个数的增加而增加。下列算法中,可能出现Belady异常现象的是(
)。 I.LRU算法 II.FIFO算法 III.OPT算法
\par\twoch{\textcolor{red}{仅II}}{仅I、II}{仅I、III}{仅II、III}
\begin{solution}所谓Belady现象是指:采用FIFO算法时,如果对一个进程未分配它所要求的全部页面,有时就会出现分配的页面数增多但缺页率反而提高的异常现象。因此能够出现Belady异常现象的只有FIFO算法。
\end{solution}
