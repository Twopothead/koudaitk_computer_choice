\question (南京理工大学,2002年)动态重定位是在作业的( )中进行的
\par\twoch{编译过程}{装入过程}{链接过程}{\textcolor{red}{执行过程}}
\begin{solution}(1)静态重定位在程序装入内存的过程中完成,是指在程序开始运行前,程序中的各个地址有关的项均已完成重定位。地址变换通常是在装入时一次完成的,以后不再改变,故称为静态重定位。
(2)动态重定位不是在程序装入内存时完成的,而是CPU每次访问内存时由动态地址变换机构(硬件)自动把相对地址转换为绝对地址。动态重定位需要软件和硬件相互配合完成。
\end{solution}
\question (华中科技大学,2001年)在某系统采用基址、限长寄存器的方法来保护存储信息,判断是否越界的判断式为
\par\fourch{\textcolor{red}{0被访问的逻辑地址<限长寄存器的内容}}{0被访问的逻辑地址限长寄存器的内容}{0被访问的物理地址<限长寄存器的内容}{0被访问的物理地址限长寄存器的内容}
\begin{solution}逻辑地址的大小不能超过分配空间的大小(即限长寄存器中的内容),假设分配的空间大小为N(即寄存器内容),那么逻辑地址范围就为{[}0,N-1{]},只有选项A满足。
还可以得到物理地址的越界判断式: 0+基址≤被访问的逻辑地址+基址
\end{solution}
\question 在虚拟内存管理中,地址变换机构将逻辑地址变为物理地址,形成该逻辑地址的阶段是(
)
\par\twoch{编辑}{编译}{\textcolor{red}{链接}}{装载}
\begin{solution}编译过后的程序需要经过链接才能装载,而链接后形成的目标程序中的地址也就是逻辑地址。以C语言为例:C语言经过预处理→编译→汇编→链接产生了可执行文件。其中链接的前一步,产生了可重定位的二进制的目标文件。C语言采用源文件独立编译的方法,如程序main.c、file.h和file.c,在链接的前一步生成了main.o和file.o,链接器将这些文件和库文件链接成一个可执行文件。链接阶段主要完成了重定位,形成逻辑的地址空间。
\end{solution}
\question 系统为某进程分配了4 个页框,
该进程已访问的页号序列为2,0,2,9,3,4,2,8,2,3,8,4,5,若进程要访问的下一页的页号为7,依据LRU
算法,应淘汰页的页号是()
\par\twoch{2}{3}{\textcolor{red}{4}}{8}
\begin{solution}LRU 算法
\end{solution}
