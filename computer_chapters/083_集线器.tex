\question 若有10台计算机连接到一台20Mbit/s的集线器上,则每台计算机分得的平均带宽为(
)
\par\twoch{\textcolor{red}{2Mbit/s}}{10Mbit/s}{20Mbit/s}{200Mbit/s}
\begin{solution}连接在集线器上的每台计算机平分集线器的带宽,即(20Mbit/s)/10=2Mbit/s。
\end{solution}
\question 一般来说,集线器连接的网络在拓扑结构上属于( )
\par\twoch{网状}{树形}{环形}{\textcolor{red}{星形}}
\begin{solution}集线器的作用是将多个网络端口连接在一起,也就是以集线器为中心。所以使用它的网络在拓扑结构上属于星形结构。
\end{solution}
\question X台计算机连接到一台YMbit/s的集线器上,则每台计算机分得的平均带宽为( )
\par\twoch{XMbit/s}{YMbit/s}{\textcolor{red}{Y/XMbit/s}}{XYMbit/s}
\begin{solution}集线器以广播的方式将信号从除输入端口外的所有端口输出,因此任意时刻只能有一个端口的有效数据输入,则平均带宽为Y/XMbit/s。如果此题改为X台计算机连接到一台YMbit/s的交换机上,则每台计算机分得的平均带宽为YMbit/s,这里就不必除以X了。
\end{solution}
