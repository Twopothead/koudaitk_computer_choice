\question (华中科技大学,2007年)若已知一个栈的入栈序列为1,2,3,4,其出栈序列为p1,p2,p3,p4,则p2,p4不可能是(
)
\par\twoch{2、4}{2、1}{\textcolor{red}{4、3}}{3、4}
\begin{solution}用排除法。A的操作序列为push,pop,push,pop,push,pop,push,pop。
B的操作序列为push,push,push,pop,pop,push,pop,pop。
D的操作序列为push,pop,push,psuh,pop,pop,push,pop。
只有C没有对应的操作序列,故选C。
\end{solution}
\question (武汉大学,2006年)设n个元素进栈序列是1,2,3,
,n,其输出序列是p1,p2, ,pn,若p1=3,则p2的值为( )
\par\twoch{一定是2}{一定是1}{\textcolor{red}{不可能是1}}{以上都不对}
\begin{solution}p1是3,即3出栈了,此时栈中剩下1,2,此时若出栈,就是2,不可能是1。但又因为后面还有进栈元素,因此也可能是其他元素。综上分析,可能是2,不可能是1,因此本题选C。
\end{solution}
\question 一个栈的入栈序列为1,2,3,\ldots{},n,其出栈序列是p1,p2,p3,\ldots{},pn。若p2=3,则p3可能取值的个数是(
)
\par\twoch{n-3}{n-2}{\textcolor{red}{n-1}}{无法确定}
\begin{solution}p1可能是1或2,那么栈中就可能是1或2,因此p3就可能取1或2。由于后面的入栈元素,可以入栈后即出栈,因此p3就可能是4或5或\ldots{}或n,因此除了3本身以外,其他的值均可以取到,因此可能取值的个数为
。
\end{solution}
\question (北京航空航天大学,2004年)若某堆栈的输入序列为1,2,3\ldots{}\ldots{},n,输出序列的第一个元素为n,则第i个输出元素为(
)
\par\twoch{i}{n-i}{\textcolor{red}{n-i+1}}{哪个元素无所谓}
\begin{solution}每次栈输出后,栈顶元素依次减1,所以第i个元素输出后,栈顶元素为n-i,而输出的元素为n-i+1
\end{solution}
