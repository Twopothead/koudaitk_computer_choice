\question (北京理工大学,2004年)取指令操作
\par\twoch{\textcolor{red}{受到上一条指令的操作码控制}}{受到当前指令的操作码控制}{受到下一条指令的操作码控制}{是控制器固有的功能,不需要在操作码的控制下进行}
\begin{solution}A。 只有完成上一条指令,PC才能自加,从而指向下一条指令。
\end{solution}
\question (北京理工大学,2002年)在CPU执行指令的过程中,操作数的地址是由(
)给出的
\par\twoch{程序计数器(PC)}{\textcolor{red}{指令的地址码字段}}{操作系统}{程序员}
\begin{solution}B。 指令的地址码字段给出操作数的地址。
\end{solution}
\question 执行一条一地址的加法指令共需( )次访问主存(含取指令)
\par\twoch{1}{\textcolor{red}{2}}{3}{4}
\begin{solution}B。
执行一条一地址的加法指令共需2次访问主存,一次取指令,一次取第二操作数。
\end{solution}
\question CPU在中断周期中
\par\twoch{执行中断服务程序}{\textcolor{red}{执行中断隐指令}}{与I/O设备传送数据}{处理异常情况}
\begin{solution}CPU在中断周期中执行中断隐指令,完成以下工作: 1)保护程序断点。
2)寻找中断服务程序的入口地址。 3)关中断。
\end{solution}
