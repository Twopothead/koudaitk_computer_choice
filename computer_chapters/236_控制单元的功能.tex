\question 在具有中断系统的CPU中有中断标记寄存器,它用来
\par\twoch{\textcolor{red}{向CPU发出中断请求}}{提示CPU是否进入中断周期}{开放或关闭中断系统}{以上都不对}
\begin{solution}A。 中断标志寄存器用来标志是否有中断申请,故选A。
\end{solution}
\question (中国科学院,2004年)在硬布线控制器中,时序信号采用(
)时序系统;在微程序控制器中,时序信号一般采用( )时序
\par\fourch{指令周期—机器周期—时钟周期、微周期脉冲}{机器周期—时钟周期、机器周期—时钟周期—脉冲}{机器周期—时钟周期—脉冲、机器周期—时钟周期}{\textcolor{red}{机器周期—时钟周期—脉冲、微周期—脉冲}}
\begin{solution}D。
分析:在硬布线控制器中,计算机每个指令周期划分为若干个机器周期,每个机器周期划分为若干个时钟周期(节拍),每个时钟周期设置一个或几个工作脉冲,即采用机器周期---时钟周期---脉冲三级时序系统。在微程序控制器中,通过执行微指令解释指令的执行,执行每条微指令的时间为微周期,每个微周期通过脉冲控制微命令序列执行微指令,即采用微周期---脉冲二级时序系统。
\end{solution}
\question 隐指令指( )
\par\twoch{操作数隐含在操作码中的指令}{在一个机器周期里完成全部操作的指令}{隐含地址码的指令}{\textcolor{red}{指令系统中没有的指令}}
\begin{solution}隐指令不是指令系统中一条真正的指令,其没有操作码。例如,中断隐指令就是一种不允许、也不可能为用户使用的特殊``指令''。
\end{solution}
\question 在单级中断系统中,中断服务程序执行顺序是( )。
a.保护现场;b.开中断;c.关中断;d.保存断点;
e.中断事件处理;f.恢复现场;g.中断返回
\par\twoch{\textcolor{red}{a→e→f→b→g}}{c→a→e→g}{c→d→e→f→g}{d→a→e→f→g}
\begin{solution}在单级中断系统中,中断服务程序的执行顺序为:①保存现场;②中断事件处理;③恢复现场;④开中断;⑤中断返回。
归纳总结:程序中断有单级中断和多级中断之分,单重中断在CPU执行中断服务程序的过程中不能被再打断,即不允许中断嵌套;而多重中断在执行某个中断服务程序的过程中,CPU可以去响应级别更高的中断请求,即允许中断嵌套。
解题技巧:B、C、D
3个选项的第一个任务(保存断点或关中断)都是中断隐指令中的操作,由硬件来完成,与中断服务程序无关,可以马上排除。
\end{solution}
\question 禁止中断的功能可以由( )来完成
\par\twoch{中断触发器}{\textcolor{red}{中断允许触发器}}{中断屏蔽触发器}{中断禁止触发器}
\begin{solution}当中断允许触发器为``1''时,某设备可以向CPU发出中断请求;当中断允许触发器为``0''时,不能向CPU发出中断请求。所以说设置中断允许触发器的目的就是来控制是否允许某设备发出中断请求。
\end{solution}
\question 某机有4级中断,优先级从高到低为1→2→3→4。若将优先级顺序修改,改后1级中断的屏蔽字为1011,2级中断的屏蔽字为1111,3级中断的屏蔽字为0011,4级中断的屏蔽字为0001,则修改后的优先顺序从高到低为
\par\twoch{3→2→1→4}{1→3→4→2}{\textcolor{red}{2→1→3→4}}{2→3→1→4}
\begin{solution}每个中断源对应一个屏蔽字,由多个中断屏蔽触发器组成。某个中断屏蔽触发器为``1''表示屏蔽,为``0''表示开放。通过改变中断屏蔽字可以动态地改变中断处理的次序。
解题技巧:优先级别越高,屏蔽字中``1''的个数就越多。在此题中有4级中断,则最高优先级的屏蔽字应为4个``1'',接下来``1''的个数将依次减少,据此很容易确定4级中断的中断处理次序。
\end{solution}
\question I/O设备提出中断请求的条件是
\par\twoch{一个CPU周期结束}{\textcolor{red}{I/O设备工作完成和系统允许}}{CPU开放中断系统}{总线空闲}
\begin{solution}I/O设备向CPU提出中断请求的条件是:I/O接口中的设备工作完成状态为1(D=1),中断屏蔽码为0
(MASK=0),且CPU查询中断时,中断请求触发器状态为1(INTR=1)。
简单来总结这些状态的意义,就是I/O设备工作完成并且系统允许。因此本题选B。
\end{solution}
\question 开中断和关中断两种操作都用于对( )进行设置
\par\twoch{\textcolor{red}{中断允许触发器}}{中断屏蔽寄存器}{中断请求寄存器}{中断向量寄存器}
\begin{solution}中断允许触发器:开中断、关中断。 中断屏蔽寄存器:中断屏蔽。
中断请求寄存器:提出中断请求。
中断向量寄存器:保存中断服务程序入口地址。
\end{solution}
\question (西安交通大学,2003年)中断发生时,由硬件保护并更新程序计数器(PC),而不是由软件完成,主要是为了
\par\fourch{\textcolor{red}{能进入中断处理程序并能正确返回源程序}}{节省内存}{提高处理速度}{使中断处理程序易于编制,不易出错}
\begin{solution}为了保证在中断服务程序执行完毕后能正确返回原来的程序,必须要将原来程序的断点(即程序计数器(PC)的内容)压入堆栈保存起来。
\end{solution}
\question (大连理工大学,2004年)中断响应由高至低的优先次序应为
\par\twoch{访管—程序性—机器故障}{\textcolor{red}{访管—程序性—重新启动}}{外部—访管—程序性}{程序性—I/O—访管}
\begin{solution}中断优先级是根据事件的轻重缓急来划分的,越紧急越重要的事件优先级就应该越高。所以,机器故障的优先级应是最高的,访管次之,程序性(软中断)优先级再次之,重新启动优先级最低。
\end{solution}
\question CPU在中断周期中
\par\twoch{执行中断服务程序}{\textcolor{red}{执行中断隐指令}}{与I/O设备传送数据}{处理异常情况}
\begin{solution}CPU在中断周期中执行中断隐指令,完成以下工作: 1)保护程序断点。
2)寻找中断服务程序的入口地址。 3)关中断。
\end{solution}
\question (南京航空航天大学,2001年)CPU响应非屏蔽中断请求的条件是
\par\fourch{\textcolor{red}{当前执行的机器指令结束而且没有DMA请求信号}}{当前执行的机器指令结束而且IF(中断允许)标志=1}{当前机器周期结束而且没有DMA请求信号}{当前执行的机器指令结束而且没有INT请求信号}
\begin{solution}DMA中断请求的优先级高于非屏蔽中断请求。且对于中断请求的检测一般是安排在一条指令执行过程的末尾。
\end{solution}
