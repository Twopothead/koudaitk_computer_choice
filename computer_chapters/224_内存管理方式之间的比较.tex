\question 下面哪种内存管理方法有利于程序的动态链接
\par\twoch{\textcolor{red}{分段存储管理}}{分页存储管理}{可变式存储管理}{固定式存储管理}
\begin{solution}动态链接是指在作业运行之前,并不把几个目标程序段链接起来。要运行时,先将主程序所对应的目标程序装入内存并启动运行,当运行过程中又需要调用某段程序时,才将该段(目标程序)调入内存并进行链接。可见,动态链接也要求以段作为管理的单位。
\end{solution}
\question 以下存储管理方式中,会产生内部碎片的是( )。 \ding{192}.请求分段存储管理
\ding{193}.请求分页存储管理 \ding{194}.段页式分区管理 \ding{195}.固定式分区管理
\par\twoch{\ding{192}、\ding{193}、\ding{194}}{\ding{194}、\ding{195}}{只有\ding{193}}{\textcolor{red}{\ding{193}、\ding{194}、\ding{195}}}
\begin{solution}只要是固定的分配就会产生内部碎片,其余的都产生外部碎片。如果固定和不固定同时存在(例如段页式),则看做固定。请求分段:每段的长度不同(不固定),产生外部碎片。请求分页:每页大小固定,产生内部碎片。段页式:视为固定,产生内部碎片。固定式分区管理产生的是内部碎片。
\end{solution}
