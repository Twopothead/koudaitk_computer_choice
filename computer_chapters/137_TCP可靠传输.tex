\question (华中理工大学)为保证数据传输的可靠性,TCP采用了对( )确认的机制
\par\twoch{\textcolor{red}{报文段}}{分组}{字节}{比特}
\begin{solution}TCP是面向字节流的。TCP将所要传送的报文看成是字节组成的数据流,并使每一个字节对应于一个序号。在连接建立时,双方要商定初始序号。每个TCP报文的序列号值表示该报文段中的数据部分的第一个字节的序号。TCP的确认是对接收到的报文段的最高序号表示确认,接收端返回的确认号是已收到的最高序号加1,因此确认号表示接收端期望下次收到的报文段中的第一个数据字节的序号。
\end{solution}
\question TCP的滑动窗口协议中规定重传分组的数量最多可以
\par\twoch{两倍滑动窗口的大小}{滑动窗口大小的一半}{滑动窗口的大小减1}{\textcolor{red}{等于滑动窗口的大小}}
\begin{solution}每次TCP最多能发送的分组数量为滑动窗口大小,最坏的情况下就是所有分组都丢失了,此时需要全部重传。
\end{solution}
