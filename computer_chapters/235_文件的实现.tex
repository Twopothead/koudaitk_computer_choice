\question 文件目录项中不包含的是
\par\twoch{文件名}{文件访问权限说明}{\textcolor{red}{文件控制块的物理位置}}{文件所在的物理位置}
\begin{solution}文件目录项即文件控制块通常由文件基本信息、存取控制信息和使用信息组成,而基本信息包含文件物理位置,显然在文件目录项中不包含文件控制块的物理位置的信息。
\end{solution}
\question 下面的说法中,错误的是( )。
\ding{192}.一个文件在同一系统中、不同介质上的复制文件,应采用同一种物理结构
\ding{193}.对一个文件的访问,常由用户访问权限和用户优先级共同限制
\ding{194}.文件系统采用树形目录结构后,对于不同用户的文件,其文件名应当不同
\ding{195}.为防止系统故障造成系统内文件受损,常采用存取控制矩阵方法保护文件
\par\twoch{\ding{193}}{\ding{192}、\ding{194}}{\ding{192}、\ding{194}、\ding{195}}{\textcolor{red}{全部}}
\begin{solution}\ding{192}错误:一个文件存放在磁带中通常采用连续存放,文件在硬盘上一般采用离散存放,不同的文件系统存放的方式是不一样的。\ding{193}错误:对一个文件的访问,常由用户访问权限和文件属性共同限制。\ding{194}错误:文件系统采用树形目录结构后,对于不同用户的文件,文件名可以相同或不同。\ding{195}错误:常采用备份的方法保护文件,而存取控制矩阵方法用于多用户之间的存取权限保护。
\end{solution}
\question (兰州大学,2005年)决定文件信息的逻辑块号到物理块号的对换是
\par\twoch{逻辑结构}{\textcolor{red}{物理结构}}{页表}{分配算法}
\begin{solution}文件的物理结构又称为文件的存储结构,是指文件在外存上的存储组织形式。由于文件的物理结构决定了文件信息在存储设备上的存储位置,因此从文件信息的逻辑块号(逻辑地址)到物理块号(物理地址)的对换也是由文件的物理结构决定的。
文件的物理结构主要有三种:连续分配、链接分配和索引分配。
\end{solution}
\question 下面对索引文件描述不正确的选项是
\par\twoch{索引文件和主文件配合使用}{一般来说,主文件为变长记录文件,使用索引文件是为了加快对主文件的检索速度}{\textcolor{red}{索引文件和顺序文件没有什么联系}}{可以说利用索引文件是用空间来换时间}
\begin{solution}选项A指出索引文件和主文件都不能单独使用,因为它们是相互存储的。
选项B说明了索引文件可以加快变长记录文件的检索速度。
选项C是错的,因为索引文件本身就是一个定长记录的顺序文件。
选项D说明使用索引文件能够加快对主文件的检索速度,但需要额外配置一张索引表,且每个记录都要有一索引项,因而提高了存储费用。
\end{solution}
\question 下面关于连续结构文件和链式结构文件的论述中,正确的是
\par\fourch{连续结构文件适合建立在顺序存储设备上,而不适合建立在硬盘上}{在显式链接结构文件中是在每个盘块中设置一链接指针,用于将文件的所有盘块链接起来}{连续结构文件必须采用连续分配方式,而链接结构文件和索引结构文件则都可以采用离散分配方式}{\textcolor{red}{都错}}
\begin{solution}A选项错误,磁盘既支持顺序存取,也支持随机存取,所以连续结构文件适合建立在硬盘上。
B描述的应该是隐式链接结构文件。
C错误,连续结构文件也可以采取离散分配方式,文件结构与磁盘分配方式没有直接联系。
所以选择D选项。
\end{solution}
\question 下列关于索引表的叙述中,正确的是
\par\fourch{索引表中每个记录的索引项可以有多个}{\textcolor{red}{对索引文件存取时,必须先查找索引表}}{索引表中含有索引文件的数据及其物理地址}{建立索引表的目的之一是减少存储空间}
\begin{solution}索引文件由逻辑文件和索引表组成,对索引文件存取时,必须先查找索引表。索引表中每个记录所对应的索引项只能有一个;索引表中仅含有索引文件的物理地址,并不包含数据;建立索引表是为了文件的快速查找,是一种空间换时间的做法,存储空间比其他存储方法要多一些,并不节省。
\end{solution}
\question 考虑一个文件存放在100个数据块中。文件控制块、索引块或索引信息都驻留内存。那么如果(
),不需要做任何磁盘I/O操作
\par\fourch{采用连续分配策略,将最后一个数据块搬到文件头部}{\textcolor{red}{采用单级索引分配策略,将最后一个数据块插入文件头部}}{采用隐式链接分配策略,将最后一个数据块插入文件头部}{采用隐式链接分配策略,将第一个数据块插入文件尾部}
\begin{solution}本题考查的是连续分配、链接分配和索引分配的特点,并考查它们各自插入数据块或移动数据块所需要的操作。对于选项A,采用连续分配策略,连续分配策略下是没有指针的,对每个数据块的访问都可以直接用块号寻址到,不过要把最后一个数据块搬到文件头部,首先要把最后一块读入内存,然后将倒数第二块放入到最后一块,将倒数第三块放入倒数第二块
将第一块放入到原本第二块的位置,最后才能把内存中原本的最后一块放入到第一块的位置,也就是文件的头部,读取和写入数据块都需要I/O操作,所以需要很多次磁盘I/O操作,具体次数和文件的长度有关;对于C选项,采用隐式链接分配,链接分配的指针都存放在数据块的末尾,也就是外存中,所以先要在内存中读出第一块的地址,然后依次读出后续块,直到找到最后一块,并在最后一块数据块的数据块指针中写入原来的第一块的地址,这儿需要写外存,最后在内存中改变文件首地址为原本的最后一块的地址,所以需要多次磁盘I/O操作;对于D选项,要读出最后一块需要多次磁盘I/O操作,修改原本的最后一块的指针指向原本的第一块,还要改变内存中的文件首地址为原本的第二块,最后再把新的最后一块的指针置为NULL,所以需要多次磁盘I/O操作;对于选项B,由于本题中单级索引的索引块驻留在内存,所以所有数据块的指针都在内存中,只需要在内存中重新排列这些指针相互间的位置,将最后一块的指针移动到最前面即可,不需要任何磁盘I/O操作,所以答案选B。
\end{solution}
\question (武汉理工大学,2005年)( )结构的文件最适合于随机存取的应用场合
\par\twoch{流式}{索引}{链接}{\textcolor{red}{顺序}}
\begin{solution}连续分配(顺序文件)具有随机存取功能,但不便于文件长度的动态增长。链接分配便于文件长度的动态增长,但不具有随机存取功能。索引分配既具有随机存取功能,也便于文件长度动态增长。
适合随机存取的程度总结为:连续分配>索引分配>链接分配。
\end{solution}
\question (南昌大学,2006年)采用直接存取法来读写磁盘上的物理记录时,效率最高的是
\par\twoch{\textcolor{red}{连续结构的文件}}{索引结构的文件}{链接结构文件}{其他结构文件}
\begin{solution}在连续文件方法下,只要知道文件在存储设备上的起始地址(首块号)和文件长度(总块数),就能很快地进行存取。适合随机存取的程度总结为:连续分配>索引分配>链接分配。
\end{solution}
\question (电子科技大学,2006年)文件的顺序存取是
\par\twoch{按终端号依次存取}{\textcolor{red}{按文件的逻辑号逐一存取}}{按物理块号依次存取}{按文件逻辑记录大小逐一存取}
\begin{solution}顺序存取文件是按其在文件中的逻辑顺序依次存取的,只能从头往下读。在4个选项中,只有逻辑号跟逻辑顺序的意思最接近,故本题选B。
\end{solution}
\question 下列关于索引表的叙述,( )是正确的
\par\fourch{索引表每个记录的索引项可以有多个}{\textcolor{red}{对索引文件存取时,必须先查找索引表}}{索引表中含有索引文件的数据及其物理地址}{建立索引表的目的之一是为减少存储空间}
\begin{solution}索引表每个记录的索引项只有一个,因此A错误。
对索引文件进行存取时,需要检索索引表,找到相应的表项,再利用该表项中给出的指向记录的指针值,去访问所需的记录。因此B正确。
对主文件的每个记录,在索引表中都设有一个相应的表项,用于记录该将记录的长度L及指向该记录的指针(指向该记录在逻辑地址空间的首址),因此C错误。
由于使用了索引表而增加了存储空间的开销,因此非但不会减少存储空间(此处意为存储开销),还会增加存储开销。
\end{solution}
\question 文件系统采用两级索引分配方式,如果每个磁盘块大小为1KB,每个盘块号占4字节,则在该系统中,文件的最大长度是
\par\twoch{\textcolor{red}{64MB}}{128MB}{32MB}{以上都不对}
\begin{solution}每个磁盘块大小为1KB,每个盘块号占4字节,则一个盘块可以存放1KB/4B=256个盘块号,则二级索引文件的最大长度是256×256×1KB=64MB。
\end{solution}
\question 假设有一个记录文件采用链接分配方式,逻辑记录的固定长度为100B,在磁盘上存储时采用记录成组分解技术,盘块长度为512B。如果该文件的目录项已经读入内存,要读第22个逻辑记录共需启动磁盘(
)次
\par\twoch{3}{4}{\textcolor{red}{5}}{6}
\begin{solution}第22个逻辑记录对应第5个物理块,即读入第5个物理块(22×100/512=4,余152),由于文件采用的物理结构是链接文件,因此需要从目录项所指的第1个物理块开始读取,一次读到第4块才得到第5块的物理地址,加上1次读第5块的操作,共需要启动磁盘5次。
\end{solution}
\question 某系统中,一个FCB占用64B,盘块大小为1KB,文件目录中共有3200个FCB,故查找一个文件平均启动磁盘次数为
\par\twoch{50}{64}{\textcolor{red}{100}}{200}
\begin{solution}为了找到一个目录项,平均需要调入盘块N/2次,其中N为目录文件所占用的总盘块数。每调入一个盘块即为启动磁盘一次。
从题目要求看,系统中的所有文件目录皆被存放于一个目录文件中。目录文件所占用的盘块数N可按下式计算。
N=3200/(1024/64)=200 因此,平均需要调入的盘块数为N/2=100。
\end{solution}
\question 设某文件为索引顺序文件,由5个逻辑记录组成,每个逻辑记录的大小与磁盘块的大小相等,均为512B,并依次存放在50、121、75、80、63号磁盘块上。若要存取文件的第1569逻辑字节处的信息,则要访问(
)号磁盘块
\par\twoch{3}{75}{\textcolor{red}{80}}{63}
\begin{solution}因为1569=512×3+33。所以要访问字节的逻辑记录号为3,对应的物理磁盘块号为80。故应访问第80号磁盘。
\end{solution}
\question 若某文件系统索引结点(inode)中有直接地址项和间接地址项,则下列选项中,与单个文件长度无关的因素是(
)
\par\twoch{\textcolor{red}{索引结点的总数}}{间接地址索引的级数}{地址项的个数}{文件块大小}
\begin{solution}一个文件对应一个索引结点,索引结点的总数只能说明有多少个文件,跟单个文件的长度没有关系。而间接地址索引的级数、地址项的个数和文件块大小都跟单个文件长度相关,因此4个选项中,只有A选项是与单个文件长度无关的。
\end{solution}
