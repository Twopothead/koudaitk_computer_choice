\question 当IP分组经过路由转发时,如果不被分片,则以下( ~)字段将不会改变。
Ⅰ.TTL ~ ~ ~ ~ ~ ~ ~ ~ Ⅱ.校验和 Ⅲ.DF ~ ~ ~ ~ ~ ~ ~ ~ Ⅳ.MF
\par\twoch{仅Ⅰ、Ⅱ、Ⅲ}{仅Ⅱ、Ⅲ、Ⅳ}{仅Ⅰ、Ⅲ、Ⅳ}{\textcolor{red}{仅Ⅲ、Ⅳ}}
\begin{solution}DF字段肯定不会改变;当IP分组不被分片时,MF字段也不会改变;而生存周期(TTL)字段每经过一个路由器都会改变,由于TTL字段值发生改变,导致首部校验和的值发生改变。
\end{solution}
\question 如果IPv4的分组太大,则会在传输中被分片,那么在(
~)地方将对分片后的数据报重组
\par\twoch{中间路由器}{下一跳路由器}{核心路由器}{\textcolor{red}{目的端主机}}
\begin{solution}数据报被分片后,每个分片都将独立地传输到目的地,期间有可能会经过不同的路径,而最后在目的端主机分组被重组。
\end{solution}
\question 下面的地址中,属于单播地址的是
\par\fourch{\textcolor{red}{172.31.128.255/18}}{10.255.255.255}{192.168.24.59/30}{224.105.5.211}
\begin{solution}10.255.255.255为A类地址,而主机位是全``1'',代表网内广播,为广播地址;192.168.24.59/30为C类地址,并可以知道只有后面2位为主机号,而59用二进制表示为00111011,可以知道后面两位都为1,即主机位是全``1'',代表网内广播,为广播地址;224.105.5.211为D类组播地址。
\end{solution}
\question 根据分类编制方案,总共有( ~)个A类地址
\par\twoch{254}{127}{255}{\textcolor{red}{126}}
\begin{solution}A类地址总共有7位的网络号,最多可以有128个地址,同样也需要去掉全``0''和全``1''的情况,剩下126个网络地址可以分配。
\end{solution}
\question 互联网规定的B类私有地址为
\par\fourch{172.16.0.0/16}{\textcolor{red}{172.16.0.0/12}}{172.15.0.0/16}{172.15.0.0/12}
\begin{solution}1个A类私有地址:10.0.0.0~10.255.255.255,即10.0.0.0/8。
16个B类私有地址:172.16.0.0~172.31.255.255,即172.16.0.0/12。
256个C类私有地址:192.168.0.0~192.168.255.255,即192.168.0.0/16。
\end{solution}
\question IP分组中的校验字段检查范围是
\par\twoch{整个IP分组}{\textcolor{red}{仅检查分组首部}}{仅检查数据部分}{根据网络情况的不同而不同}
\begin{solution}IP分组的校验字段仅检查分组首部信息,不包括数据部分。
\end{solution}
