\question 在CPU的状态字寄存器中,若符号标志位SF为``1'',表示运算结果是
\par\twoch{正数}{负数}{非正数}{\textcolor{red}{不能确定}}
\begin{solution}D。
状态字寄存器用来存放PSW,PSW包括两个部分:一是状态标志,如进位标志(C)、结果为零标志(Z)等,大多数指令的执行将会影响到这些标志位;二是控制标志,如中断标志、陷阱标志等。
SF符号标志位,当运算结果最高有效位是1,SF==1;否则,SF==0。当此数是有符号数时,该数是个负数;当此数为无符号数时,SF的值没有参考价值。
\end{solution}
\question 下面有关程序计数器(PC)的叙述中,错误的是
\par\fourch{每条指令执行后,PC的值都会被改变}{PC的值由CPU在执行指令过程中进行修改}{\textcolor{red}{条件转移指令时,PC的值总是修改为转移目标指令的地址}}{PC的位数一般和存储器地址寄存器(MAR)的位数一样}
\begin{solution}C。
当执行指令(包括转移指令)时,CPU将自动修改PC的内容,即每执行一条指令PC增加一个量,这个量等于指令所含的字节数,以便使其保持的总是将要执行的下一条指令的地址,故A正确。
在程序开始执行前,必须将它的起始地址,即程序的第一条指令所在的内存单元地址送入PC。当执行指令时,CPU将自动修改PC内容,使其保存的总是将要执行的下一条指令的地址,故B正确。
当执行到转移指令时,对于无条件转移或调用、返回等指令,则PC的值直接修改为目标指令地址;对于条件转移(分支)指令,则必须根据前面指令或当前指令执行的结果标志,确定是把转移目标地址还是把下一条指令地址送到PC。所以转移指令时,PC的值并不总是直接修改为转移目标指令的地址,所以C错误。
程序计数器的位数取决于CPU能够访问的程序存储空间的大小,一般情况下为主存储器,所以程序计数器的位数与主存储器地址的位数相等,而主存储器地址取决于主存储器的容量。也就是说,程序计数器(PC)的位数跟存储器地址寄存器(MAR)的位数相等,所以D正确。
\end{solution}
\question 累加器中
\par\fourch{没有加法器功能,也没有寄存器功能}{\textcolor{red}{没有加法器功能,有寄存器功能}}{有加法器功能,没有寄存器功能}{有加法器功能,也有寄存器功能}
\begin{solution}B。
在中央处理器CPU中,累加器是一种暂存器,用来存储计算所产生的中间结果。如果没有累加器这样的寄存器,那么在每次计算(加法,乘法,移位等)后就必须要把结果写回到内存中,然后也需再读回来。而从内存读的速度远不如ALU从累加器读取数据的速度。故本题选B。
\end{solution}
\question 下列部件中不属于控制部件的是
\par\twoch{指令寄存器}{操作控制器}{程序计数器}{\textcolor{red}{状态条件寄存器}}
\begin{solution}D。
CPU控制器主要由3个部件组成:指令寄存器、程序计数器和操作控制器。状态条件寄存器通常属于运算器的部件,保存由算术指令和逻辑指令运行或测试的结果建立的各种条件码内容,如运算结果进位标志(C)、运算结果溢出标志(V)、运算结果为零标志(Z)、运算结果为负标志(N)、中断标志(I)、方向标志(D)和单步标识等。
\end{solution}
\question 下列寄存器中,汇编语言程序员可见的是( )
\par\twoch{存储器地址寄存器(MAR)}{\textcolor{red}{程序计数器(PC)}}{存储器数据寄存器(MDR)}{指令寄存器(IR)}
\begin{solution}汇编语言程序员可见寄存器是指程序员可以通过程序去访问的寄存器(如通用寄存器组、PC等)。IR、MAR、MDR是CPU的内部工作寄存器,在程序执行的过程中是自动赋值的,程序员无法对其操作,或者称为用户不可见。而程序计数器中存放的是下一条需要执行的指令,因而程序员可以通过转移指令、调动子程序等指令来改变其内容,故PC可见。
\end{solution}
