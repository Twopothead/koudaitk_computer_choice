\question 某磁盘组的每个盘面上有200个磁道,格式化时每个磁道被分成4个扇区,整个盘组共有8000个物理块,那么该盘组的磁盘数为
\par\twoch{4}{\textcolor{red}{5}}{8}{10}
\begin{solution}每个盘面的物理块数=200×4=800个,盘面数=8000/800=10,一张盘有两个盘面,磁盘数=10/2=5。
\end{solution}
\question 现有一个容量为10GB的磁盘分区,磁盘空间以簇(Cluster)为单位进行分配,簇的大小为4KB,若采用位图法管理该分区的空闲空间,即用一位(bit)标识一个簇是否被分配,则存放该位图所需簇的个数为(
)
\par\twoch{\textcolor{red}{80}}{320}{80K}{320K}
\begin{solution}先计算该磁盘空间有多少簇。容量为10GB,簇大小为4KB,那么就有10GB/4KB=10240MB/4KB=2560K。用一位bit标识一个簇是否被分配,那么就需要的空间为2560Kbit=320KB。每个簇大小为4KB,那么需要的簇个数=320KB/4KB=80个。
\end{solution}
