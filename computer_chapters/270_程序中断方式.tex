\question 在中断处理中,输入/输出中断是指( )。 \ding{192}.设备出错 \ding{193}.数据传输结束
\par\twoch{\ding{192}}{\ding{193}}{\textcolor{red}{\ding{192}和\ding{193}}}{都不是}
\begin{solution}I/O中断是指由输入/输出设备引起的中断,如数传输结束、设备出错等。
\end{solution}
\question 在采用中断方式进行打印控制时,在打印控制接口电路和打印部件之间交换的信息不包括以下的
\par\twoch{打印字符点阵信息}{打印控制信息}{打印机状态信息}{\textcolor{red}{中断请求信号}}
\begin{solution}首先我们应该把信息通路理清楚。
CPU通过I/O总线连接到I/O接口上,I/O接口再通过通信总线连接到打印部件上。
那么我们再看中断请求信号,中断请求信号是由I/O接口向CPU发出的,那么它就不可能是通信总线上传输的信息,因此本题选D。A、B、C信息分别是通信总线上传输的数据信息,控制信息和状态信息。
\end{solution}
\question (西南交通大学,2005年)在程序中断方式中,中断的概念是指
\par\twoch{暂停正在运行的程序}{暂停对内存的访问}{暂停CPU运行}{\textcolor{red}{I/O设备的输入或输出}}
\begin{solution}程序中断方式是当主机启动外设后,无须等待查询,而是继续执行原来的程序,外设在做好输入/输出准备时,向主机发出中断请求,主机接到请求后就暂时中止原来执行的程序,转去执行中断服务程序对外部请求进行处理,在中断处理完毕后返回原来的程序继续执行。不过,这里要求转移到中断服务子程序的请求是由外部设备发出的。即此处的中断是指I/O设备的输入或输出。
\end{solution}
\question 内部异常(内中断)可分为故障(fault)、陷阱(trap)和终止(abort)三类。下列有关内部异常的叙述中,错误的(
)
\par\fourch{\textcolor{red}{内部异常的产生与当前执行指令相关}}{内部异常的检测由CPU 内部逻辑实现}{内部异常的响应发生在指令执行过程中}{内部异常处理的返回到发生异常的指令继续执行}
\begin{solution}内部异常概念。
\end{solution}
