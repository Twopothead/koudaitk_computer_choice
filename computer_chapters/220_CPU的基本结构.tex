\question 下列部件中不属于执行部件的是
\par\twoch{\textcolor{red}{控制器}}{存储器}{运算器}{外部设备}
\begin{solution}A。
一台数字计算机基本上可以划分为两大部分:控制部件和执行部件。控制器就是控制部件,而运算器、存储器、外部设备相对控制器来说就是执行部件。控制部件与执行部件的一种联系就是通过控制线。控制部件通过控制线向执行部件发出各种控制命令,通常这种控制命令叫做微命令,而执行部件接受微命令后所执行的操作就叫做微操作。控制部件与执行部件之间的另一种联系就是反馈信息。执行部件通过反馈线向控制部件反映操作情况,以便使得控制部件根据执行部件的状态来下达新的微命令,这也叫做``状态测试''。
\end{solution}
\question (大连理工大学,2004)在计算机系统中表征系统运行时序状态的部件是
\par\twoch{程序计数器}{累加计数器}{中断计数器}{\textcolor{red}{程序状态字}}
\begin{solution}D。 表征系统运行时序状态的部件是程序状态字。
\end{solution}
\question 下列有关控制器各部件功能的描述中,错误的是
\par\fourch{控制单元是其核心部件,用于对指令操作码译码并生成控制信息}{PC称为程序计数器,用于存放下一条指令所在单元的地址}{通过将PC按当前指令长度增量,可实现指令的按序执行}{\textcolor{red}{IR称为指令寄存器,用来存放当前指令的操作码}}
\begin{solution}D。 前三个选项都正确,D错误,指令寄存器(IR
)用来保存当前正在执行的一条指令,而不光是操作码。 【扩展】
本题B选项如果改为:
无论IR中为何指令,PC中存放的肯定都是下一条将要执行的指令的地址。
这就不对了,如果IR中是转移指令,那么下一条要执行的指令地址就不是存于PC中的地址。
这里其实涉及对语言逻辑性的理解,举个例子:
菜篮子用来装菜,是对的陈述。但无论何时,菜篮子中的东西一定是菜,就是不对的陈述了。
\end{solution}
\question (西南交通大学)主机中能对指令进行译码的器件是( )
\par\twoch{ALU}{运算器}{\textcolor{red}{控制器}}{存储器}
\begin{solution}运算逻辑部件执行定点或浮点的算术运算操作、移位操作及逻辑操作,也可执行地址的运算和转换。存储器主要用来存储数据。控制器主要负责指令译码,并且发出为完成每条指令所要执行的各个操作的控制信号。
\end{solution}
\question CPU包括( )。 I.ALU II.寄存器 III.CU IV.Cache
\par\twoch{I和II}{I和III}{\textcolor{red}{I、II和III}}{I、II、III和IV}
\begin{solution}CPU包括运算逻辑部件(ALU)、寄存器部件和控制部件(CU)等。Cache是高速缓存,不属于CPU的组成部分。
\end{solution}
