\question (武汉大学)关于无连接的通信,下面的描述中正确的是
\par\fourch{由于为每一个分组独立地建立和释放逻辑连接,所以无连接的通信不适合传送大量的数据}{由于通信对方和通信线路都是预设的,所以在通信过程中无须任何有关连接的操作}{\textcolor{red}{目标的地址信息被加在每个发送的分组上}}{无连接的通信协议UDP不能运行在电路交换或租用专线网络上}
\begin{solution}在无连接的通信中,不需要建立和释放逻辑连接,故A选项不正确;如果通信对方和通信线路都是预设的,那么无连接的协议也就毫无意义了;在D选项中,UDP是建立在IP基础之上的,任何支持IP的网络都能进行UDP通信,也不正确。
\end{solution}
\question 一个UDP用户数据报的数据字段为8192B。在链路层要使用以太网来传输,那么应该分成(
)IP数据片
\par\twoch{3个}{4个}{5个}{\textcolor{red}{6个}}
\begin{solution}以太网的帧的最大数据负载是1500B,IP首部长度为20B。所以每个分片的数据字段长度为1480B,所以需要6个分片来传输该数据报。
\end{solution}
\question 下列说法正确的是( )。
Ⅰ.传输层是OSI模型的第四层,是整个网络协议体系的核心
Ⅱ.传输层传输数据只能在源主机和目的主机之间进行点到点的传输
Ⅲ.TCP和UDP是典型的传输层协议,TCP是面向连接的,而UDP是面向无连接的
Ⅳ.TCP连接进行流量控制和拥塞控制,而UDP既不进行流量控制,又不进行拥塞控制
\par\twoch{Ⅰ、Ⅱ、Ⅲ}{Ⅰ、Ⅱ、Ⅳ}{\textcolor{red}{Ⅰ、Ⅲ、Ⅳ}}{Ⅱ、Ⅲ、Ⅳ}
\begin{solution}传输层实现的是端到端的传输,即实现进程之间的通信,故Ⅱ错误,Ⅰ、Ⅲ、Ⅳ正确。
\end{solution}
\question 下列关于TCP和UDP的描述正确的是( )。 Ⅰ.TCP和UDP都是无连接的
Ⅱ.TCP是无连接的,UDP是面向连接的
Ⅲ.TCP适用于可靠性较差的网络,UDP适用于可靠性较高的网络
Ⅳ.TCP适用于可靠性较高的网络,UDP适用于可靠性较差的网络
\par\twoch{Ⅱ、Ⅳ}{Ⅱ、Ⅲ}{Ⅳ}{\textcolor{red}{Ⅲ}}
\begin{solution}TCP是面向连接的,UDP是无连接的,故Ⅰ、Ⅱ错误。由于TCP是可靠的传输,即使网络本身性能比较差,TCP也能保证较好的性能,故适用于可靠性比较差的网络;而UDP是不可靠的传输,如果网络本身还不可靠,就会造成错误太多,故UDP适用于可靠性较高的网络,Ⅲ正确,Ⅳ错误。
\end{solution}
\question 下列关于TCP和UDP的描述,正确的是
\par\twoch{TCP和UDP都是无连接的}{TCP是无连接的,UDP是面向连接的}{\textcolor{red}{TCP适用于可靠性较差的网络,UDP适用于可靠性较高的网络}}{TCP适用于可靠性较高的网络,UDP适用于可靠性较差的网络}
\begin{solution}显然A、B错误。由于TCP是可靠的传输,所以适用于可靠性比较差的网络;而UDP是不可靠的传输,如果网络本身还不可靠,就会造成错误太多,所以UDP适用于可靠性较高的网络。
\end{solution}
\question 下列关于UDP协议的叙述中,正确的是( )。 I. 提供无连接服务 II.
提供复用/分用服务 III. 通过差错校验,保障可靠数据传输
\par\twoch{仅I}{\textcolor{red}{仅I、II}}{仅II、III}{I、II、III}
\begin{solution}考察UDP协议的特征,首先很容易排出第三个,UDP确实有差错检测机制,这种机制仅仅是为了从安全角度考虑,接收方通过检测,如果没错误就接受,如果有错误就选择相应的措施。但这并不代表它能保证可靠的数据传输,因为UDP是基于无连接的服务,它提供的是一种尽力而为的服务。另外UDP提供了复用和分用功能,复用和分用都是通过端口号来实现的。
【总结】UDP的复用、分解与端口UDP软件应用程序之间的复用与分解都要通过端口机制来实现。每个应用程序在发送数据报之前必须与操作系统协商以获得协议端口和相应的端口号。UDP分解操作:从IP层接收了数据报之后,根据UDP的目的端口号进行分解操作。
\end{solution}
