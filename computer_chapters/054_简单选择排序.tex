\question (北京航空航天大学,2000年)下面给出的4种排序方法中,排序过程中的比较次数与序列初始状态无关的是(
)
\par\twoch{\textcolor{red}{选择排序法}}{插入排序法}{快速排序法}{堆积排序法}
\begin{solution}选择排序法主要是两个循环,第一个循环是遍历该序列,第二个从无序序列中挑出一个最小的元素,然后和无序序列的第一个元素进行交换,可以看出,两层循环的执行次数和初始序列没有关系。
\end{solution}
\question (武汉大学,2005年)设线性表中每个元素有两个数据项K1和K2,现对线性表按下列规则进行排序:先看数据项K1,K1值小的在前,大的在后;在K1值相同的情况下,再看数据项K2,K2值小的在前,大的在后。满足这种要求的排序方法是(
)
\par\fourch{先按K1值进行直接插入排序,再按K2值进行简单选择排序}{先按K2值进行直接插入排序,再按K1值进行简单选择排序}{先按K1值进行简单选择排序,再按K2值进行直接插入排序}{\textcolor{red}{先按K2值进行简单选择排序,再按K1值进行直接插入排序}}
\begin{solution}若先按K1值排序后,再按K2值排序,那么就会打乱原先K1值的次序,这不符合题目中K1优先的要求,因此排除A和C。
因此需要先进行K2的排序,在K1值相等的情况下,要保持原来K2值的次序,即要求进行K1值排序的算法是稳定的,由于直接插入排序是稳定的,简单选择排序是不稳定的,因此应该先按K2值进行简单选择排序,再按K1值进行直接插入排序。
\end{solution}
\question (北京师范大学,2004年)用某种排序方法对线性表\{24,88,21,48,15,27,69,35,20\}进行排序时,元素序列的变化情况如下:
(1)24,88,21,48,15,27,69,35,20 (2)20,15,21,24,48,27,69,35,88
(3)15,20,21,24,35,27,48,69,88 (4)15,20,21,24,27,35,48,69,88
则所采用的排序方法是( )
\par\twoch{\textcolor{red}{快速排序}}{选择排序}{希尔排序}{归并排序}
\begin{solution}如果是选择排序,则在4轮排序过程中无法得到最后的排序结构。如果是希尔排序不可能在第一步将20换到第一位。同理也不是归并排序。这4次过程中是子序列同时进行的快速排序
\end{solution}
