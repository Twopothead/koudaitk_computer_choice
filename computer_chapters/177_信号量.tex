\question 用P、V操作可以解决( )互斥问题
\par\twoch{某些}{一个}{\textcolor{red}{一切}}{大多数}
\begin{solution}P、V操作可以解决一切互斥问题。
\end{solution}
\question 下列关于PV操作描述正确的是
\par\fourch{PV操作是原语,原语是一串可断续执行的指令序列}{执行P操作或V操作后,一定会引起进程阻塞或进程唤醒}{PV操作是现代操作系统常用的进程同步手段}{\textcolor{red}{执行V操作后,被唤醒的进程变为就绪状态}}
\begin{solution}PV操作是一对原语,原语是指不可中断的一串指令序列,A选项错误。
P操作的含义是进程申请某资源,如果当前有足够的空闲资源,则会将资源分配给进程,进程继续执行。只有当申请资源得不到满足时,才会导致进程阻塞。V操作的含义是释放某资源,如果此时存在等待该资源的进程时,会唤醒等待的进程,而如果没有需要唤醒的进程,则仅仅释放资源即可。因此B选项错误。
PV操作是低级进程通信机制,效率很低,现代操作系统往往不采用这种方法来实现进程同步。C选项错误。
执行V操作后,唤醒的进程会被插入到就绪队列中,等待调度执行,D选项正确。
★这里要注意被唤醒进程的状态是就绪态而不是执行态。V操作唤醒之后的进程,因为等待的事件发生,只需要得到处理机就可以执行,因此变为就绪态,被插入到就绪队列等待调度。
\end{solution}
\question 对于两个并发进程,设置互斥信号量为mutex(初值为1),若mutex=-1,则表示
\par\twoch{没有进程进入临界区}{有一个进程进入临界区,另一个进程撤销}{\textcolor{red}{有一个进程进入临界区,另一个进程等待进入}}{两个进程进入临界区}
\begin{solution}互斥信号量初值为1,表示同时只允许1个进程访问临界资源。当有1个进程提出访问临界资源请求时,执行P操作,互斥信号量减1,变为0,同时该进程进入临界区。如果另一个进程此时也请求访问临界资源,则同样执行P操作,由于互斥信号量执行P操作之前的值为0,执行过P操作后,信号量值变为-1,小于0,不能允许进程访问临界资源,将其阻塞并加入阻塞队列中。
因此,mutex=-1时,表示有一个进程进入临界区,另一个进程等待进入。
\end{solution}
\question 当一个进程因在互斥信号量上执行V操作而导致唤醒另一个进程时,则互斥信号量现在的取值为
\par\twoch{大于0}{小于0}{大于等于0}{\textcolor{red}{小于等于0}}
\begin{solution}V操作能够唤醒另一个进程,表明执行过V操作之后,在等待队列中或许还存在等待的进程,或者被唤醒的进程是等待队列中的最后一个。因此信号量取值为负数或刚好为0,本题选择D选项。
有的题也会对P操作考查类似的问题,下面举一个简单例子。
如果对信号量S执行P操作后,则进入等待队列的条件就是当前没有资源能够分配给进程,因此反映在信号量的值上就是在P操作之前信号量的值已经小于等于0,表示刚好没有资源且无等待进程或者没有资源但还有其他等待进程。因此执行P操作前,信号量的值小于等于0,执行过P操作之后,信号量的值小于0。
\end{solution}
\question 若一个信号量的初值为3,经过多次P、V操作之后当前值为-1,则表示等待进入临界区的进程数为
\par\twoch{\textcolor{red}{1}}{2}{3}{4}
\begin{solution}信号量是一种整型的特殊变量,只有初始化和P、V操作可以改变其值。通常,信号量的初值表示可以使用资源的总数。当信号量为0时,表示资源已经分配完;当信号量为负值时,表示有进程正在等待分配资源,等待的进程数就是信号量的绝对值。
\end{solution}
\question (上海交通大学,2005年)用PV操作实现进程同步,信号量的初值
\par\twoch{-1}{0}{1}{\textcolor{red}{由用户决定}}
\begin{solution}信号量初值应根据具体情况来确定,由用户决定。
\end{solution}
\question 设与某资源相关联的信号量初值为3,当前值为1。若M表示该资源的可用个数,N表示等待该资源的进程数,则M、N分别是(
)
\par\twoch{0、1}{\textcolor{red}{1、0}}{1、2}{2、0}
\begin{solution}首先应该明确资源型信号量的含义。资源型信号量可以用来表示某资源的当前可用数量,初值与对应资源的初始数量相同,题目中信号量初值为3,表示该资源初始时有3个。
信号量当前值K大于0时,表示此资源还有K个资源可用,题目中信号量当前值为1,表示还有1个可用资源,M应该为1。由于还存在可用资源,所以此时不应存在等待该资源的进程,N应当为0。因此答案选B。
【总结】
如果题目改一下,变成信号量当前值K小于0时,则表示此资源有\textbar{}K\textbar{}个进程在等待该资源(如果每个进程只请求1个资源),此时可用资源M为0,等待进程N应该为\textbar{}K\textbar{}。
\end{solution}
